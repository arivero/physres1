% Options for packages loaded elsewhere
\PassOptionsToPackage{unicode}{hyperref}
\PassOptionsToPackage{hyphens}{url}
\documentclass[
]{article}
\usepackage{xcolor}
\usepackage{amsmath,amssymb}
\setcounter{secnumdepth}{-\maxdimen} % remove section numbering
\usepackage{iftex}
\ifPDFTeX
  \usepackage[T1]{fontenc}
  \usepackage[utf8]{inputenc}
  \usepackage{textcomp} % provide euro and other symbols
\else % if luatex or xetex
  \usepackage{unicode-math} % this also loads fontspec
  \defaultfontfeatures{Scale=MatchLowercase}
  \defaultfontfeatures[\rmfamily]{Ligatures=TeX,Scale=1}
\fi
\usepackage{lmodern}
\ifPDFTeX\else
  % xetex/luatex font selection
\fi
% Use upquote if available, for straight quotes in verbatim environments
\IfFileExists{upquote.sty}{\usepackage{upquote}}{}
\IfFileExists{microtype.sty}{% use microtype if available
  \usepackage[]{microtype}
  \UseMicrotypeSet[protrusion]{basicmath} % disable protrusion for tt fonts
}{}
\makeatletter
\@ifundefined{KOMAClassName}{% if non-KOMA class
  \IfFileExists{parskip.sty}{%
    \usepackage{parskip}
  }{% else
    \setlength{\parindent}{0pt}
    \setlength{\parskip}{6pt plus 2pt minus 1pt}}
}{% if KOMA class
  \KOMAoptions{parskip=half}}
\makeatother
\setlength{\emergencystretch}{3em} % prevent overfull lines
\providecommand{\tightlist}{%
  \setlength{\itemsep}{0pt}\setlength{\parskip}{0pt}}
\usepackage{bookmark}
\IfFileExists{xurl.sty}{\usepackage{xurl}}{} % add URL line breaks if available
\urlstyle{same}
\hypersetup{
  pdftitle={From Newton to the Path Integral},
  hidelinks,
  pdfcreator={LaTeX via pandoc}}

\title{From Newton to the Path Integral}
\author{}
\date{}

\begin{document}
\maketitle
\begin{abstract}
This paper develops a single structural thesis across classical and
quantum theory: physically meaningful laws arise as controlled limits of
composable local refinements. We begin with Newton's polygonal
approximation of central-force motion and its limit to continuous
dynamics, then re-express the same logic in modern variational form
through additive action functionals. We treat the path integral as a
composition law over refined time slices, not as an isolated quantum
postulate, and we frame deformation quantization and renormalization as
two mathematically distinct control mechanisms for limit consistency.
The narrative is constructive: each stage retains the previous one as a
limiting or compatibility condition rather than replacing it. Within
this architecture we reserve a dedicated role for point-like
(Dirac-supported) probes in weak formulations of the action principle,
emphasizing where they are mathematically valid and where regularization
is mandatory. The result is a staged program from Newtonian limit
methods to quantum amplitudes in which the classical equations are
recovered as stationary limits of a broader compositional framework.
\end{abstract}

\section{1. Introduction}\label{introduction}

The historical and technical problem addressed here is not merely ``how
to quantize,'' but ``how to define a stable continuum theory from
iterative refinement.'' The paper therefore treats Newtonian mechanics,
action principles, path integration, deformation quantization, and
renormalization as parts of one continuity problem.

The first anchor is Newton's geometric method in central-force motion:
replace a curve by a sequence of short segments, impose a local update
rule, and pass to a limit while controlling what is meant by
``vanishing'' quantities. In modern language, the key object is not a
smallest geometric piece but a refinement procedure with invariant
content {[}Newton1687{]}.

The second anchor is the action formulation. Action is additive under
temporal partitioning, and that additivity is exactly the algebraic
structure needed to compare coarse and fine descriptions. This creates
the bridge to quantum composition: if local contributions compose
multiplicatively while the underlying functional is additive,
exponential weighting is structurally natural {[}Dirac1933{]}
{[}Feynman1948{]}.

The long-standing foundational tension can be phrased as a Zeno-style
refinement paradox: a refinement description is an \emph{infinite-limit}
construction, and the limit is not automatically unique or even defined
once the refined objects become singular (Dirac-supported probes,
distributional derivatives) or once intermediate quantities diverge.
Classical mechanics often hides this by treating ``send the refinement
parameter to zero'' as an axiomatically legitimate operation. The
program pursued here is instead: keep refinement explicit, isolate where
limit-taking needs extra control, and treat \textbf{quantization} and
\textbf{renormalization} as two distinct mechanisms for making refined
composition stable when the naive Newtonian limit is not rigorous as
written.

\texttt{Heuristic\ H0.2\ (Concrete\ failure\ modes\ of\ naive\ refinement-to-zero).}
Three recurring obstructions that make ``refine \(\to 0\)'' nontrivial
in practice are: 1. \textbf{Singular probes:} point-supported variations
and corners/impulses force distributional weak forms (mollifiers and
contact terms). 2. \textbf{Non-uniqueness:} refinement/composition can
admit multiple classically equivalent but quantum-distinct schemes
(ordering/discretization choices), requiring an explicit equivalence or
control map. (Minimal witness: time-slicing \(H=pq\) can yield
\(-\hat p\hat q\) vs \(-\hat q\hat p\), differing by \(O(\hbar)\);
requiring unitarity selects a symmetric (half-density) convention.) 3.
\textbf{Divergence:} some refinement limits do not converge without
subtraction/parameter flow (renormalization). (Toy witness: the
derivative exists only after subtracting a \(1/\varepsilon\) divergence
in the difference quotient.) This manuscript treats these as
limit-control problems rather than as postulates about ``nature at the
smallest scale.''

\texttt{Heuristic\ H0.2a\ (No\ Lebesgue\ measure\ on\ path\ space).} In
infinite-dimensional spaces there is no nontrivial translation-invariant
\(\sigma\)-finite Borel measure (no Lebesgue/Haar measure)
{[}Velhinho2017InfDimMeasure{]}, so the formal symbol \(Dq\) in a path
integral cannot be interpreted as an ordinary ``Lebesgue measure on
trajectories.'' Therefore the slogan ``refine the time slicing,
integrate over paths, and send \(\Delta t\to0\)'' is not a raw Newtonian
limit statement but a definition-by-refinement that must specify
normalization and, when singularities are present, regulator/subtraction
rules.

\texttt{Heuristic\ H0.3\ (Constants\ as\ control\ parameters\ for\ compatibility\ limits).}
In this program, \(\hbar\), \(c\), and \(G\) can be read as control
parameters for three distinct compatibilities: \(\hbar\to0\) recovers
classical stationarity from oscillatory composition, \(c\to\infty\)
recovers Galilean/Newtonian kinematics from Lorentz-compatible
refinement, and \(G\to0\) switches off geometric backreaction (with
\(\hbar,c,G\) together defining the Planck scale where quantum and
gravitational refinement controls meet). \texttt{Derivation\ D0.2} below
gives a concrete example of a \(c\to\infty\) passage requiring an
explicit subtraction convention.

The third anchor is methodological. In this manuscript, deformation
quantization and renormalization are not presented as detached
specialist topics. They are two ways to control limits: 1. Deformation
quantization controls the classical-to-quantum passage through algebraic
deformation and recovery of Poisson structure in the small-parameter
limit {[}Landsman1998{]} {[}Connes1994{]}. 2. Renormalization controls
divergent refinement procedures by regulator-dependent intermediate
steps and regulator-independent observables {[}ConnesKreimer2000{]}.

Section 2 fixes the formal vocabulary and claim taxonomy used in later
sections. It also narrows one foundational ambiguity: the paper does not
assume that continuum limits are ontological statements about nature. It
assumes only that they are operational definitions of stable predictive
objects. This narrowed statement will be stress-tested in later
sections.

\textbf{Contributions (what is new here).} 1. A refinement/composition
reading of the Newton \(\to\) action \(\to\) kernel chain in which each
stage is retained as a compatibility condition, not replaced. 2. An
intrinsic half-density formulation of the composition law for
propagators, separating coordinate-free kernel composition from
scalarization conventions. 3. A semigroup-closure derivation showing the
short-time normalization exponent \(t^{-d/2}\) is forced by composition
(the ``square-root Jacobian''). 4. A refinement-compatibility framing of
renormalization in which RG invariance is the consistency condition
demanded by divergent refinement limits. 5. A fully explicit ``RG
appears before QFT'' computation (2D delta/contact interaction) included
as an appendix-level witness.

\section{2. Notation and Claim
Taxonomy}\label{notation-and-claim-taxonomy}

\textbf{Dimension bookkeeping.} Throughout Sections 2--7, \(d\) denotes
the dimension of the manifold being integrated over in the composition
law (typically configuration-space/spatial dimension in nonrelativistic
kernels). When we write field-theory-style spacetime integrals, we will
denote spacetime dimension by \(D\) to avoid conflating it with the
composition-variable dimension.

\subsection{2.1 Core Objects}\label{core-objects}

Let \(q:[t_i,t_f]\to \mathbb{R}^d\) be a configuration-space trajectory
and \(\mathcal{L}(q,\dot q,t)\) a Lagrangian density. Define the action:

\[
S[q] = \int_{t_i}^{t_f} \mathcal{L}(q,\dot q,t)\,dt.
\]

For a partition \(t_i=t_0< t_1<\cdots<t_N=t_f\) with
\(\Delta t_k=t_{k+1}-t_k\), define the discrete action functional:

\[
S_N[q] = \sum_{k=0}^{N-1} \mathcal{L}\!\left(q_k,\frac{q_{k+1}-q_k}{\Delta t_k},t_k\right)\Delta t_k.
\]

For planar central motion \(q=(r,\theta)\), define areal velocity and
angular momentum:

\[
\dot A = \frac{1}{2}r^2\dot\theta,\qquad
L_{\mathrm{ang}} = m r^2\dot\theta = 2m\dot A.
\]

These definitions are used as the Newtonian-to-variational bridge in
Section 3 and Section 4.

\subsection{2.2 Weak-Form Preliminaries for Point-Like
Probes}\label{weak-form-preliminaries-for-point-like-probes}

Let \(\eta\in C_c^\infty((t_i,t_f);\mathbb{R}^d)\) be a smooth compactly
supported test variation. The first variation is written
\(\delta S[q;\eta]\), and stationarity means \(\delta S[q;\eta]=0\) for
all admissible \(\eta\).

To model point-supported probes later, introduce a mollifier family
\(\rho_\varepsilon\) with \(\rho_\varepsilon \rightharpoonup \delta\) in
distributions as \(\varepsilon\to 0^+\). Any use of Dirac-supported
variations in this manuscript is understood as a mollified limit unless
explicitly labeled heuristic.

\subsection{2.3 Claim Taxonomy}\label{claim-taxonomy}

Every substantive claim is marked by one of: 1. \texttt{Proposition}:
statement intended as mathematically valid under explicit assumptions.
2. \texttt{Derivation}: explicit calculation from stated premises. 3.
\texttt{Heuristic}: physically motivated bridge that is not presented as
full proof.

\subsection{2.4 Seed Claims for the
Program}\label{seed-claims-for-the-program}

\texttt{Proposition\ P0.1\ (Additive\ refinement\ structure).} Given a
partition of \([t_i,t_f]\), the discrete action \(S_N\) is additive over
concatenated subintervals by construction. Therefore action is a natural
candidate for refinement comparison.

\texttt{Derivation\ D0.1\ (Composition-compatible\ exponential\ form).}
Suppose a weight map \(W\) on time-sliced paths satisfies: 1.
\(W[\gamma_1\!\circ\!\gamma_2]=W[\gamma_1]W[\gamma_2]\) for composable
segments. 2. \(\log W\) is local in the slice contributions. 3. The
corresponding additive functional is proportional to \(S_N\) in the
refinement limit.

Then there exists a scale \(\kappa\) and constant \(c_0\) such that, up
to normalization, \(W[\gamma]\propto \exp(c_0 S[\gamma]/\kappa)\).
Choosing \(c_0=i\) and \(\kappa=\hbar\) recovers the standard
oscillatory quantum weighting form.

\texttt{Heuristic\ H0.1\ (Classical\ recovery\ as\ concentration).} When
the phase scale is small relative to action variation, dominant
contributions concentrate near stationary-action trajectories. This is
the structural claim later made precise through stationary-phase
analysis.

\texttt{Derivation\ D0.2\ (Nonrelativistic\ limit\ as\ a\ controlled\ \textbackslash{}(c\textbackslash{}to\textbackslash{}infty\textbackslash{})\ subtraction).}
Consider the relativistic free-particle action written with an explicit
speed-of-light parameter: \[
S_{\mathrm{rel}}[q]
=-mc^2\int_{t_i}^{t_f} dt\;\sqrt{1-\frac{\|\dot q(t)\|^2}{c^2}}.
\] For \(\|\dot q\|\ll c\), expand the square root: \[
S_{\mathrm{rel}}[q]
=\int_{t_i}^{t_f} dt\left(-mc^2+\frac12 m\|\dot q(t)\|^2+O(c^{-2})\right).
\] The term \(-mc^2(t_f-t_i)\) diverges as \(c\to\infty\). Classically
it is inert (adding a constant to \(\mathcal L\) does not change the
Euler-Lagrange equations), so one may subtract it as an allowed additive
counterterm to obtain a finite \(c\to\infty\) limit: the Newtonian
kinetic action \(\int \frac12 m\|\dot q\|^2 dt\) plus higher-order
relativistic corrections. In quantum amplitudes, the same subtraction
corresponds to an overall phase \(e^{-imc^2(t_f-t_i)/\hbar}\). This
remark is at the particle-mechanics level; field-theory and gravity
effects of constant terms (vacuum energy) are a separate issue not
addressed here.

\subsection{2.5 Scope Boundary
Established}\label{scope-boundary-established}

This section fixes notation and methodological boundaries: 1. Historical
statements are used only as source-anchored motivation. 2. Mathematical
validity requires explicit assumptions and, for singular objects,
explicit regularization. 3. Quantum and QFT-level statements are
introduced only after the composition law and refinement language are
fixed.

\begin{quote}
\textbf{Reader map (compatibilities; where to look).} -
\textbf{Partition compatibility} (\(\mathcal C_t\)): temporal
refinement/composition (time slicing). See Sections 3--4. -
\textbf{Representation compatibility} (\(\mathcal Q_\hbar\)):
ordering/discretization choices with the same \(\hbar\to0\) limit. See
Sections 6--7. - \textbf{Scale compatibility} (\(\mathcal R_\Lambda\)):
coarse/fine scale comparison after parameter running (RG). See Section 8
(and Appendix 10.5 for an explicit witness).

Symbol definitions and formal summary: Appendix 10.3.
\end{quote}

Transition to Section 3: with notation fixed, the next section derives
the Newtonian area-law refinement argument in modern symbols and links
it to \(L_{\mathrm{ang}}\) conservation.

\section{3. Newtonian Refinement and Area
Law}\label{newtonian-refinement-and-area-law}

\subsection{3.1 Source-Critical Framing}\label{source-critical-framing}

In Book I, Proposition I of the \emph{Principia}, Newton proves that a
centripetal forcing rule implies equal areas swept in equal times by the
radius vector. The historical proof is polygonal and limit-based: one
constructs a piecewise-linear trajectory with impulses directed to a
fixed center, then passes to a continuous curve by refinement
{[}Newton1687{]}.

This section uses that structure directly and only then translates to
modern vector notation. Source-critically, the statements below
distinguish: 1. Newton's geometric argument about polygons and limits.
2. A modern reformulation via torque and angular momentum.

The reformulation is mathematically equivalent under the same
assumptions, but it is an interpretive translation, not a verbatim
historical rendering.

\subsection{3.2 Discrete Refinement
Model}\label{discrete-refinement-model}

Fix equal time steps \(\Delta t>0\), times \(t_k=t_0+k\Delta t\), and a
fixed center \(O\). Let \(\mathbf r_k\) be the position vector at
\(t_k\). The stepwise model is: 1. Free inertial drift between \(t_k\)
and \(t_{k+1}\). 2. Instantaneous impulse at each vertex \(t_k\),
directed along \(\mathbf r_k\) (centripetal/central).

Let \(\mathbf v_k^{-}\) be velocity just before the impulse at \(t_k\),
and \(\mathbf v_k^{+}\) just after. The impulse condition is

\[
m\big(\mathbf v_k^{+}-\mathbf v_k^{-}\big)=J_k\,\hat{\mathbf r}_k,
\qquad \hat{\mathbf r}_k=\frac{\mathbf r_k}{\|\mathbf r_k\|}.
\]

Drift then gives

\[
\mathbf r_{k+1}=\mathbf r_k+\mathbf v_k^{+}\Delta t,
\qquad
\mathbf v_{k+1}^{-}=\mathbf v_k^{+}.
\]

\texttt{Derivation\ D1.1\ (Finite-step\ angular\ momentum\ invariance).}
Define \(\mathbf L_k^{-}=m\,\mathbf r_k\times \mathbf v_k^{-}\),
\(\mathbf L_k^{+}=m\,\mathbf r_k\times \mathbf v_k^{+}\).

At impulse:

\[
\mathbf L_k^{+}-\mathbf L_k^{-}
=m\,\mathbf r_k\times(\mathbf v_k^{+}-\mathbf v_k^{-})
=\mathbf r_k\times (J_k\hat{\mathbf r}_k)=\mathbf 0.
\]

During drift:

\[
\mathbf L_{k+1}^{-}
=m\,\mathbf r_{k+1}\times \mathbf v_{k+1}^{-}
=m(\mathbf r_k+\mathbf v_k^{+}\Delta t)\times \mathbf v_k^{+}
=m\,\mathbf r_k\times \mathbf v_k^{+}
=\mathbf L_k^{+}.
\]

Hence \(\mathbf L_{k+1}^{-}=\mathbf L_k^{-}\): angular momentum is
exactly conserved at every finite step in this refinement model.

\texttt{Derivation\ D1.2\ (Equal\ areas\ in\ equal\ times,\ discrete\ form).}
The swept area in step \(k\) is the triangle area

\[
\Delta A_k
=\frac12\left\|\mathbf r_k\times(\mathbf r_{k+1}-\mathbf r_k)\right\|
=\frac12\left\|\mathbf r_k\times \mathbf v_k^{+}\right\|\Delta t
=\frac{\|\mathbf L\|}{2m}\Delta t.
\]

Therefore for fixed \(\Delta t\), \(\Delta A_k\) is independent of
\(k\). This is the equal-areas statement at finite polygonal level.

\subsection{3.3 Continuum Passage and Central-Force
Generality}\label{continuum-passage-and-central-force-generality}

\texttt{Proposition\ P1.1\ (Refinement\ limit\ of\ areal\ velocity).} If
\(\max_k \Delta t_k\to 0\) under consistent refinement, the finite-step
law above yields

\[
\frac{dA}{dt}=\frac{\|\mathbf L\|}{2m},
\]

for the limiting trajectory whenever the limit exists in the standard
differentiable sense.

For a smooth central force \(\mathbf F(\mathbf r)=f(r)\hat{\mathbf r}\),
this same invariant follows from torque:

\[
\frac{d\mathbf L}{dt}=\mathbf r\times \mathbf F=\mathbf 0.
\]

So the areal law is independent of the inverse-power index \(n\) in
\(\mathbf F=-(K/r^n)\hat{\mathbf r}\): \(n\) changes radial dynamics and
orbit families, but not the areal-velocity conservation mechanism
itself.

\texttt{Heuristic\ H1.1\ (Impulse-to-continuous\ interpretation).} The
impulse model is a refinement scaffold for continuous forcing, not a
literal claim that nature acts by discrete kicks. Its value is
structural: invariants proven exactly at finite step survive controlled
refinement.

\subsection{3.4 Closed Question from the Section 2
Setup}\label{closed-question-from-the-section-2-setup}

Section 2 left one key ambiguity open: is Newton's area law a small-step
approximation or a genuine invariant statement? The derivations above
close that point: within the polygonal central-impulse model, the
equal-area law is exact at each finite step and only the curve
interpolation is a limiting passage.

Transition to Section 4: with the Newtonian invariant fixed in modern
notation, the next section derives Euler-Lagrange equations and Noether
charge conservation to show the same structure directly in action
language.

\section{4. Action as Additive
Invariant}\label{action-as-additive-invariant}

\subsection{4.1 Stationarity Setup}\label{stationarity-setup}

The Section 3 invariant was derived from a refinement model in
configuration geometry. We now restate the same physics through
stationarity of action.

Assume: 1. \(q:[t_i,t_f]\to\mathbb R^d\) is at least \(C^2\), and
variations \(\eta\) are \(C^1\) (or smooth with compact support). 2.
\(\mathcal L(q,\dot q,t)\) is \(C^1\) in \(t\) and \(C^2\) in
\((q,\dot q)\) on the region reached by \((q(t),\dot q(t))\).

Let the action be

\[
S[q]=\int_{t_i}^{t_f}\mathcal L\big(q(t),\dot q(t),t\big)\,dt,
\]

and define \(q_\varepsilon=q+\varepsilon\eta\) for an admissible
variation \(\eta\), with either: 1. fixed endpoints
\(\eta(t_i)=\eta(t_f)=0\), or 2. compact support in \((t_i,t_f)\).

Stationarity means

\[
\delta S[q;\eta]
=\left.\frac{d}{d\varepsilon}\right|_{\varepsilon=0} S[q_\varepsilon]
=0
\quad\text{for all admissible }\eta.
\]

\texttt{Proposition\ P2.0\ (Fundamental\ lemma,\ vector\ form).} If
\(F:[t_i,t_f]\to\mathbb R^d\) is continuous and
\(\int_{t_i}^{t_f} F(t)\cdot\eta(t)\,dt=0\) for all
\(\eta\in C_c^\infty((t_i,t_f);\mathbb R^d)\), then \(F(t)=0\) for all
\(t\in(t_i,t_f)\).

\subsection{4.2 Euler-Lagrange
Derivation}\label{euler-lagrange-derivation}

\texttt{Derivation\ D2.1\ (Euler-Lagrange\ equation).} Differentiate
under the integral sign (justified by the smoothness assumptions). By
the chain rule,
\(\left.\frac{d}{d\varepsilon}\right|_{0}\mathcal L(q+\varepsilon\eta,\dot q+\varepsilon\dot\eta,t)
=\frac{\partial\mathcal L}{\partial q}\cdot\eta+\frac{\partial\mathcal L}{\partial\dot q}\cdot\dot\eta\).
Therefore:

\[
\delta S[q;\eta]
=\int_{t_i}^{t_f}
\left(
\frac{\partial\mathcal L}{\partial q}\cdot\eta
+
\frac{\partial\mathcal L}{\partial \dot q}\cdot\dot\eta
\right)dt.
\]

Integrating the second term by parts:

\[
\delta S[q;\eta]
=
\left[
\frac{\partial\mathcal L}{\partial\dot q}\cdot\eta
\right]_{t_i}^{t_f}
+
\int_{t_i}^{t_f}
\left(
\frac{\partial\mathcal L}{\partial q}
-\frac{d}{dt}\frac{\partial\mathcal L}{\partial\dot q}
\right)\cdot\eta\,dt.
\]

Endpoint or compact-support conditions remove the boundary term. By
\texttt{Proposition\ P2.0}, stationarity for all admissible \(\eta\)
implies:

\[
\frac{d}{dt}\frac{\partial\mathcal L}{\partial\dot q}
-\frac{\partial\mathcal L}{\partial q}
=0.
\]

This is the Euler-Lagrange equation.

\subsection{4.3 Rotational Symmetry and Angular
Momentum}\label{rotational-symmetry-and-angular-momentum}

For planar central motion with

\[
\mathcal L(r,\theta,\dot r,\dot\theta)=
\frac{m}{2}\big(\dot r^2+r^2\dot\theta^2\big)-V(r),
\]

\(\theta\) is cyclic (\(\partial\mathcal L/\partial\theta=0\)). Applying
Euler-Lagrange to \(\theta\) gives:

\[
p_\theta=\frac{\partial\mathcal L}{\partial\dot\theta}=m r^2\dot\theta
=L_{\mathrm{ang}}
\quad\Rightarrow\quad
\frac{dL_{\mathrm{ang}}}{dt}=0,
\]

which is the rotational Noether conservation law {[}Noether1918{]}.

In full vector form for
\(\mathcal L(\mathbf r,\dot{\mathbf r})=\frac{m}{2}\|\dot{\mathbf r}\|^2-V(\|\mathbf r\|)\),
the canonical momentum is
\(\mathbf p=\partial\mathcal L/\partial\dot{\mathbf r}=m\dot{\mathbf r}\)
and rotational invariance yields the conserved angular momentum vector

\[
\mathbf L=\mathbf r\times\mathbf p.
\]

\texttt{Proposition\ P2.1\ (Geometric-variational\ invariant\ equivalence).}
Under the regularity assumptions above, the Section 3 area-law invariant
and the Noether charge are the same quantity in different language:

\[
\dot A=\frac12 r^2\dot\theta=\frac{L_{\mathrm{ang}}}{2m}.
\]

Thus Section 3 and Section 4 do not provide competing derivations; they
provide geometric and variational presentations of one conserved
structure.

\texttt{Proposition\ P2.2\ (Energy\ for\ autonomous\ central\ motion).}
If \(\mathcal L\) has no explicit time dependence, then the energy
function

\[
E=\dot q\cdot\frac{\partial\mathcal L}{\partial\dot q}-\mathcal L
\]

is conserved (time-translation symmetry, another Noether law)
{[}Noether1918{]}. For the central-motion Lagrangian above,

\[
E=\frac{m}{2}\dot r^2+\frac{L_{\mathrm{ang}}^2}{2mr^2}+V(r),
\]

showing the standard reduction to one-dimensional radial motion with
effective potential
\(V_{\mathrm{eff}}(r)=V(r)+L_{\mathrm{ang}}^2/(2mr^2)\).

\subsection{4.4 Additivity and Composition
Pre-Bridge}\label{additivity-and-composition-pre-bridge}

Recall the discrete action functional from the refinement viewpoint:

\[
S_N[q]=\sum_{k=0}^{N-1}
\mathcal L\!\left(q_k,\frac{q_{k+1}-q_k}{\Delta t_k},t_k\right)\Delta t_k.
\]

It is additive under interval concatenation by construction. This
additivity is the structural input used later for composition-based
quantum weighting in Section 6.

\texttt{Heuristic\ H2.1\ (Toward\ distributional\ probes).} Point-like
probes of extrema can be expressed in distributional language. In this
manuscript, technical use of such probes is deferred to Section 5, where
mollifier limits and admissibility are stated explicitly.

Transition to Section 5: with Euler-Lagrange and Noether structure
fixed, we next extend stationarity analysis to weak/distributional
settings and clarify where Dirac-supported constructions are valid.

\section{5. Dirac Distributions and Extremal
Action}\label{dirac-distributions-and-extremal-action}

\subsection{5.1 Why Weak Formulations Appear
Here}\label{why-weak-formulations-appear-here}

The story so far treated trajectories as classically smooth. Two themes
force a more careful formulation: 1. Refinement limits often produce
objects that are only piecewise smooth (corners) or are best handled by
weak limits. 2. The ``point-like probe'' idea (Dirac-supported
localization) is naturally stated in distribution theory.

We keep the role of distributions narrow and explicit: distributions are
used as linear functionals on test functions and as limits of smooth
approximations. Nonlinear operations on distributions are not assumed
unless regularized.

\subsection{5.2 Weak Euler-Lagrange
Equation}\label{weak-euler-lagrange-equation}

Let \(q\in C^1([t_i,t_f];\mathbb R^d)\) be a candidate trajectory and
assume \(\mathcal L(q,\dot q,t)\) is smooth enough that
\(\partial_q\mathcal L\) and \(\partial_{\dot q}\mathcal L\) are
well-defined along \(q\).

\texttt{Proposition\ P3.1\ (Weak\ stationarity\ statement).} If
\(\delta S[q;\eta]=0\) for all
\(\eta\in C_c^\infty((t_i,t_f);\mathbb R^d)\), then the Euler-Lagrange
operator

\[
F[q](t)\equiv \frac{\partial\mathcal L}{\partial q}(q,\dot q,t)
-\frac{d}{dt}\frac{\partial\mathcal L}{\partial\dot q}(q,\dot q,t)
\]

vanishes as a distribution on \((t_i,t_f)\): for all test \(\eta\),

\[
\int_{t_i}^{t_f} F[q](t)\cdot\eta(t)\,dt=0.
\]

Equivalently, \(F[q]=0\) in \(\mathcal D'((t_i,t_f);\mathbb R^d)\),
where \(\mathcal D'\) is the dual of \(C_c^\infty\).

\texttt{Derivation\ D3.1\ (Weak\ form\ from\ first\ variation).} Start
from the first-variation identity (as in Section 4):

\[
\delta S[q;\eta]
=\int_{t_i}^{t_f}\left(
\frac{\partial\mathcal L}{\partial q}\cdot\eta
+
\frac{\partial\mathcal L}{\partial\dot q}\cdot\dot\eta
\right)dt.
\]

Integrate the second term by parts. Compact support eliminates the
boundary term and yields the stated distributional identity.

\subsection{5.3 Point-Like Probes via Mollifiers (Not Raw
Deltas)}\label{point-like-probes-via-mollifiers-not-raw-deltas}

Pick a nonnegative mollifier \(\rho\in C_c^\infty(\mathbb R)\) with
\(\int\rho=1\), and define
\(\rho_\varepsilon(t)=\varepsilon^{-1}\rho(t/\varepsilon)\).

\texttt{Proposition\ P3.2\ (Localized\ probing\ under\ continuity).}
Assume \(F[q](t)\) is continuous at a time \(t_0\in(t_i,t_f)\). Then
weak stationarity implies the pointwise condition \(F[q](t_0)=0\).

\texttt{Derivation\ D3.2.} For any fixed vector \(u\in\mathbb R^d\),
choose a localized test function
\(\eta_\varepsilon(t)=\rho_\varepsilon(t-t_0)\,u\). Then the weak
identity gives

\[
0=\int_{t_i}^{t_f} F[q](t)\cdot \rho_\varepsilon(t-t_0)\,u\,dt
=u\cdot\int_{t_i}^{t_f} \rho_\varepsilon(t-t_0)\,F[q](t)\,dt.
\]

As \(\varepsilon\to 0^+\), the convolution integral tends to
\(F[q](t_0)\) by continuity. Since \(u\) was arbitrary, \(F[q](t_0)=0\).

This is the precise sense in which ``Dirac-supported probes'' recover
pointwise Euler-Lagrange equations: they do so through mollifier limits,
not by inserting nonlinear expressions involving \(\delta(t-t_0)\).

\subsection{5.4 Corners and Impulses: Jump
Conditions}\label{corners-and-impulses-jump-conditions}

There are two distinct phenomena that look ``singular'' in time: 1.
\textbf{Corners}: \(q\) is continuous but \(\dot q\) has a jump at
\(t_0\), with no delta forcing. 2. \textbf{Impulses}: the dynamics
includes a delta force at \(t_0\), producing a momentum jump.

We record both conditions explicitly.

\texttt{Proposition\ P3.3\ (Corner\ condition\ without\ impulse).}
Assume \(q\) is piecewise \(C^2\) with a velocity discontinuity at
\(t_0\), and satisfies the unforced Euler-Lagrange equation on each side
of \(t_0\). Then:

\[
\left[\frac{\partial\mathcal L}{\partial\dot q}\right]_{t_0^-}^{t_0^+}=0.
\]

\texttt{Derivation\ D3.3\ (Corner\ condition).} Integrate the unforced
Euler-Lagrange equation on \([t_0-\varepsilon,t_0+\varepsilon]\):

\[
\left[\frac{\partial\mathcal L}{\partial\dot q}\right]_{t_0-\varepsilon}^{t_0+\varepsilon}
=\int_{t_0-\varepsilon}^{t_0+\varepsilon}\frac{\partial\mathcal L}{\partial q}\,dt.
\]

Let \(\varepsilon\to0^+\). Under local boundedness of
\(\partial_q\mathcal L\), the right-hand side vanishes, yielding the
jump condition above. This is the local corner continuity of canonical
momentum (Weierstrass-Erdmann form in this one-corner setting).

\texttt{Proposition\ P3.4\ (Impulse\ force\ implies\ momentum\ jump).}
Consider the forced Euler-Lagrange equation in distribution form

\[
\frac{d}{dt}\frac{\partial\mathcal L}{\partial\dot q}
-\frac{\partial\mathcal L}{\partial q}
=J\,\delta(t-t_0),
\]

for a fixed impulse vector \(J\in\mathbb R^d\). If
\(\partial_{\dot q}\mathcal L\) has one-sided limits at \(t_0\), then

\[
\left[\frac{\partial\mathcal L}{\partial\dot q}\right]_{t_0^-}^{t_0^+}
\equiv
\frac{\partial\mathcal L}{\partial\dot q}(t_0^+)-\frac{\partial\mathcal L}{\partial\dot q}(t_0^-)
=J.
\]

\texttt{Derivation\ D3.4.} Integrate the equation on
\([t_0-\varepsilon,t_0+\varepsilon]\). The integral of the smooth term
\(\partial_q\mathcal L\) tends to 0 as \(\varepsilon\to 0\). The
derivative term integrates to the jump in
\(\partial_{\dot q}\mathcal L\). The right-hand side integrates to
\(J\).

For the standard mechanical Lagrangian
\(\mathcal L=\frac{m}{2}\|\dot q\|^2-V(q)\), this reduces to the
familiar momentum jump:

\[
m\big(\dot q(t_0^+)-\dot q(t_0^-)\big)=J.
\]

This connects directly to the Section 3 impulse scaffold: central
impulses preserve angular momentum because they change momentum only in
the radial direction.

\subsection{5.5 Extremal Measures: Finite-Dimensional Analogy and
Limits}\label{extremal-measures-finite-dimensional-analogy-and-limits}

The phrase ``Dirac distributions to calculate extrema'' is unambiguous
in finite dimensions. For a smooth \(f:\mathbb R\to\mathbb R\), the
distribution \(\delta(f'(x))\) is supported on the critical points of
\(f\). In higher dimensions one analogously uses \(\delta(\nabla f)\).

\texttt{Derivation\ D3.5\ (Square-root\ delta\ normalization\ and\ Born-rule\ form).}
Let \(f:\mathbb R^N\to\mathbb R\) be smooth and define, for
\(\varepsilon>0\),
\(A_\varepsilon(O):=\varepsilon^{-N/2}\int e^{\frac{i}{\varepsilon}f(x)}O(x)\,dx\).
Then
\(|A_\varepsilon(O)|^2=\varepsilon^{-N}\iint e^{\frac{i}{\varepsilon}(f(x)-f(y))}O(x)\overline{O(y)}\,dx\,dy\).
Under the near-diagonal scaling \(y=x+\varepsilon z\) (so
\(dy=\varepsilon^Ndz\)), one formally obtains
\(|A_\varepsilon(O)|^2\to (2\pi)^N\int \delta(\nabla f(x))\,|O(x)|^2\,dx\).
This exhibits the pattern ``density = \(|\text{amplitude}|^2\)'', with
the exponent \(N/2\) matching the half-density scaling needed to cancel
Jacobians under refinement.

\texttt{Derivation\ D3.5a\ (Nondegenerate\ critical\ points:\ why\ the\ weights\ are\ square\ roots).}
In one dimension, if \(f\) has finitely many nondegenerate critical
points \(x_i\) (so \(f'(x_i)=0\) and \(f''(x_i)\neq 0\)), then the
distributional identity \[
\delta(f'(x))=\sum_i \frac{\delta(x-x_i)}{|f''(x_i)|}
\] makes explicit that \(\delta(f')\,dx\) is a density supported on
stationary points with weights \(1/|f''|\). Stationary phase, by
contrast, gives amplitude contributions weighted by
\(1/\sqrt{|f''(x_i)|}\). This is the clean finite-dimensional reason the
``half-density first'' viewpoint is natural: amplitude weights are
square roots of density weights.

Section 6 recovers the same ``square-root Jacobian'' in the dynamical
setting: semigroup composition of short-time kernels forces the
characteristic \(t^{-d/2}\) normalization (Derivation D4.1a).

In infinite-dimensional settings (paths), one is tempted to write
``formal measures'' supported on stationary-action trajectories. In this
manuscript we treat such expressions as roadmap heuristics until they
are regularized and made compatible with composition (Section 6); see
also \texttt{Heuristic\ H0.2a}.

\subsection{5.6 Caveats (Nonlinear Distribution
Pitfalls)}\label{caveats-nonlinear-distribution-pitfalls}

\begin{enumerate}
\def\labelenumi{\arabic{enumi}.}
\tightlist
\item
  Products like \(\delta(t)^2\) are not defined in standard distribution
  theory; any appearance requires a regularization scheme and a proof of
  scheme-independence for claimed observables.
\item
  ``Evaluate at a point'' is only legitimate for quantities known to be
  continuous (or otherwise well-defined) at that point; mollifier
  probing must state this assumption explicitly.
\item
  Stationarity (\(\delta S=0\)) is not the same as minimality; second
  variation and convexity conditions are separate and are not assumed
  here.
\end{enumerate}

Transition to Section 6: we now have a controlled notion of ``extremal
classical dynamics'' (including impulses and corners) and a precise
language for refinement-local probes. The next section uses composition
under time slicing to motivate amplitude weights and the path integral.

\section{6. Composition and Path
Integral}\label{composition-and-path-integral}

\subsection{6.1 Composition Postulate for
Amplitudes}\label{composition-postulate-for-amplitudes}

Let \(K(q_f,t_f;q_i,t_i)\) denote the transition amplitude. The
structural postulate is composition on intermediate time slices:

\[
K(q_f,t_f;q_i,t_i)
=\int dq\,K(q_f,t_f;q,t)\,K(q,t;q_i,t_i),
\qquad t_i<t<t_f.
\]

\texttt{Heuristic\ H4.0\ (Half-density\ kernels\ make\ composition\ measure-free).}
On a configuration manifold \(M\), the coordinate-free object underlying
the displayed formula is a \textbf{bi-half-density kernel}:
\(K_t(q',q)\in|\Lambda^dT^\ast_{q'}M|^{1/2}\otimes|\Lambda^dT^\ast_qM|^{1/2}\).
Then composition is the canonical pairing in the intermediate variable
\(q\), and it does not require choosing a background measure. Writing
\(\int dq\) is what one gets after choosing a reference density to
identify half-densities with scalar functions.

\texttt{Derivation\ D4.0\ (Coordinate\ invariance\ of\ composition\ via\ half-densities).}
In local coordinates \(q\), write
\(K_t(q',q)=k_t(q',q)\,|dq'|^{1/2}|dq|^{1/2}\). Then
\(K_{t_f-t}(q_f,q)\,K_{t-t_i}(q,q_i)
=k_1k_2\,|dq_f|^{1/2}|dq|\,|dq_i|^{1/2}\) is a density in \(q\), so
\(\int_M K_{t_f-t}(q_f,q)K_{t-t_i}(q,q_i)\) is coordinate invariant.
This is the intrinsic meaning of the composition postulate.

\texttt{Heuristic\ H4.0a\ (Scalarization\ gauge\ and\ scale).} Writing a
half-density kernel as an ordinary scalar function with an explicit
``\(dq\)'' implicitly chooses a reference density \(\rho_\ast\) on \(M\)
(equivalently a reference half-density \(\sigma_\ast=\rho_\ast^{1/2}\)).
Different choices are related by pointwise multiplication and give
unitarily equivalent scalar representations. If one additionally demands
scalar amplitudes be dimensionless, then \(\sigma_\ast\) must carry the
full \(\text{length}^{d/2}\) weight, so a universal choice of
\(\sigma_\ast\) is equivalent to choosing a universal
\(\text{length}^{d/2}\) scale. In a spacetime QFT setting where the
scalarization problem is formulated over an integration variable of
dimension \(D\), this is a \(\text{length}^{D/2}\) scale; in \(D=4\) it
is an area, with the Planck area \(\ell_P^2=\hbar G/c^3\) a natural
universal candidate. A companion note explores further (optional)
hypotheses about such scale suppliers; no such identification is
required for the present paper's structural chain.

\texttt{Heuristic\ H4.0b\ (Independent\ \textbackslash{}(D=4\textbackslash{})\ filter:\ operator\ simplicity\ under\ conformal\ scalarization\ changes).}
The ``scale supplier'' question above is distinct from a different way
\(D=4\) can appear once one insists on scalar representatives:
simplicity of how \emph{kinetic operators} depend on scalarization
choices. In a covariant/QFT setting (spacetime dimension \(D\)), the
scalar Laplacian \(\Delta_g\) induces an operator on half-densities by
conjugation, \[
\widetilde\Delta_g := |g|^{1/4}\Delta_g|g|^{-1/4}.
\] Under a conformal rescaling \(g=e^{2\sigma}\bar g\), the half-density
conjugation produces a quadratic-gradient term
\(\propto |\nabla\sigma|^2\) with universal coefficient \(D(4-D)/4\),
hence it cancels at \(D=4\) (within the conformal class). This is an
operator-simplicity filter (scale-neutral) and is independent of the
coupling-dimension sieve discussed above.

\texttt{Derivation\ D4.1\ (Time\ slicing\ from\ repeated\ composition).}
Iterating the composition law over a partition
\(t_i=t_0<\cdots<t_N=t_f\) gives

\[
K(q_f,t_f;q_i,t_i)=
\int \prod_{k=1}^{N-1} dq_k
\prod_{k=0}^{N-1}
K_\Delta(q_{k+1},q_k;t_k),
\]

with \(q_0=q_i,\;q_N=q_f,\;\Delta t_k=t_{k+1}-t_k\), and \(K_\Delta\)
the short-time kernel.

\texttt{Derivation\ D4.1a\ (Semigroup\ fixes\ the\ \textbackslash{}(t\^{}\{-d/2\}\textbackslash{})\ normalization).}
On \(M=\mathbb R^d\), translation invariance suggests a bi-half-density
kernel of the form \(K_t(x,y)=k_t(x-y)\,|dx|^{1/2}|dy|^{1/2}\), so the
semigroup law becomes a scalar convolution: \(k_{t+s}=k_t*k_s\). Assume
a quadratic short-time phase and write \[
k_t(u)=A(t)\,\exp\!\left(\frac{i m}{2\hbar}\frac{\|u\|^2}{t}\right),
\] interpreting the Gaussian integral in Euclidean time (heat kernel)
and then analytically continuing, or with the usual \(i0\) prescription.
Then \[
(k_t*k_s)(u)=A(t)A(s)\int_{\mathbb R^d}
\exp\!\left(\frac{i m}{2\hbar}\left(\frac{\|u-v\|^2}{t}+\frac{\|v\|^2}{s}\right)\right)\,dv.
\] Completing the square yields \[
\frac{\|u-v\|^2}{t}+\frac{\|v\|^2}{s}
=\frac{\|u\|^2}{t+s}+\frac{t+s}{ts}\left\|v-\frac{s}{t+s}u\right\|^2,
\] so the convolution closes on the same family with \[
A(t+s)=A(t)A(s)\left(\frac{ts}{t+s}\right)^{d/2}\times(\text{phase}).
\] The unique solution (up to an overall constant phase) is
\(A(t)\propto t^{-d/2}\). Thus the exponent \(d/2\) is forced by
semigroup composition: it is the half-density ``square-root Jacobian''
needed for refinement-stable kernel composition. Imposing the delta
initial condition as \(t\to0^+\) fixes the remaining normalization
constant and forces \(\hbar\) into the prefactor (in standard flat-space
scalar conventions, \(A(t)=(m/2\pi i\hbar t)^{d/2}\) up to phase).

\subsection{6.2 From Additive Action to Multiplicative
Weights}\label{from-additive-action-to-multiplicative-weights}

The Section 4/Section 5 structure gives an additive discrete action:

\[
S_N[q]=\sum_{k=0}^{N-1}\mathcal L\!\left(q_k,\frac{q_{k+1}-q_k}{\Delta t_k},t_k\right)\Delta t_k.
\]

Assume short-time locality: each slice contributes a factor depending
only on local step data. Write

\[
K_\Delta(q_{k+1},q_k;t_k)=\mathcal N_k\,W_k.
\]

\texttt{Proposition\ P4.1\ (Exponential\ form\ under\ locality\ +\ composition).}
If 1. total path weight is multiplicative across concatenated slices,
and 2. \(\log W_k\) is additive in \(\Delta t_k\) to first order,

then, up to normalization and higher-order slicing corrections,

\[
\prod_{k=0}^{N-1}W_k
\propto
\exp\!\big(c_0\,S_N[q]\big),
\]

for a constant \(c_0\) with dimensions \([\text{action}]^{-1}\).

Choosing oscillatory quantum time evolution gives \(c_0=i/\hbar\), hence
the standard phase factor \(\exp(iS_N/\hbar)\) {[}Dirac1933{]}
{[}Feynman1948{]}.

\subsection{6.3 Ordering, Discretization, and Quantum
Ambiguity}\label{ordering-discretization-and-quantum-ambiguity}

Different short-time discretizations (left/right/midpoint or more
general \(\alpha\)-schemes) typically correspond to different operator
orderings. In deformation language, this is the same ambiguity as
choosing a star-product representative; these constructions agree in the
classical limit but can differ at subleading quantum order
{[}Landsman1998{]} {[}deGosson2018ShortTimePropagators{]}.

\texttt{Heuristic\ H4.1\ (Same\ classical\ limit,\ different\ quantum\ corrections).}
Two discretizations that differ by \(O(\Delta t)\) in each slice can
produce equivalent classical equations while shifting \(O(\hbar)\) terms
in quantum generators. Thus ordering is a controlled modeling choice,
not a contradiction.

\subsection{6.4 Formal Continuum Limit and Stationary
Phase}\label{formal-continuum-limit-and-stationary-phase}

Formally, as mesh size \(\max_k\Delta t_k\to0\):

\[
K(q_f,t_f;q_i,t_i)\sim
\int_{q_i}^{q_f}\mathcal Dq\;
\exp\!\left(\frac{i}{\hbar}S[q]\right).
\]

This expression is formal at this stage: we do not claim a countably
additive measure construction on full path space.

\texttt{Derivation\ D4.2\ (Classical\ recovery\ mechanism).} Let
\(q=q_{\mathrm{cl}}+\xi\), where \(q_{\mathrm{cl}}\) is stationary:
\(\delta S[q_{\mathrm{cl}};\eta]=0\). Expand:

\[
S[q_{\mathrm{cl}}+\xi]
=S[q_{\mathrm{cl}}]
+\frac12\langle \xi,\mathcal H_{q_{\mathrm{cl}}}\xi\rangle
+O(\xi^3).
\]

Fast phase oscillations cancel nonstationary contributions, while
neighborhoods of stationary paths contribute coherently. This is the
precise sense in which the classical equations reappear as \(\hbar\to0\)
asymptotics.

\texttt{Derivation\ D4.2a\ (Soft\ extremum\ and\ why\ a\ finite\ \textbackslash{}(\textbackslash{}hbar\textbackslash{})\ stabilizes\ refinement).}
Even before any infinite-dimensional measure issues, the
refinement-composition pattern forces a ``softened'' version of
extremization. Consider a finite-dimensional two-slice action
\(S(q_2,q_1)+S(q_1,q_0)\) and form the Euclideanized composed weight \[
W_\hbar(q_2,q_0):=\int dq_1\;\exp\!\left(-\frac{1}{\hbar}\big(S(q_2,q_1)+S(q_1,q_0)\big)\right).
\] Define the associated coarse effective action (a log-partition
functional) \[
S_{\mathrm{eff}}^{(\hbar)}(q_2,q_0):=-\hbar\ln W_\hbar(q_2,q_0).
\] Then refinement-composition is exact at the level of weights (add
actions in the exponent, integrate the intermediate variable), and the
hard elimination/extremization rule appears only as the sharpening
limit: under standard nondegeneracy assumptions, Laplace's method gives
\[
S_{\mathrm{eff}}^{(\hbar)}(q_2,q_0)
\xrightarrow{\hbar\to0}
\inf_{q_1}\big(S(q_2,q_1)+S(q_1,q_0)\big),
\] with \(O(\hbar)\) corrections determined by local quadratic data near
the minimizer(s). So \(\hbar\) plays the role of a universal control
parameter that makes ``refine then compare'' stable, with classical
extremals recovered as a limit. In real time, the same pattern appears
with \(e^{iS/\hbar}\) and stationary phase in place of positivity and
Laplace concentration.

\texttt{Remark\ D4.2b\ (Delocalized\ angles\ in\ angular-momentum\ eigenstates).}
The stationary-phase mechanism explains how classical trajectories
reappear in semiclassical packets. It does not imply that a single
stationary eigenstate is a localized classical orbit. A simple witness
occurs already in central potentials: in polar coordinates the azimuthal
angle \(\phi\) is conjugate to \(L_z\), and separation of variables
yields {[}TongQMLectures{]} \[
\psi(r,\phi)=R(r)\,e^{im\phi}.
\] Therefore \[
|\psi(r,\phi)|^2=|R(r)|^2
\] is independent of \(\phi\), i.e.~the azimuthal angle is uniformly
distributed in an \(L_z\) eigenstate. Localized ``orbit phase'' pictures
correspond to coherent superpositions/wavepackets, consistent with the
manuscript's use of stationary-phase concentration rather than
``eigenstate \(=\) orbit'' identification.

\texttt{Derivation\ D4.3\ (Van\ Vleck\ prefactor\ is\ a\ bi-half-density).}
In the semiclassical approximation, the endpoint kernel has the standard
form \[
K(q_f,t_f;q_i,t_i)\approx
\frac{1}{(2\pi i\hbar)^{d/2}}
\left|\det\!\left(-\frac{\partial^2 S_{\mathrm{cl}}}{\partial q_f\,\partial q_i}\right)\right|^{1/2}
\exp\!\left(\frac{i}{\hbar}S_{\mathrm{cl}}(q_f,t_f;q_i,t_i)\right),
\] where \(S_{\mathrm{cl}}\) is the classical action evaluated on the
stationary path with those endpoints. Under changes of coordinates
\(q_f=q_f(q_f')\), \(q_i=q_i(q_i')\), the mixed Hessian transforms by
the chain rule, and \[
\det\!\left(-\frac{\partial^2 S_{\mathrm{cl}}}{\partial q_f'\,\partial q_i'}\right)
=
\det\!\left(\frac{\partial q_f}{\partial q_f'}\right)
\det\!\left(\frac{\partial q_i}{\partial q_i'}\right)
\det\!\left(-\frac{\partial^2 S_{\mathrm{cl}}}{\partial q_f\,\partial q_i}\right).
\] Taking square roots yields a factor
\(|\det(\partial q_f/\partial q_f')|^{1/2}|\det(\partial q_i/\partial q_i')|^{1/2}\),
so the prefactor transforms as a half-density in each endpoint variable.
This is the concrete semiclassical origin of the half-density viewpoint
in \texttt{Heuristic\ H4.0}. For an early
semiclassical/correspondence-principle anchor in the ``Van Vleck''
tradition, see {[}VanVleck1928Correspondence{]}. For a modern OA
statement of the Van Vleck propagator/prefactor and the associated ``Van
Vleck density'', see {[}deGosson2018ShortTimePropagators{]}. An explicit
finite-dimensional quadratic-elimination template (Schur complement)
shows how mixed endpoint Hessians arise in time slicing; we omit it
here.

\subsection{6.5 Link Back to Section 5 Singular
Dynamics}\label{link-back-to-section-5-singular-dynamics}

The composition picture naturally includes piecewise-smooth
trajectories. At impulses, the dominant classical skeleton must satisfy
the jump laws from Section 5:

\[
\left[\frac{\partial\mathcal L}{\partial\dot q}\right]_{t_0^-}^{t_0^+}=J
\quad\text{(impulse)}
\qquad
\left[\frac{\partial\mathcal L}{\partial\dot q}\right]_{t_0^-}^{t_0^+}=0
\quad\text{(corner, unforced)}.
\]

So the ``extremal set'' entering semiclassical evaluation is broader
than globally smooth trajectories; it includes admissible broken
trajectories obeying the correct matching conditions.

Transition to Section 7: with composition, weighting, and
classical-recovery logic in place, we can now present quantization as
deformation of algebraic products, linking path-integral discretization
choices to tangent/cotangent groupoid deformation structure.

\section{7. Deformation Quantization
Bridge}\label{deformation-quantization-bridge}

\subsection{7.1 From Path Weights to Product
Deformation}\label{from-path-weights-to-product-deformation}

Section 6 established that discretized composition introduces nonunique
short-time prescriptions (left/right/midpoint and related schemes). The
algebraic restatement is: quantization should deform the classical
product of observables rather than replace classical mechanics by
unrelated objects {[}Landsman1998{]} {[}Connes1994{]}.

Let \(M\) be phase space with Poisson bracket \(\{\cdot,\cdot\}\), and
let \(\mathcal A_0\) be a commutative algebra of classical observables
(e.g., smooth functions with suitable decay/domain conditions). A
deformation quantization is a family of associative products
\(\star_\hbar\) on \(\mathcal A_0\) such that:

\[
f\star_\hbar g
=
fg+\sum_{n\ge 1}\hbar^n B_n(f,g),
\]

where \(B_n\) are bilinear operators, with \(f\star_0 g=fg\).

\texttt{Proposition\ P5.1\ (Classical\ compatibility\ conditions).} If
\(\star_\hbar\) is associative for each \(\hbar\) and depends
smoothly/formally on \(\hbar\), then the antisymmetric part of \(B_1\)
controls the leading noncommutativity:

\[
[f,g]_{\star_\hbar}
\equiv
f\star_\hbar g-g\star_\hbar f
=
\hbar\big(B_1(f,g)-B_1(g,f)\big)+O(\hbar^2).
\]

So first-order noncommutativity is fully determined by
\(B_1^{\mathrm{anti}}\).

\subsection{7.2 Commutator-to-Poisson
Recovery}\label{commutator-to-poisson-recovery}

\texttt{Derivation\ D5.1\ (Correspondence\ limit).} Impose the
correspondence requirement that first-order antisymmetry matches the
Poisson bracket:

\[
B_1(f,g)-B_1(g,f)=i\,\{f,g\}.
\]

Then

\[
[f,g]_{\star_\hbar}
=
i\hbar\,\{f,g\}+O(\hbar^2),
\]

and therefore

\[
\lim_{\hbar\to0}
\frac{1}{i\hbar}[f,g]_{\star_\hbar}
=
\{f,g\}.
\]

Dimensional closure: \([\hbar]=[\text{action}]\), while \(\{f,g\}\)
carries one inverse action factor relative to \(fg\) in canonical
coordinates, so \(i\hbar\{f,g\}\) has the same physical dimension as
\(fg\). This is the same unit-consistency condition already used in
Section 6 for \(\exp(iS/\hbar)\).

\subsection{7.3 Concrete Model and Ordering
Content}\label{concrete-model-and-ordering-content}

For flat phase space, the Moyal product provides an explicit
representative:

\[
(f\star_M g)(q,p)
=
f(q,p)\exp\!\left[
\frac{i\hbar}{2}
\left(
\overleftarrow{\partial_q}\overrightarrow{\partial_p}
-\overleftarrow{\partial_p}\overrightarrow{\partial_q}
\right)\right]g(q,p),
\]

which reproduces the Poisson bracket at leading order and higher quantum
corrections at higher orders {[}Landsman1998{]}.

\texttt{Heuristic\ H5.1\ (Ordering\ as\ deformation\ gauge\ choice).}
The Section 6 discretization ambiguity is naturally interpreted as
choosing different but deformation-equivalent star products; they share
the same classical bracket data but differ in \(O(\hbar)\) and higher
corrections {[}Landsman1998{]}.

\subsection{7.4 Equivalence Classes and Groupoid
Viewpoint}\label{equivalence-classes-and-groupoid-viewpoint}

\texttt{Proposition\ P5.2\ (Equivalent\ star\ products,\ same\ classical\ limit).}
If two products \(\star_\hbar\) and \(\star'_\hbar\) are related by a
formal automorphism

\[
T_\hbar=\mathrm{id}+\hbar T_1+O(\hbar^2),
\qquad
f\star'_\hbar g
=
T_\hbar^{-1}\!\big((T_\hbar f)\star_\hbar(T_\hbar g)\big),
\]

then they define the same Poisson bracket in the \(\hbar\to0\) limit,
while generally differing in subleading quantum terms.

This is the algebraic side of the same continuity narrative:
quantization data are organized into equivalence classes compatible with
one classical limit. Geometric deformation programs (including
tangent-groupoid viewpoints) encode the same bridge from commutative
classical data to noncommutative quantum products {[}Connes1994{]}.

\subsection{7.5 Formal Deformation
Boundary}\label{formal-deformation-boundary}

In this section we use formal/asymptotic deformation language for local
bridge statements. We do not require the full \(C^\ast\)-algebraic
deformation-quantization program for the manuscript's main argument; the
needed ingredient is compatibility of the classical limit and quantum
corrections under the stated assumptions {[}Landsman1998{]}.

Transition to Section 8: with the deformation bridge in place, the
remaining problem is not how to define first-order quantum corrections,
but how to keep refined predictions finite and scale-consistent when
naive limits diverge. That control problem is precisely the
renormalization step.

\section{8. Renormalization as Controlled
Refinement}\label{renormalization-as-controlled-refinement}

\subsection{8.1 Why Renormalization Appears in Refinement
Limits}\label{why-renormalization-appears-in-refinement-limits}

The previous sections treated refinement as benign: polygonal refinement
in Section 3, time-slicing in Section 6, and deformation parameter
limits in Section 7. In quantum field theory and in several singular
quantum-mechanical models (e.g.~contact interactions), the same
refinement step can instead \emph{diverge}
{[}ManuelTarrach1994PertRenQM{]} {[}BoyaRivero1994Contact{]}: as the
cutoff scale is removed, intermediate quantities blow up even when
low-energy physics is expected to remain finite.

Renormalization is the mechanism that restores the program's central
thesis in the divergent case: it provides a controlled rule for taking
refinement limits so that observables remain stable. Operationally, it
accepts that intermediate expressions depend on a regulator (cutoff),
but requires that properly defined observables do not.

To keep the discussion aligned with the paper's composition language, we
treat renormalization as an invariance/consistency condition across
composed refinement steps.

\texttt{Heuristic\ H6.1\ (Renormalization\ as\ part\ of\ “what\ a\ theory\ is”).}
In benign refinement problems, one can often send the refinement
parameter to zero without further choices. In divergent refinement
problems, the renormalization prescription (what is regulated, what is
held fixed, and how parameters are re-expressed as the cutoff moves) is
not optional bookkeeping: it is part of the definition of the continuum
theory, because it specifies which composed/refined predictions are
declared physically stable.

\subsection{8.2 Regulator, Bare Data, and Renormalized
Observables}\label{regulator-bare-data-and-renormalized-observables}

Let \(\Lambda\) denote a refinement cutoff (e.g., momentum cutoff
\(|k|<\Lambda\) or lattice spacing \(a\) with \(\Lambda\sim 1/a\)). Let
\(g_B(\Lambda)\) denote the cutoff-dependent \emph{bare} parameters of
the regulated theory (couplings, masses, field normalizations), and let
\(O_\Lambda\) be a regulated prediction for some observable \(O\).

\texttt{Proposition\ P6.1\ (Renormalized\ observable\ as\ cutoff-stable\ limit).}
If there exists a choice of cutoff-dependent bare parameters
\(g_B(\Lambda)\) such that the limit

\[
O_{\mathrm{phys}}
\equiv
\lim_{\Lambda\to\infty} O_\Lambda\big(g_B(\Lambda)\big)
\]

exists and is finite (or has a controlled asymptotic expansion) for the
observables of interest, then the refinement limit is \emph{defined} by
this prescription.

This statement is intentionally operational: it does not assume that the
cutoff-free object exists without tuning. It states that ``physical
theory'' means a stable target under refinement.

It is often convenient to introduce a renormalization scale \(\mu\) (a
reference resolution) and a renormalization map \(R_{\Lambda\to\mu}\)
from bare to renormalized parameters:

\[
g_R(\mu) = R_{\Lambda\to\mu}\big(g_B(\Lambda)\big).
\]

The composition viewpoint suggests a compatibility condition:
renormalizing from \(\Lambda\) down to \(\mu\) should be the same as
renormalizing from \(\Lambda\) to an intermediate scale \(\kappa\) and
then from \(\kappa\) to \(\mu\):

\[
R_{\Lambda\to\mu} = R_{\kappa\to\mu}\circ R_{\Lambda\to\kappa}.
\]

This is the renormalization-group (RG) semigroup property in refinement
language; for a standard Wilsonian/ERG discussion of coarse-graining
flows and fixed points, see {[}Rosten2012ERG{]}.

\texttt{Derivation\ D6.0\ (Control\ map\ \textbackslash{}(\textbackslash{}tau\textbackslash{}):\ comparing\ refinements\ at\ fixed\ ruler).}
The same compatibility condition can be stated without committing to a
particular regulator. Fix a reference ruler \(h>0\) (the resolution at
which we compare predictions), and let \(A_{h,\theta}\) denote a family
of amplitudes/prediction functionals indexed by parameters \(\theta\)
(couplings, normalizations, and any fixed conventions such as
scalarization gauge). For a refinement factor \(b>1\), choose a
``compare at ruler \(h\)'' operation \(\mathcal C_{b,h}\): take a
prediction written at finer ruler \(h/b\) and express it back at ruler
\(h\) (e.g.~by composing fine steps or integrating intermediate
variables when such a representation exists). Scale compatibility is the
closure requirement that refinement-comparison lands back in the same
family after a parameter update:

\[
\mathcal C_{b,h}\!\big(A_{h/b,\theta}\big)=A_{h,\tau_b(\theta)}.
\]

Here \(\tau_b\) is the control/flow map induced by the
compare-at-fixed-ruler operation. When no such \(\tau_b\) exists within
the chosen parameter family, refinement generates new operators and the
family must be enlarged (counterterms/effective operators). A concrete
micro-witness is \texttt{Derivation\ D6.2a}, where step-halving induces
\(\tau_2(a)=a/2+1/4\) with fixed point \(a_\ast=1/2\). In the cutoff
notation above, one may view \(R_{\Lambda\to\mu}\) as a special case of
\(\tau\) written with explicit reference scales.

\subsection{8.3 RG Equation from Cutoff
Independence}\label{rg-equation-from-cutoff-independence}

\texttt{Derivation\ D6.1\ (RG\ equation\ as\ consistency\ under\ refinement).}
Assume a regulated observable depends on the cutoff \(\Lambda\) both
explicitly and through the bare parameters \(g_B(\Lambda)\):

\[
O_{\mathrm{phys}} = O_\Lambda\big(g_B(\Lambda)\big),
\]

and impose cutoff-independence of the physical prediction:

\[
\frac{d}{d\ln\Lambda} O_\Lambda\big(g_B(\Lambda)\big)=0.
\]

By the chain rule,

\[
0
=
\frac{\partial O_\Lambda}{\partial \ln\Lambda}
+
\sum_a \frac{d g_B^a}{d\ln\Lambda}\frac{\partial O_\Lambda}{\partial g_B^a}
,
\]

where \(a\) ranges over the components of the parameter vector. Defining
the beta functions \(\beta_B^a(g_B)\equiv \frac{d g_B^a}{d\ln\Lambda}\),
we obtain the RG equation:

\[
\left(\frac{\partial}{\partial \ln\Lambda}
+\sum_a \beta_B^a(g_B)\frac{\partial}{\partial g_B^a}\right)O_\Lambda(g_B)=0.
\]

In the \(\mu\)-parametrized form with renormalized parameters
\(g_R(\mu)\), the same reasoning yields a flow equation
\(\mu \frac{d}{d\mu}g_R(\mu)=\beta(g_R)\) plus corresponding RG
invariance equations for renormalized observables. The key point for
this manuscript is structural: RG is not extra physics added after
quantization; it is the \emph{consistency condition} that makes composed
refinement meaningful when naive limits diverge.

\texttt{Proposition\ P6.2\ (Flow\ generator\ from\ refinement\ semigroup).}
Let \(W_{\Lambda\to\kappa}\) be the map sending effective parameters at
scale \(\Lambda\) to effective parameters at scale \(\kappa<\Lambda\),
and assume: 1. \(W_{\Lambda\to\Lambda}=\mathrm{id}\), 2.
\(W_{\kappa\to\mu}\circ W_{\Lambda\to\kappa}=W_{\Lambda\to\mu}\), 3.
differentiability with respect to \(\ln\Lambda\).

Then the infinitesimal generator defines beta functions:

\[
\beta^a(g)
=
\left.\frac{d}{dt}\,W^a_{e^t\mu\to\mu}(g)\right|_{t=0},
\]

and finite scale changes are recovered by integrating this vector field
on parameter space. So RG flow is the differential form of composed
refinement.

\texttt{Derivation\ D6.2\ (Toy\ logarithmic\ divergence\ and\ subtraction).}
Consider a single-coupling situation with a logarithmic cutoff
dependence in a regulated prediction:

\[
O_\Lambda(g_B;\mu)=g_B+c\,g_B^2\ln\!\left(\frac{\Lambda}{\mu}\right)+O(g_B^3),
\]

where \(c\) is a dimensionless constant determined by the model and by
the chosen renormalization convention. Define the renormalized coupling
at scale \(\mu\) by a renormalization condition
\(g_R(\mu)\equiv O_\Lambda(g_B(\Lambda);\mu)\) which is held fixed as
\(\Lambda\to\infty\). Cutoff-independence then enforces:

\[
0=\frac{d}{d\ln\Lambda}g_R(\mu)
=\frac{d g_B}{d\ln\Lambda}+c\,g_B^2+O(g_B^3),
\]

so \(\beta_B(g_B)\equiv \frac{d g_B}{d\ln\Lambda}=-c\,g_B^2+O(g_B^3)\).
Equivalently, at fixed bare coupling one finds the running with \(\mu\):

\[
\beta(g_R)\equiv \mu\frac{d g_R}{d\mu}=-c\,g_R^2+O(g_R^3),
\]

illustrating how renormalization turns the divergent \(\ln\Lambda\)
refinement effect into a scale-dependent coupling consistent with stable
observables.

\subsection{8.4 Refinement Control Before QFT: Scale-Halving as a
Model}\label{refinement-control-before-qft-scale-halving-as-a-model}

One can see the same logic in purely classical numerical refinement.
Consider an evolution operator \(\Phi_\varepsilon\) representing ``one
step'' at resolution \(\varepsilon\). Composition gives
\(\Phi_{2\varepsilon}\approx \Phi_\varepsilon\circ\Phi_\varepsilon\). A
refinement-control question is then: what correction to
\(\Phi_\varepsilon\) makes the two-step composition agree with a
one-step method after rescaling back to the same reference resolution?

\texttt{Derivation\ D6.2a\ (Step-halving\ induces\ a\ control\ map\ \textbackslash{}(\textbackslash{}tau\textbackslash{})\ in\ a\ toy\ ODE).}
Consider the scalar ODE \(y'=f(y)\) and a one-parameter family of
one-step maps at step size \(h\), \[
\Phi_h^{(a)}(y)=y+h f(y)+a\,h^2 f'(y)f(y)+O(h^3).
\] Define the step-halving comparison
\(H(\Phi_h):=\Phi_{h/2}\circ \Phi_{h/2}\). A direct expansion to order
\(h^2\) gives \[
H(\Phi_h^{(a)})(y)=y+h f(y)+\left(\frac14+\frac{a}{2}\right)h^2 f'(y)f(y)+O(h^3).
\] Thus, within this ansatz family, refinement comparison closes by a
parameter update \[
H(\Phi_h^{(a)})=\Phi_h^{(\tau_2(a))}+O(h^3),\qquad \tau_2(a)=\frac{a}{2}+\frac14,
\] with fixed point \(a_\ast=1/2\) (the second-order Taylor coefficient
of the exact flow). This is a clean micro-model for
\texttt{Derivation\ D6.0}: \(\tau_b\) is the control map required so
that ``refine and compare'' lands back in the chosen family; failure of
closure forces enlarging the family (counterterms).

\texttt{Heuristic\ H6.2\ (Rooted\ trees\ as\ refinement\ bookkeeping).}
In Runge-Kutta and related integrators, the comparison between composed
steps and a single step organizes into rooted-tree expansions; the
corresponding composition law forms a group (the Butcher group).
Interpreting ``step-halving then rescaling back'' as a scale-update
operation makes the analogy with RG bookkeeping explicit, and
rooted-tree/Hopf-algebra combinatorics also appears in perturbative
renormalization {[}Brouder1999{]} {[}McLachlan2017{]}
{[}ConnesKreimer2000{]}.

This example is included not to replace QFT renormalization, but to
reinforce the paper's thesis with a clean model: renormalization is what
you do when ``refine and compare'' is not automatically stable. The
Butcher \emph{group} concerns formal method composition, whereas
Wilsonian coarse-graining is generally a \emph{semigroup} because
information is discarded at each coarse-graining step.

\subsection{8.5 Counterterms as Refinement
Corrections}\label{counterterms-as-refinement-corrections}

In field theory language, refinement is implemented by a regulated
action \(S_\Lambda\) with cutoff-dependent parameters. Schematically,

\[
S_\Lambda[\phi]
=
\int d^D x\left(
\frac{Z(\Lambda)}{2}(\partial\phi)^2
+\frac{m^2(\Lambda)}{2}\phi^2
+\frac{\lambda(\Lambda)}{4!}\phi^4
+\cdots
\right),
\]

where \(D\) is the spacetime dimension and the ``\(\cdots\)'' stands for
additional operators allowed by symmetries and by the desired accuracy.
The counterterm viewpoint is simply the statement that
\(Z,m^2,\lambda,\ldots\) must be chosen as functions of \(\Lambda\) so
that the cutoff-stable limits of observables exist. In this
compositional narrative, counterterms are refinement corrections
required to keep the ``same theory'' after integrating out short scales.

\texttt{Derivation\ D6.3\ (Difference\ quotient\ as\ counterterm\ subtraction).}
Let \(f\in C^1\) and \(\varepsilon\to 0^+\). The two regulated
quantities \(f(x+\varepsilon)/\varepsilon\) and \(f(x)/\varepsilon\)
each diverge like \(1/\varepsilon\). Subtracting the local counterterm
\(f(x)/\varepsilon\) produces a finite limit:

\[
\frac{f(x+\varepsilon)}{\varepsilon}-\frac{f(x)}{\varepsilon}
=\frac{f(x+\varepsilon)-f(x)}{\varepsilon}
\xrightarrow{\varepsilon\to0} f'(x).
\]

This is a minimal model of renormalization: divergent regulated
expressions become finite after subtracting a divergence that depends
only on local data, and the renormalized quantity is the cutoff-stable
remainder.

The Connes-Kreimer framework makes this consistency structural by
encoding perturbative renormalization as a factorization problem with a
Hopf-algebra organization of divergences {[}ConnesKreimer2000{]}. For
this manuscript, the take-away is not the full machinery but the
alignment: renormalization is a principled method for producing
regulator-independent predictions from composable local pieces when
refinement alone does not converge.

\subsection{8.6 Assumptions and Boundaries
Audit}\label{assumptions-and-boundaries-audit}

\texttt{Proposition\ P6.3\ (Closure\ assumption\ for\ finite-parameter\ flow).}
Finite-dimensional beta-function systems are closed only after choosing
a truncation/ansatz for allowed operators (or a complete effective
basis). If new operators are generated by refinement and omitted from
the parameter vector, the reduced flow is only approximate.

This caveat is essential for interpreting section 8 correctly: the RG
equations written here are exact at the level of the chosen variable
set, but practical truncations can make them approximate. The main
thesis is unaffected: renormalization remains the rule that restores
cross-scale consistency under composed refinement.

Transition to Section 9: we now have the full chain Newtonian refinement
\(\to\) action additivity \(\to\) path-integral composition \(\to\)
deformation compatibility \(\to\) renormalized refinement control. The
final synthesis section stress-tests the transitions, labels what
remains heuristic, and consolidates the manuscript into a single
coherent argument.

\section{9. Unified Perspective and Open
Problems}\label{unified-perspective-and-open-problems}

\subsection{9.1 End-to-End Claim Graph}\label{end-to-end-claim-graph}

The manuscript has built one chain across seven technical steps: 1.
Section 3: refinement geometry in central-force motion yields an exact
finite-step invariant (equal areas / angular momentum preservation). 2.
Section 4: action stationarity and Noether symmetry express the same
invariant in variational language. 3. Section 5: weak/distributional
formulation extends stationarity to singular limits (mollifiers,
corners, impulses) with explicit admissibility boundaries. 4. Section 6:
composition under slicing plus additive action yields exponential
weighting and stationary-phase classical recovery. 5. Section 7:
ordering ambiguity is recast as deformation-equivalence data with a
shared Poisson classical limit. 6. Section 8: divergent refinement is
controlled by renormalization maps and RG semigroup consistency.

The unifying thesis is therefore not ``classical then quantum then QFT''
as disconnected layers, but ``one refinement/composition problem under
progressively stricter consistency requirements.''

\texttt{Proposition\ P7.1\ (Compatibility\ chain\ of\ limits).} Under
the regularity and admissibility assumptions stated in sections 3--8,
the following compatibility conditions can be imposed simultaneously:

\[
\delta S[q;\eta]=0
\;\Longleftrightarrow\;
\text{Euler-Lagrange in weak form},
\]

\[
K \sim \int \mathcal Dq\,e^{iS[q]/\hbar}
\;\Longrightarrow\;
\hbar\to 0 \text{ concentrates on } \delta S=0,
\]

\[
\lim_{\hbar\to 0}\frac{1}{i\hbar}[f,g]_{\star_\hbar}=\{f,g\},
\qquad
\left(\partial_{\ln\Lambda}+\beta\cdot\partial_g\right)O_\Lambda=0.
\]

These equations are not identical statements; they are compatibility
constraints on one staged construction: classical extremals, quantum
composition, algebraic deformation, and scale consistency must match in
their overlap domains.

\subsection{9.2 Transition Stress Test}\label{transition-stress-test}

\texttt{Derivation\ D7.1\ (No\ hidden\ leap\ audit\ across\ transitions).}
The manuscript can be stress-tested by checking each transition against
one explicit closure condition:

\begin{enumerate}
\def\labelenumi{\arabic{enumi}.}
\item
  \texttt{Section\ 3\ -\textgreater{}\ Section\ 4} closure: finite-step
  angular momentum invariance and variational Noether charge agree
  through \(\dot A=\frac{L_{\mathrm{ang}}}{2m}\). This closes the
  geometry-to-variational bridge.
\item
  \texttt{Section\ 4\ -\textgreater{}\ Section\ 5} closure:
  Euler-Lagrange equations in classical smooth form imply weak
  distributional form when tested against \(C_c^\infty\), and mollifier
  localization recovers pointwise equations under continuity
  assumptions. This closes smooth-to-weak extension.
\item
  \texttt{Section\ 5\ -\textgreater{}\ Section\ 6} closure: the
  admissible classical set in semiclassics includes smooth and
  piecewise-smooth trajectories satisfying matching laws
  \([\partial_{\dot q}\mathcal L]^+_- =0\) (corner) or \(=J\) (impulse).
  This closes singular dynamics into composition.
\item
  \texttt{Section\ 6\ -\textgreater{}\ Section\ 7} closure:
  discretization/ordering freedom in short-time kernels maps to
  star-product representatives that share the same Poisson boundary at
  \(\hbar\to0\). This closes path-integral ambiguity into deformation
  language.
\item
  \texttt{Section\ 7\ -\textgreater{}\ Section\ 8} closure: deformation
  handles classical/quantum compatibility at fixed scale;
  renormalization handles cross-scale consistency when refinement
  diverges. This closes fixed-scale quantization into multiscale
  consistency.
\end{enumerate}

The audit criterion is simple: every bridge states its assumptions and
carries an explicit invariant or equation through the transition. Where
this fails, the claim is downgraded to heuristic.

\subsection{9.3 What Is Proven vs
Heuristic}\label{what-is-proven-vs-heuristic}

For navigation, Sections 3--8 contain the following mix of results and
bridges:

\begin{enumerate}
\def\labelenumi{\arabic{enumi}.}
\item
  \textbf{Section 3:} polygonal central-impulse refinement preserves
  angular momentum and equal areas at finite step.
  (\texttt{Proposition}, \texttt{Derivation}) Boundary: central impulses
  and consistent refinement limit.
\item
  \textbf{Section 4:} Euler--Lagrange plus Noether recover
  angular-momentum and energy invariants. (\texttt{Derivation},
  \texttt{Proposition}) Boundary: \(C^2\) trajectory regularity and
  standard variational hypotheses.
\item
  \textbf{Section 5:} weak Euler--Lagrange, mollifier probes, and jump
  laws for corners/impulses. (\texttt{Proposition}, \texttt{Derivation})
  Boundary: distribution linearity and no undefined nonlinear products.
\item
  \textbf{Section 6:} composition plus additivity imply exponential
  weighting; stationary phase yields classical recovery. (mixed:
  \texttt{Proposition}, \texttt{Derivation}, \texttt{Heuristic})
  Boundary: formal path-integral usage and local stationary-phase
  assumptions.
\item
  \textbf{Section 7:} deformation products recover the Poisson bracket;
  ordering appears as deformation-equivalence choice. (mixed:
  \texttt{Proposition}, \texttt{Derivation}, \texttt{Heuristic})
  Boundary: formal/asymptotic deformation setting and scope of
  equivalence.
\item
  \textbf{Section 8:} RG appears as semigroup consistency under composed
  refinement; counterterms appear as refinement corrections. (mixed:
  \texttt{Proposition}, \texttt{Derivation}, \texttt{Heuristic})
  Boundary: closure/truncation assumptions and regulator-scheme choice.
\end{enumerate}

\texttt{Heuristic\ H7.1\ (Programmatic\ interpretation).} The
Newton-to-path-integral narrative is best read as a \emph{compatibility
program} rather than a single theorem: each layer adds new consistency
constraints while preserving prior invariants in its domain of validity.

\subsection{9.4 Residual
Vulnerabilities}\label{residual-vulnerabilities}

\begin{enumerate}
\def\labelenumi{\arabic{enumi}.}
\tightlist
\item
  The path integral remains formal at full measure-theoretic level;
  constructive control is not provided here.
\item
  Deformation equivalence is stated at the structural level; explicit
  model-by-model operator-domain analysis is deferred.
\item
  RG flow is derived structurally; no full one-loop or higher-loop
  field-theory computation is included in the main text.
\item
  Truncation closure in section 8 is identified but not benchmarked by
  an explicit truncation-error study.
\end{enumerate}

These are not hidden defects; they are explicit scope boundaries. The
manuscript now separates proven derivations from heuristic bridges,
which was a core objective of the staged design.

\subsection{9.5 Future Work}\label{future-work}

The present manuscript emphasizes explicit derivations at the level of
mechanics and simple quantum models. Natural extensions include: 1. one
field-theory loop computation in a fixed scheme, to complement the
structural RG discussion, 2. upgrade the compact reader map into a
consolidated diagram/figure for reader navigation, 3. a final
notation/normalization pass tailored to the submission venue.

\subsection{9.6 Conclusion}\label{conclusion}

The paper's central thesis is that ``continuum laws'' are most cleanly
understood as \emph{stable targets of controlled refinement}:
finite-step invariants (Newton's polygonal dynamics) persist through the
action formulation, and action additivity is the algebraic structure
that forces a consistent composition law. In the quantum setting,
multiplicative composition together with local additivity makes
exponential weighting structurally natural, and classical dynamics are
recovered by stationary-phase concentration. Two further control
mechanisms enter when naive refinement is not uniquely defined:
deformation quantization organizes ordering/discretization choices into
equivalence classes compatible with a common classical limit, and
renormalization supplies the compatibility condition required when
refinement limits diverge.

Throughout, the manuscript keeps refinement explicit, separates
derivations from heuristics, and highlights the additional hypotheses
needed for each continuum passage.

\section{10. Technical Appendices}\label{technical-appendices}

This section provides the appendices announced at the end of Section 9.
Each subsection is a compact worked extension tied to one residual
vulnerability.

\subsection{10.1 Worked Renormalization Template (Single
Coupling)}\label{worked-renormalization-template-single-coupling}

The objective is to replace purely structural RG language with one
explicit subtraction-and-running calculation.

Assume a regulated quantity has the perturbative form

\[
F_\Lambda(g_B;\mu)
=
g_B
+c\,g_B^2\ln\!\left(\frac{\Lambda}{\mu}\right)
+d\,g_B^2
+O(g_B^3),
\]

with dimensionless constants \(c,d\). Define a renormalized coupling by
a subtraction condition at scale \(\mu\):

\[
g_R(\mu)\equiv g_B+c\,g_B^2\ln\!\left(\frac{\Lambda}{\mu}\right).
\]

\texttt{Derivation\ D8.1\ (Finite\ renormalized\ prediction\ at\ fixed\ subtraction\ scale).}
Invert the definition to second order:

\[
g_B
=
g_R-c\,g_R^2\ln\!\left(\frac{\Lambda}{\mu}\right)+O(g_R^3).
\]

Substitute into \(F_\Lambda\):

\[
F_\Lambda
=
g_R
+d\,g_R^2
+O(g_R^3),
\]

so the explicit logarithmic cutoff dependence cancels at this order.
This is the concrete implementation of the Section 8 rule: tune bare
data so predictions at reference scale remain stable.

\texttt{Derivation\ D8.2\ (Running\ from\ cutoff-independence).} At
fixed bare coupling \(g_B\), differentiating the renormalization
condition gives

\[
\mu\frac{d g_R}{d\mu}
=
-c\,g_R^2+O(g_R^3)
\equiv
\beta(g_R).
\]

This turns divergent cutoff dependence into controlled scale dependence.

\texttt{Proposition\ P8.1\ (Leading\ beta\ coefficient\ under\ analytic\ scheme\ change).}
For a reparametrization \(g_R' = g_R + a\,g_R^2 + O(g_R^3)\), the
leading coefficient of \(\beta\) is unchanged:

\[
\beta'(g_R')=-c\,{g_R'}^2+O({g_R'}^3).
\]

So scheme changes shift higher-order terms while preserving the first
nontrivial flow coefficient in this template.

\subsection{10.2 Ordering/Discretization Pair with Same Classical
Limit}\label{orderingdiscretization-pair-with-same-classical-limit}

This appendix gives an explicit example of the Section 6/Section 7 claim
that discretization choice changes \(O(\hbar)\) terms while preserving
classical dynamics.

Take the classical symbol \(A(q,p)=f(q)p\), with smooth \(f\). Consider
two quantizations: 1. Left ordering: \(Q_L(A)=f(\hat q)\hat p\). 2. Weyl
(symmetric) ordering:
\(Q_W(A)=\frac12\left(f(\hat q)\hat p+\hat p f(\hat q)\right)\).

Using \([\hat p,f(\hat q)]=-i\hbar f'(\hat q)\):

\[
Q_W(A)
=
f(\hat q)\hat p
-\frac{i\hbar}{2}f'(\hat q)
=
Q_L(A)-\frac{i\hbar}{2}f'(\hat q).
\]

\texttt{Derivation\ D9.1\ (Classical\ agreement,\ quantum\ shift).} The
difference operator is \(O(\hbar)\):

\[
Q_W(A)-Q_L(A)=-\frac{i\hbar}{2}f'(\hat q).
\]

Therefore

\[
\lim_{\hbar\to0}\big(Q_W(A)-Q_L(A)\big)=0
\]

in the formal classical limit, while quantum generators differ at
subleading order.

\texttt{Proposition\ P9.1\ (Discretization-ordering\ equivalence\ class\ statement).}
If two short-time kernel prescriptions map to \(Q_L\)-type and
\(Q_W\)-type representatives of the same classical symbol algebra, then
they define the same classical equations and differ only by controlled
\(O(\hbar)\) corrections. This is the worked version of the Section 6 to
Section 7 transition claim {[}Landsman1998{]}.

\subsection{10.3 Foundational Compatibility
Principle}\label{foundational-compatibility-principle}

This appendix states a foundational compatibility principle aimed at the
manuscript's core objective.

\texttt{Proposition\ P10.1\ (Refinement\ Compatibility\ Principle,\ RCP).}
A dynamical framework is admissible when three compatibility conditions
hold simultaneously: 1. \textbf{Partition compatibility}: composition
across temporal subdivisions preserves the same action-based extremal
equations in the refinement limit. 2. \textbf{Representation
compatibility}: alternative quantum representations
(ordering/discretization choices) agree in the classical limit and
differ only by controlled subleading corrections. 3. \textbf{Scale
compatibility}: observable predictions remain stable under composed
coarse/fine scale changes after parameter flow adjustment.

In compact form, for any prediction map \(\mathcal O\),

\[
\mathcal O
=
\mathcal O\circ\mathcal C_t
=
\mathcal O\circ\mathcal Q_\hbar
=
\mathcal O\circ\mathcal R_\Lambda,
\]

where \(\mathcal C_t\) is temporal composition/refinement,
\(\mathcal Q_\hbar\) is representation change within a fixed
classical-limit class, and \(\mathcal R_\Lambda\) is
scale-refinement/renormalization flow.

\texttt{Derivation\ D10.1a\ (Operational\ closure\ form:\ compatibility\ as\ “there\ exists\ \textbackslash{}(\textbackslash{}theta\textquotesingle{}\textbackslash{})”).}
The schematic equalities above suppress the crucial fact that each
operation generally requires a \textbf{parameter update} (coupling flow,
normalization change, or a controlled representation change) to land
back in the same admissible family. Concretely, the operators
\(\mathcal C_t,\mathcal Q_\hbar,\mathcal R_\Lambda\) can be instantiated
by indexed families: \(\mathcal C_t\) by a ``compose \(b\) fine steps
into one coarse step'' map \(\mathcal C_{b,h}\), \(\mathcal Q_\hbar\) by
a family \(\mathcal Q_\alpha\) of representation/ordering changes at
fixed \(\hbar\), and \(\mathcal R_\Lambda\) by
scale-compare/coarse-grain maps \(\mathcal R_b\) (with \(\Lambda\)
standing for whatever cutoff/subtraction convention parametrizes the
family being compared). Here \(h\) is the ruler at which we compare
predictions, \(b\) is a refinement/coarse-graining factor, and
\(\alpha\) labels a choice of representation within a fixed
classical-limit class. An operational way to state RCP is to write
predictions as a family \(\{\mathcal O_{h,\theta}\}\) indexed by \(h>0\)
and parameters \(\theta\), and require that for each operation there
exists an update map \(\tau\) such that the post-operation object is
again representable inside the same family: \[
\mathcal O_{h,\theta}
=
\mathcal O_{h,\tau_C(b,h;\theta)}\circ \mathcal C_{b,h}
=
\mathcal O_{h,\tau_Q(\alpha;\theta)}\circ \mathcal Q_\alpha
=
\mathcal O_{h,\tau_R(b;\theta)}\circ \mathcal R_b.
\] Written this way, compatibility is falsifiable: closure can fail when
no admissible \(\theta'\) exists. The manuscript's simplest explicit
witness is the free short-time kernel ansatz: demanding semigroup
closure under time-slice composition fixes the normalization exponent
\(A(t)\propto t^{-d/2}\) (Derivation D4.1a); choosing any other power
breaks closure.

\texttt{Derivation\ D10.1\ (Bridge\ to\ sections\ 3-\/-8).} 1.
\textbf{Partition compatibility} (\(\mathcal C_t\)): Sections 3--4
(area-law refinement; action/Noether bridge). 2. \textbf{Representation
compatibility} (\(\mathcal Q_\hbar\)): Sections 6--7
(ordering/discretization choices with identical \(\hbar\to0\) limit). 3.
\textbf{Scale compatibility} (\(\mathcal R_\Lambda\)): Section 8 (RG
semigroup consistency).

Therefore the Newton-to-path-integral narrative is an implementation of
RCP rather than a sequence of disconnected formalisms.

\texttt{Heuristic\ H10.1\ (Foundational\ reading).} RCP can be
interpreted as a candidate foundational postulate: physical laws are
those statements that survive controlled changes of partition,
representation, and scale.

\subsection{10.4 Appendix Summary}\label{appendix-summary}

Appendix sections 10.1--10.3 close the three technical gaps identified
in Section 9: 1. explicit renormalization subtraction and running, 2.
explicit ordering/discretization \(O(\hbar)\) shift with fixed classical
limit, 3. explicit foundational compatibility principle unifying the
full chain.

These additions do not alter the thesis; they increase computational
accountability of the existing chain.

\subsection{10.5 Singular Contact Interaction as an Explicit RG
Computation (2D
Delta)}\label{singular-contact-interaction-as-an-explicit-rg-computation-2d-delta}

Section 8 argues that RG is the scale-compatibility condition required
when refinement limits diverge. This appendix supplies a fully explicit
example in a singular quantum-mechanical model where the continuum
theory is defined only after a renormalization prescription is chosen.
For a perturbative-QFT-style treatment of this mechanism in quantum
mechanics (including the 2D delta interaction), see
{[}ManuelTarrach1994PertRenQM{]}. For a standard discussion of
delta-function potentials in two and three dimensions (and their
renormalization issues), see {[}Jackiw1991DeltaPotentials{]}.

Consider the two-dimensional contact interaction

\[
H
=
-\frac{\hbar^2}{2m}\Delta
+g\,\delta^{(2)}(x)
\quad \text{on }\mathbb R^2.
\]

The interaction is Dirac-supported and the naive continuum limit is
ill-defined: loop integrals diverge logarithmically.

\texttt{Derivation\ D11.1\ (Cutoff\ evaluation\ of\ the\ contact\ loop).}
Let \(E>0\) and write \(E=\hbar^2 k^2/(2m)\). The Lippmann--Schwinger
equation yields an algebraic \(T\)-matrix

\[
T(E;\Lambda)
=
\frac{1}{g_B(\Lambda)^{-1}-I(E;\Lambda)},
\]

where the loop integral is the free resolvent at the origin with a
wavevector cutoff \(|q|<\Lambda\):

\[
I(E;\Lambda)
=
\int_{|q|<\Lambda}\frac{d^2q}{(2\pi)^2}\,
\frac{1}{E-\frac{\hbar^2 q^2}{2m}+i0}
=
-\frac{m}{2\pi\hbar^2}\left[\ln\!\left(\frac{\Lambda^2}{k^2}\right)+i\pi\right]
+O\!\left(\frac{k^2}{\Lambda^2}\right).
\]

Thus the regulated theory contains a logarithmic divergence
\(\sim -\frac{m}{2\pi\hbar^2}\ln\Lambda^2\).

\texttt{Derivation\ D11.2\ (Renormalized\ coupling\ and\ beta\ function).}
Define a renormalized coupling at subtraction scale \(\mu\) by

\[
\frac{1}{g_R(\mu)}
\equiv
\frac{1}{g_B(\Lambda)}
+\frac{m}{2\pi\hbar^2}\ln\!\left(\frac{\Lambda^2}{\mu^2}\right).
\]

Substituting into \(T(E;\Lambda)\) cancels the explicit cutoff
dependence and gives a finite amplitude:

\[
T(E)
=
\frac{1}{
\frac{1}{g_R(\mu)}
+\frac{m}{2\pi\hbar^2}\ln\!\left(\frac{\mu^2}{k^2}\right)
+ i\,\frac{m}{2\hbar^2}
}.
\]

Since \(\mu\) is arbitrary, physical predictions must satisfy
\(dT/d\ln\mu=0\). This yields the RG equation

\[
\mu\frac{d}{d\mu}\left(\frac{1}{g_R(\mu)}\right)
=-\frac{m}{\pi\hbar^2},
\qquad
\beta(g_R)\equiv \mu\frac{d g_R}{d\mu}
=\frac{m}{\pi\hbar^2}\,g_R^2.
\]

This is the explicit ``scale-compatibility vector field'' promised by
Section 8, obtained from the demand that the subtraction scale not
affect the composed prediction.

\texttt{Proposition\ P11.1\ (Dimensional\ transmutation:\ an\ RG-invariant\ bound-state\ scale).}
For \(E<0\), write \(E=-\hbar^2\kappa^2/(2m)\). The bound state
corresponds to a pole of \(T\), which occurs when

\[
\frac{1}{g_R(\mu)}
+\frac{m}{2\pi\hbar^2}\ln\!\left(\frac{\mu^2}{\kappa^2}\right)
=0.
\]

Define

\[
\kappa_\ast^2
\equiv
\mu^2\exp\!\left(\frac{2\pi\hbar^2}{m}\frac{1}{g_R(\mu)}\right).
\]

Using the RG equation for \(1/g_R(\mu)\), one checks
\(d\kappa_\ast/d\mu=0\). Thus the renormalized delta interaction trades
the regulator-dependent coupling for a physical scale \(\kappa_\ast\)
(equivalently a bound-state energy \(E_B=\hbar^2\kappa_\ast^2/(2m)\)).

\texttt{Derivation\ D11.3\ (Scheme\ dependence\ as\ rescaling\ of\ the\ transmutation\ scale).}
The subtraction defining \(g_R(\mu)\) is not unique: one may shift it by
a finite constant \(C\) by defining

\[
\frac{1}{g_R^{(C)}(\mu)}
\equiv
\frac{1}{g_R(\mu)}+\frac{m}{2\pi\hbar^2}C.
\]

Differentiation in \(\ln\mu\) removes the constant, so the beta function
is unchanged. However, the RG-invariant scale rescales:

\[
\kappa_{\ast}^{(C)\,2}
\equiv
\mu^2\exp\!\left(\frac{2\pi\hbar^2}{m}\frac{1}{g_R^{(C)}(\mu)}\right)
=e^{C}\,\kappa_\ast^2.
\]

Thus, in this one-scale model, ``scheme dependence'' is precisely the
freedom to rescale the single physical scale. Fixing one physical datum
(e.g.~\(E_B\)) fixes \(\kappa_\ast\) and removes the ambiguity from
predictions.

\section{References}\label{references}

\begin{enumerate}
\def\labelenumi{\arabic{enumi}.}
\tightlist
\item
  {[}Newton1687{]} Isaac Newton, \emph{Philosophiae Naturalis Principia
  Mathematica} (1687), Book I (limit methods and central-force
  geometry). Reference access point: Newton Project and standard
  translated editions.
\item
  {[}Noether1918{]} Emmy Noether, ``Invariante Variationsprobleme''
  (1918). English translation commonly used in modern mechanics
  references.
\item
  {[}Dirac1933{]} P. A. M. Dirac, ``The Lagrangian in Quantum
  Mechanics,'' \emph{Physikalische Zeitschrift der Sowjetunion} 3
  (1933), 64--72. (Commonly cited as the earliest explicit path-integral
  style phase-weighting statement.)
\item
  {[}Feynman1948{]} R. P. Feynman, ``Space-Time Approach to
  Non-Relativistic Quantum Mechanics,'' \emph{Reviews of Modern Physics}
  20 (1948), 367--387. DOI \texttt{10.1103/RevModPhys.20.367}. OA:
  Caltech author-repository PDF.
\item
  {[}Connes1994{]} Alain Connes, \emph{Noncommutative Geometry}
  (Academic Press, 1994). ISBN \texttt{978-0-12-185860-5}.
\item
  {[}Landsman1998{]} N. P. Landsman, \emph{Mathematical Topics Between
  Classical and Quantum Mechanics} (Springer, 1998), Springer Monographs
  in Mathematics. DOI \texttt{10.1007/978-1-4612-1680-3}. ISBN
  \texttt{978-0-387-98318-9}.
\item
  {[}ConnesKreimer2000{]} Alain Connes and Dirk Kreimer,
  ``Renormalization in quantum field theory and the Riemann-Hilbert
  problem I,'' \emph{Communications in Mathematical Physics} 210 (2000),
  249--273. DOI \texttt{10.1007/s002200050779}.
\item
  {[}Brouder1999{]} Ch. Brouder, ``Runge-Kutta methods and
  renormalization,'' arXiv:\texttt{hep-th/9904014} (2 Apr 1999).
\item
  {[}McLachlan2017{]} Robert I. McLachlan, Klas Modin, Hans Munthe-Kaas,
  Olivier Verdier, ``Butcher series: A story of rooted trees and
  numerical methods for evolution equations,'' arXiv:\texttt{1512.00906}
  (v3, 27 Feb 2017).
\item
  {[}BoyaRivero1994Contact{]} Luis J. Boya and Alejandro Rivero,
  ``Renormalization in 1-D Quantum Mechanics: contact interactions,''
  arXiv:\texttt{hep-th/9411081} (v1, 11 Nov 1994).
\item
  {[}ManuelTarrach1994PertRenQM{]} Cristina Manuel and Rolf Tarrach,
  ``Perturbative Renormalization in Quantum Mechanics,'' \emph{Physics
  Letters B} 328 (1994), 113--118. arXiv:\texttt{hep-th/9309013} (v1, 2
  Sep 1993). DOI \texttt{10.1016/0370-2693(94)90437-5}.
\item
  {[}Jackiw1991DeltaPotentials{]} R. Jackiw, ``Delta-function potentials
  in two- and three-dimensional quantum mechanics,'' MIT-CTP-1937 (Jan
  1991). Reprinted in \emph{M.A.B. Bég Memorial Volume} (World
  Scientific, 1991), pp.~25--42. OA mirror:
  \url{https://www.physics.smu.edu/scalise/P6335fa21/notes/Jackiw.pdf}.
\item
  {[}VanVleck1928Correspondence{]} J. H. Van Vleck, ``The Correspondence
  Principle in the Statistical Interpretation of Quantum Mechanics,''
  \emph{Proceedings of the National Academy of Sciences of the United
  States of America} 14(2) (1928), 178--188. DOI
  \texttt{10.1073/pnas.14.2.178}.
\item
  {[}deGosson2018ShortTimePropagators{]} Maurice A. de Gosson,
  ``Short-Time Propagators and the Born--Jordan Quantization Rule,''
  \emph{Entropy} 20(11) (2018), 869. DOI \texttt{10.3390/e20110869}. OA:
  PubMed Central (PMCID: PMC 7512447).
\item
  {[}Rosten2012ERG{]} Oliver J. Rosten, ``Fundamentals of the Exact
  Renormalization Group,'' \emph{Physics Reports} 511 (2012), 177--272.
  arXiv:\texttt{1003.1366} (v4, 15 Feb 2012). DOI
  \texttt{10.1016/j.physrep.2011.12.003}.
\item
  {[}Velhinho2017InfDimMeasure{]} José Velhinho, ``Topics of Measure
  Theory on Infinite Dimensional Spaces,'' \emph{Mathematics} 5(3)
  (2017), 44. DOI \texttt{10.3390/math5030044}. OA: MDPI.
\item
  {[}TongQMLectures{]} David Tong, ``Quantum Mechanics'' (lecture notes,
  no DOI). OA: lecture-note PDF. (Contains
  \(Y_{l,m}(\theta,\phi)=P_{l,m}(\cos\theta)e^{im\phi}\) as simultaneous
  eigenstates of \(L^2\) and \(L_z\).)
\end{enumerate}

\end{document}
