% Options for packages loaded elsewhere
\PassOptionsToPackage{unicode}{hyperref}
\PassOptionsToPackage{hyphens}{url}
\documentclass[
]{article}
\usepackage{xcolor}
\usepackage{amsmath,amssymb}
\setcounter{secnumdepth}{-\maxdimen} % remove section numbering
\usepackage{iftex}
\ifPDFTeX
  \usepackage[T1]{fontenc}
  \usepackage[utf8]{inputenc}
  \usepackage{textcomp} % provide euro and other symbols
\else % if luatex or xetex
  \usepackage{unicode-math} % this also loads fontspec
  \defaultfontfeatures{Scale=MatchLowercase}
  \defaultfontfeatures[\rmfamily]{Ligatures=TeX,Scale=1}
\fi
\usepackage{lmodern}
\ifPDFTeX\else
  % xetex/luatex font selection
\fi
% Use upquote if available, for straight quotes in verbatim environments
\IfFileExists{upquote.sty}{\usepackage{upquote}}{}
\IfFileExists{microtype.sty}{% use microtype if available
  \usepackage[]{microtype}
  \UseMicrotypeSet[protrusion]{basicmath} % disable protrusion for tt fonts
}{}
\makeatletter
\@ifundefined{KOMAClassName}{% if non-KOMA class
  \IfFileExists{parskip.sty}{%
    \usepackage{parskip}
  }{% else
    \setlength{\parindent}{0pt}
    \setlength{\parskip}{6pt plus 2pt minus 1pt}}
}{% if KOMA class
  \KOMAoptions{parskip=half}}
\makeatother
\setlength{\emergencystretch}{3em} % prevent overfull lines
\providecommand{\tightlist}{%
  \setlength{\itemsep}{0pt}\setlength{\parskip}{0pt}}
\usepackage{bookmark}
\IfFileExists{xurl.sty}{\usepackage{xurl}}{} % add URL line breaks if available
\urlstyle{same}
\hypersetup{
  pdftitle={Half-Densities in QFT: Propagators as Bi-Half-Density Kernels},
  hidelinks,
  pdfcreator={LaTeX via pandoc}}

\title{Half-Densities in QFT: Propagators as Bi-Half-Density Kernels}
\author{}
\date{}

\begin{document}
\maketitle
\begin{abstract}
In QFT, the basic free object is the inverse of a kinetic operator,
i.e.~a propagator/Green kernel. On a manifold, writing
``\(P_x G(x,y)=\delta(x,y)\)'' hides conventions: which volume form
defines the adjoint, and which delta normalization realizes the
identity. This note adopts an organizing choice consistent with the
repo's main paper: treat fields (or kernels) as \textbf{half-densities},
so the identity kernel is canonical and kernel composition is
coordinate-invariant without choosing a background measure. A worked
computation shows how a scalar field on \((M,g)\) becomes a half-density
\(\psi=|g|^{1/4}\phi\), with kinetic operator
\(\widetilde P=|g|^{1/4}P|g|^{-1/4}\) symmetric in the coordinate
pairing. We also record a kernel-level remark: local
counterterms/contact terms appear as distributions supported on the
diagonal \((x=y)\) (delta kernels and their derivatives), which are most
naturally expressed using the canonical bi-half-density delta.

This note is a dependent follow-up to \texttt{paper/main.md} and relates
to \texttt{papers/planck-area/main.md} (scalarization scales) and
\texttt{papers/rg-fundamental/main.md} (RG as compatibility).
\end{abstract}

\section{1. Purpose and Scope}\label{purpose-and-scope}

This note is intentionally narrow: 1. establish the ``kernel as
bi-half-density'' semantics for spacetime propagators in QFT, 2. isolate
what is \textbf{canonical} (half-density kernels, identity delta kernel)
versus what is \textbf{a convention} (scalarization choices such as
\(\sqrt{|g|}\)), 3. give one explicit computation that can later be
promoted (densitized scalar field).

BV/BRST/field-space half-densities are only flagged as outlook here; a
full treatment would require additional dedicated sources and is beyond
scope.

\section{2. Kernels on a Manifold: half-densities make the identity
canonical}\label{kernels-on-a-manifold-half-densities-make-the-identity-canonical}

Let \(M\) be a \(D\)-dimensional manifold. A half-density is a section
of \(|\Lambda^D T^\ast M|^{1/2}\).

The key operational point (as in the main paper's kernel-composition
spine) is: if an operator acts on half-densities, then its Schwartz
kernel is naturally a \textbf{bi-half-density} \[
K_A(x,y)=a(x,y)\,|dx|^{1/2}|dy|^{1/2},
\] and composition is intrinsic: \[
(A\circ B)(x,z)=\int_M K_A(x,y)\,K_B(y,z),
\] because the product in the intermediate variable \(y\) is a density.

\texttt{Derivation\ D1.1\ (Identity\ kernel).} The identity operator on
half-densities has kernel \[
K_{\mathrm{Id}}(x,y)=\delta^{(D)}(x-y)\,|dx|^{1/2}|dy|^{1/2},
\] which is canonical: it does not require choosing a background
density/volume form.

This is the same object used repeatedly in the repo's half-density
scaling thread: the diagonal delta kernel is the universal
distributional witness of the \(\varepsilon^{-D/2}\) exponent (see
\texttt{papers/planck-area/main.md}, Derivation D1.2b).

\section{\texorpdfstring{3. Worked computation: densitized scalar field
\(\psi=|g|^{1/4}\phi\)}{3. Worked computation: densitized scalar field \textbackslash psi=\textbar g\textbar\^{}\{1/4\}\textbackslash phi}}\label{worked-computation-densitized-scalar-field-psig14phi}

Consider a real scalar field on a fixed Lorentzian/Euclidean background
\((M,g)\) with quadratic action \[
S[\phi]=\frac12\int_M d^Dx\,\sqrt{|g|}\;\phi\,P\,\phi,
\qquad
P=-\nabla^2 + m^2 + \xi R \ \ (\text{example}).
\] The pairing for which \(P\) is symmetric is \[
(\phi_1,\phi_2)_g=\int d^Dx\,\sqrt{|g|}\;\phi_1\phi_2.
\]

Define the densitized field (a half-density in coordinates) \[
\psi := |g|^{1/4}\phi,
\qquad\text{so}\qquad
\phi=|g|^{-1/4}\psi.
\] Then the action becomes \[
S[\phi]
=\frac12\int d^Dx\;\psi\;\widetilde P\;\psi,
\qquad
\widetilde P := |g|^{1/4}P|g|^{-1/4},
\] so the pairing is now just the coordinate density \(d^Dx\).

\texttt{Derivation\ D1.2\ (Explicit\ form\ of\ the\ conjugated\ kinetic\ operator).}
For \(P_{\mathrm{kin}}=-\nabla^2\) one has in coordinates \[
\nabla^2\phi=|g|^{-1/2}\partial_\mu\!\Bigl(\sqrt{|g|}\,g^{\mu\nu}\partial_\nu\phi\Bigr),
\] hence \[
\widetilde P_{\mathrm{kin}}
=-|g|^{1/4}\nabla^2|g|^{-1/4}
=-|g|^{-1/4}\partial_\mu\!\Bigl(\sqrt{|g|}\,g^{\mu\nu}\partial_\nu\bigl(|g|^{-1/4}\,\cdot\,\bigr)\Bigr).
\] Assuming compact support (or boundary conditions killing boundary
terms), \[
\int d^Dx\;\psi_1\,\widetilde P_{\mathrm{kin}}\psi_2
=\int d^Dx\;\sqrt{|g|}\,g^{\mu\nu}\,
\partial_\mu\bigl(|g|^{-1/4}\psi_1\bigr)\,
\partial_\nu\bigl(|g|^{-1/4}\psi_2\bigr),
\] which is manifestly symmetric under \((1\leftrightarrow 2)\).

\texttt{Derivation\ D1.3\ (Conformal\ metric\ expansion;\ D=4\ simplification\ in\ the\ conformal\ class).}
As a worked expansion, take a conformally flat background
\(g_{\mu\nu}=e^{2\sigma(x)}\delta_{\mu\nu}\) (Euclidean for simplicity).
Then \(\sqrt{|g|}=e^{D\sigma}\), \(|g|^{1/4}=e^{D\sigma/2}\),
\(g^{\mu\nu}=e^{-2\sigma}\delta^{\mu\nu}\), and one finds \[
\widetilde\Delta\psi:=|g|^{1/4}\Delta_g|g|^{-1/4}\psi
=e^{-2\sigma}\Big(
\partial^2\psi
-2\,\partial\sigma\cdot\partial\psi
-\frac{D}{2}(\partial^2\sigma)\,\psi
+\frac{D(4-D)}{4}(\partial\sigma)^2\,\psi
\Big),
\] so the kinetic operator
\(\widetilde P_{\mathrm{kin}}=-\widetilde\Delta\) contains a term
proportional to \(D(4-D)(\partial\sigma)^2\), which cancels at \(D=4\)
in this conformal ansatz. This is recorded only as a concrete
simplification in a natural metric class; its broader meaning (if any)
is a separate question. For the derivation and a SymPy coefficient/sign
check, see
\texttt{blackboards/2026-02-10-half-density-laplacian-conformal-metric.md}.

Interpretation: - the metric half-density \(|g|^{1/4}|dx|^{1/2}\) is a
\textbf{scalarization gauge} (a choice of reference half-density) on a
fixed background, - writing the field as \(\psi\) makes the
``half-density prioritary'' viewpoint explicit: both the field and the
kernels live naturally as half-density objects.

\section{4. Propagators/Green functions as bi-half-density
kernels}\label{propagatorsgreen-functions-as-bi-half-density-kernels}

Let \(\widetilde G\) be the inverse kernel of \(\widetilde P\) with
respect to the coordinate pairing: \[
(\widetilde P^{-1}f)(x)=\int \widetilde G(x,y)\,f(y)\,d^Dy,
\qquad
\widetilde P_x\,\widetilde G(x,y)=\delta^{(D)}(x-y).
\] Then the corresponding canonical bi-half-density kernel is \[
K_{\widetilde P^{-1}}(x,y)=\widetilde G(x,y)\,|dx|^{1/2}|dy|^{1/2}.
\]

Equivalently, if \(G_g(x,y)\) denotes the usual \textbf{scalar} Green
function for \(P\) defined with respect to the metric pairing
\(\int \sqrt{|g|}\,d^Dy\) (so
\((P^{-1}J)(x)=\int G_g(x,y)\,J(y)\,\sqrt{|g(y)|}\,d^Dy\)), then the
kernels are related by \[
\widetilde G(x,y)=|g(x)|^{1/4}\,G_g(x,y)\,|g(y)|^{1/4}.
\]

This is exactly the same ``kernel as bi-half-density'' structure used
for QM propagators in the main manuscript, now applied to spacetime
Green functions in QFT.

\texttt{Remark\ D4.1\ (Doubling:\ densities\ live\ on\ \textbackslash{}(M\textbackslash{}times\ M\textbackslash{})).}
Half-density kernels also make the amplitude-vs-density doubling
completely explicit. Let \(U_t\) be a (unitary) evolution operator on
half-densities with kernel \(K_t(x,y)\). Then a density operator
\(\rho_t=U_t\rho_0U_t^{-1}\) has a kernel satisfying \[
K_{\rho_t}(x,y)
=\int_{M\times M} K_t(x,x')\,K_{\rho_0}(x',y')\,\overline{K_t(y,y')}.
\] So densities naturally propagate by a kernel on the doubled space
\(M\times M\), built from the forward kernel and its conjugate. This is
the kernel-level origin of bra/ket (forward/backward) doubling in real
expectation values; a fuller discussion is beyond this note's scope.

\section{5. Contact terms and counterterms as diagonal delta kernels
(kernel-level
remark)}\label{contact-terms-and-counterterms-as-diagonal-delta-kernels-kernel-level-remark}

In QFT, divergences are removed by adding local counterterms to the
action, e.g. \(\delta m^2\,\phi^2\), \(\delta Z\,(\partial\phi)^2\),
curvature couplings, etc. At the operator/kernel level this means: the
kinetic operator \(P\) is modified by local (differential) operators,
and therefore its kernel acquires \textbf{distributions supported on the
diagonal} \(x=y\).

In the half-density kernel language the diagonal object is canonical: \[
K_{\mathrm{Id}}(x,y)=\delta^{(D)}(x-y)\,|dx|^{1/2}|dy|^{1/2}.
\] Multiplication counterterms correspond to
\(c(x)\,K_{\mathrm{Id}}(x,y)\), and derivative counterterms correspond
to derivatives acting on the delta kernel
(e.g.~\(\partial_x\delta^{(D)}(x-y)\) and higher).

\texttt{Remark\ D5.1\ (Derivative\ of\ the\ diagonal\ delta\ kernel;\ coordinate-free\ identity).}
The slogan ``\(\partial_x\delta(x-y)=-\partial_y\delta(x-y)\)'' has a
clean, connection-free formulation in the half-density kernel calculus.
The identity kernel \(K_{\mathrm{Id}}\) is invariant under diagonal
diffeomorphisms \((\Phi\times\Phi)\), so for any vector field \(V\) on
\(M\) one has \[
(\mathcal L_{V_x}+\mathcal L_{V_y})\,K_{\mathrm{Id}}=0,
\qquad\text{hence}\qquad
\mathcal L_{V_x}K_{\mathrm{Id}}=-\mathcal L_{V_y}K_{\mathrm{Id}},
\] where \(\mathcal L\) is the Lie derivative acting on half-densities.
In local coordinates, taking \(V=\partial/\partial x^\mu\) recovers
\(\partial_{x^\mu}\delta^{(D)}(x-y)=-\partial_{y^\mu}\delta^{(D)}(x-y)\).
This is the kernel-level mechanism behind ``moving derivatives between
slots'' in integration by parts and in contact-term identities. For a
compact derivation and context, see
\texttt{blackboards/2026-02-10-derivatives-of-diagonal-delta-kernel.md}.

\texttt{Remark\ D5.2\ (Point\ splitting\ produces\ \textbackslash{}u03b4\textquotesingle{}\ contact\ terms).}
Point splitting makes the simplest derivative contact term explicit: in
one dimension, \[
\frac{\delta(x+\varepsilon)-\delta(x)}{\varepsilon}\xrightarrow[\varepsilon\to 0]{}\delta'(x),
\qquad \langle\delta',\varphi\rangle=-\varphi'(0).
\] This limit is the distributional companion of the ``difference
quotient as divergence + subtraction'' toy model; see
\texttt{blackboards/2026-02-10-difference-quotients-counterterms-and-delta-prime.md}.

This framing is useful for two reasons: 1. it makes ``locality =
diagonal support'' literal at the kernel level, 2. it separates the
canonical distributional objects from scheme-dependent scalarizations
and finite-subtraction conventions.

For a quick distribution-theory dictionary distinguishing \(\delta\),
\(\delta'\), and \(\delta(f')\), see
\texttt{blackboards/2026-02-10-distribution-theory-for-extremals.md}.

\section{6. Link to the half-density scale program (where Planck-area
enters
conditionally)}\label{link-to-the-half-density-scale-program-where-planck-area-enters-conditionally}

On a fixed background \((M,g)\), the metric provides a natural reference
half-density \(|g|^{1/4}|dx|^{1/2}\). The Planck-area program begins
only when we ask for a \textbf{background-free} scalarization convention
that turns half-density amplitudes into universal dimensionless numbers.
That stronger hypothesis ladder is developed in
\texttt{papers/planck-area/main.md}.

This paper's role is only to show that half-densities are not a QM
quirk: the same kernel semantics is already present in standard QFT
propagator definitions, once the hidden measure conventions are made
explicit.

\section{7. Outlook: BV half-densities}\label{outlook-bv-half-densities}

Gauge theories suggest a second, deeper appearance of half-densities:
the BV formalism treats the integrand as a (half-)density on an (odd)
symplectic space of fields/antifields, and the quantum master equation
expresses independence of gauge-fixing choices. This note does not
develop BV beyond this remark; doing so responsibly would require
additional authoritative sources and a separate dedicated treatment.

\end{document}
