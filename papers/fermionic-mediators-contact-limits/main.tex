% Options for packages loaded elsewhere
\PassOptionsToPackage{unicode}{hyperref}
\PassOptionsToPackage{hyphens}{url}
%
\documentclass[
]{article}
\usepackage{amsmath,amssymb}
\usepackage{lmodern}
\usepackage{iftex}
\ifPDFTeX
  \usepackage[T1]{fontenc}
  \usepackage[utf8]{inputenc}
  \usepackage{textcomp} % provide euro and other symbols
\else % if luatex or xetex
  \usepackage{unicode-math}
  \defaultfontfeatures{Scale=MatchLowercase}
  \defaultfontfeatures[\rmfamily]{Ligatures=TeX,Scale=1}
\fi
% Use upquote if available, for straight quotes in verbatim environments
\IfFileExists{upquote.sty}{\usepackage{upquote}}{}
\IfFileExists{microtype.sty}{% use microtype if available
  \usepackage[]{microtype}
  \UseMicrotypeSet[protrusion]{basicmath} % disable protrusion for tt fonts
}{}
\makeatletter
\@ifundefined{KOMAClassName}{% if non-KOMA class
  \IfFileExists{parskip.sty}{%
    \usepackage{parskip}
  }{% else
    \setlength{\parindent}{0pt}
    \setlength{\parskip}{6pt plus 2pt minus 1pt}}
}{% if KOMA class
  \KOMAoptions{parskip=half}}
\makeatother
\usepackage{xcolor}
\IfFileExists{xurl.sty}{\usepackage{xurl}}{} % add URL line breaks if available
\IfFileExists{bookmark.sty}{\usepackage{bookmark}}{\usepackage{hyperref}}
\hypersetup{
  pdftitle={Fermionic Mediators, Static Potentials, and Contact/Boundary-Condition Limits},
  pdfauthor={Alejandro Rivero},
  hidelinks,
  pdfcreator={LaTeX via pandoc}}
\urlstyle{same} % disable monospaced font for URLs
\setlength{\emergencystretch}{3em} % prevent overfull lines
\providecommand{\tightlist}{%
  \setlength{\itemsep}{0pt}\setlength{\parskip}{0pt}}
\setcounter{secnumdepth}{-\maxdimen} % remove section numbering
\ifLuaTeX
  \usepackage{selnolig}  % disable illegal ligatures
\fi

\title{Fermionic Mediators, Static Potentials, and
Contact/Boundary-Condition Limits}
\author{Alejandro Rivero}
\date{2026}

\begin{document}
\maketitle
\begin{abstract}
The textbook derivation of a static potential from ``field exchange''
uses a bosonic mediator linearly sourced by a commuting classical
density, yielding an effective action quadratic in the source and (in a
static limit) a central Yukawa/Coulomb potential. This derivation does
not transplant verbatim to fermionic fields: the linear source terms for
fermions require Grassmann-valued sources, so there is no ordinary
commuting classical source whose elimination produces a classical
potential in the same way. This note isolates the precise obstruction
and records the robust infrared replacement: when a microscopic
description reduces to local operators at low resolution, the effective
interaction is encoded by contact terms (delta kernels and their
derivatives) or, equivalently, boundary-condition/self-adjoint-extension
data, with renormalization-group running when the contact limit is
singular.

This is a dependent note aligned with the broader
refinement-compatibility program: contact terms are diagonal-support
kernels, and their scale dependence is a compatibility condition rather
than an afterthought.
\end{abstract}

\hypertarget{purpose-and-scope}{%
\section{1. Purpose and scope}\label{purpose-and-scope}}

This note answers a narrowly phrased question: what can it mean for a
\textbf{fermionic} field to ``generate a (central) potential'' in the
same sense that a massive bosonic field generates a Yukawa potential?

We keep the scope bounded: 1. state the bosonic sourcing \(\Rightarrow\)
potential mechanism (derivation-first, brief), 2. state the fermionic
obstruction precisely (Grassmann sources), 3. give one explicit IR
matching witness: \textbf{local operators \(\Rightarrow\)
contact/derivative-contact kernels}, 4. connect contact kernels to
related point-interaction/RG witnesses.

We do \textbf{not} claim that fermions cannot affect forces; we only
isolate which parts of the ``classical source \(\Rightarrow\)
potential'' story fail, and what the correct replacement statement is at
low resolution.

\hypertarget{what-a-field-generates-a-potential-means-in-the-bosonic-source-story}{%
\section{2. What ``a field generates a potential'' means in the bosonic
source
story}\label{what-a-field-generates-a-potential-means-in-the-bosonic-source-story}}

The archetypal construction is a bosonic mediator \(\varphi\) linearly
coupled to a commuting source \(J(x)\): \[
S[\varphi;J]=\int d^Dx\left(\frac12\,\varphi\,K\,\varphi + J\,\varphi\right),
\qquad K=(\Box+m^2)\ \text{(example)}.
\] Integrating out \(\varphi\) (Gaussian elimination) yields an
effective action quadratic in the source, \[
S_{\mathrm{eff}}[J]=-\frac12\int d^Dx\,d^Dy\; J(x)\,K^{-1}(x,y)\,J(y),
\] so the static, nonrelativistic limit of \(K^{-1}\) produces a central
potential (Yukawa for \(m\neq 0\), Coulomb-type when \(m=0\)).

The key structural ingredient is that the source is an ordinary
commuting function (a classical background density).

\hypertarget{fermionic-fields-linear-sources-are-grassmann-so-the-classical-source-story-does-not-transplant}{%
\section{3. Fermionic fields: linear sources are Grassmann, so the
classical-source story does not
transplant}\label{fermionic-fields-linear-sources-are-grassmann-so-the-classical-source-story-does-not-transplant}}

For a Dirac fermion \(\Psi\), the generating functional with sources is
written with \textbf{Grassmann-valued} sources \(\eta,\bar\eta\): \[
Z[\bar\eta,\eta]
=\int D\bar\Psi\,D\Psi\;
\exp\!\left(
i\int d^Dx\;\bar\Psi\,(i\gamma^\mu\partial_\mu-m)\Psi
+i\int d^Dx\;(\bar\eta\,\Psi+\bar\Psi\,\eta)
\right).
\] An explicit statement of this form, including that \(\eta,\bar\eta\)
are Grassmann-valued, is recorded in
{[}Floerchinger2024QFT1Lecture21{]}.

Two immediate consequences follow.

\texttt{Remark\ 3.1\ (Obstruction\ statement).} The bosonic derivation
``choose a commuting classical source \(J\), integrate out the field,
and read off a classical potential'' does not directly apply to
fermions, because the linear source terms that couple to \(\Psi\)
require Grassmann sources rather than commuting c-number densities.
Therefore, ``fermion exchange generates a classical potential between
commuting sources'' is not a well-posed transplant of the bosonic story.

This does \textbf{not} mean fermions are irrelevant: fermions can and do
affect effective interactions through loop effects, through bosonic
composite modes (bilinears), and through low-energy EFT operators. The
point is that the meaning of ``generates a potential'' must be stated
through one of these controlled mechanisms.

\hypertarget{the-controlled-alternative-fermion-loops-modify-bosonic-propagators}{%
\subsection{3.1 The controlled alternative: fermion loops modify bosonic
propagators}\label{the-controlled-alternative-fermion-loops-modify-bosonic-propagators}}

The standard example is vacuum polarization in quantum electrodynamics.
A closed electron--positron loop inserted into the photon propagator
gives a momentum-dependent correction to the effective electromagnetic
coupling, \[
\alpha_{\mathrm{eff}}(q^2)
=
\frac{\alpha}{1-\Pi(q^2)},
\qquad
\Pi(q^2)
=
-\frac{\alpha}{3\pi}\ln\!\left(\frac{q^2}{\mu^2}\right)+\cdots,
\] where \(\Pi(q^2)\) is the vacuum polarization function (the photon
self-energy from a fermion one-loop diagram). At low momentum transfer
(\(|q|\ll m_e\)), the loop correction is analytic in \(q^2\) and
generates precisely the local operators \(C_0 + C_2 q^2 + \cdots\)
discussed in Section 4 below.

The structural point: fermions affect forces, but the path from
``fermion field'' to ``effective interaction'' runs through a quantum
loop (not through a tree-level Gaussian elimination of a classical
source), and the low-energy residue takes the form of local/contact
operators.

\hypertarget{ir-replacement-local-operators-rightarrow-contact-kernels-boundary-condition-data}{%
\section{\texorpdfstring{4. IR replacement: local operators
\(\Rightarrow\) contact kernels / boundary-condition
data}{4. IR replacement: local operators \textbackslash Rightarrow contact kernels / boundary-condition data}}\label{ir-replacement-local-operators-rightarrow-contact-kernels-boundary-condition-data}}

At low resolution, integrating out heavy degrees of freedom typically
produces local operators. In a two-body, nonrelativistic sector, this
appears as an amplitude expansion analytic in momentum transfer \(q\):
\[
\mathcal A(q)=C_0+C_2 q^2+O(q^4).
\]

The coordinate-space interaction associated to such an analytic
expansion is distributional and diagonal-supported. The invariant core
is a Fourier-transform identity: \[
\int \frac{d^d q}{(2\pi)^d}\,e^{iq\cdot r}= \delta^{(d)}(r),
\qquad
\int \frac{d^d q}{(2\pi)^d}\,q^2\,e^{iq\cdot r}= -\nabla^2\delta^{(d)}(r).
\]

\texttt{Derivation\ 4.1\ (Contact\ expansion\ gives\ \textbackslash{}(\textbackslash{}delta\textbackslash{})\ and\ derivative\ contacts).}
Interpreting the low-energy interaction kernel as the inverse Fourier
transform of \(\mathcal A(q)\) (Born-level language, up to overall
convention-dependent factors), we obtain \[
V(r)\ \propto\ \int \frac{d^d q}{(2\pi)^d}\,e^{iq\cdot r}\,\mathcal A(q)
\ \propto\
C_0\,\delta^{(d)}(r)\;-\;C_2\,\nabla^2\delta^{(d)}(r)\;+\;\cdots.
\] Thus locality at low energy naturally becomes \textbf{contact data}:
delta kernels and their derivatives supported at a point (or on the
diagonal, in kernel language).

In singular cases (notably \(\delta\) interactions in \(d\ge 2\) in
various channels), this contact data is not simply a fixed number: it is
defined by a renormalization condition and can generate RG-invariant
scales and bound states {[}Jackiw1991DeltaPotentials{]}
{[}ManuelTarrach1994PertRenQM{]}.

\hypertarget{boundary-condition-reading-point-interactions}{%
\section{5. Boundary-condition reading (point
interactions)}\label{boundary-condition-reading-point-interactions}}

Point-supported interactions can be encoded as self-adjoint extension /
boundary-condition data rather than as ordinary functions \(V(r)\). This
is the natural operator-theoretic counterpart of ``diagonal-support
kernels.'' For standard references and pedagogical framing, see
{[}BonneauFarautValent2001SAE{]} and the delta-potential discussion in
{[}Jackiw1991DeltaPotentials{]}.

This viewpoint matches the controlled-refinement perspective: when a
continuum description is defined as a refinement limit, UV data can
survive in the limit precisely as boundary-condition parameters (contact
terms), with RG flow expressing compatibility across resolutions.

\hypertarget{outlook-kept-minimal}{%
\section{6. Outlook (kept minimal)}\label{outlook-kept-minimal}}

Longer-range effects associated to fermionic degrees of freedom can
arise through loop-induced mechanisms or through emergent bosonic
composite modes. Treating those responsibly would require a separate
bibliography-hardening pass and is outside this note's scope.

\hypertarget{references}{%
\section{References}\label{references}}

\begin{enumerate}
\def\labelenumi{\arabic{enumi}.}
\tightlist
\item
  {[}ManuelTarrach1994PertRenQM{]} Cristina Manuel and Rolf Tarrach,
  ``Perturbative Renormalization in Quantum Mechanics,'' \emph{Physics
  Letters B} 328 (1994), 113--118. arXiv:\texttt{hep-th/9309013} (v1, 2
  Sep 1993). DOI \texttt{10.1016/0370-2693(94)90437-5}.
\item
  {[}BonneauFarautValent2001SAE{]} Guy Bonneau, Jacques Faraut, and
  Galliano Valent, ``Self-adjoint extensions of operators and the
  teaching of quantum mechanics,'' \emph{American Journal of Physics} 69
  (2001), 322--331. arXiv:\texttt{quant-ph/0103153}. DOI
  \texttt{10.1119/1.1328351}.
\item
  {[}Jackiw1991DeltaPotentials{]} R. Jackiw, ``Delta-function potentials
  in two- and three-dimensional quantum mechanics,'' MIT-CTP-1937 (Jan
  1991). Reprinted in \emph{M.A.B. Bég Memorial Volume} (World
  Scientific, 1991), pp.~25--42. OA mirror:
  \url{https://www.physics.smu.edu/scalise/P6335fa21/notes/Jackiw.pdf}.
\item
  {[}Floerchinger2024QFT1Lecture21{]} Stefan Floerchinger, ``Quantum
  field theory 1, lecture 21'' (updated 11 Jun 2024). (States the
  Dirac-fermion generating functional \(Z[\bar\eta,\eta]\) with
  Grassmann-valued sources.) OA: lecture webpage.
\end{enumerate}

\end{document}
