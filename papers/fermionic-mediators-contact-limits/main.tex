% Options for packages loaded elsewhere
\PassOptionsToPackage{unicode}{hyperref}
\PassOptionsToPackage{hyphens}{url}
%
\documentclass[
]{article}
\usepackage{amsmath,amssymb}
\usepackage{lmodern}
\usepackage{iftex}
\ifPDFTeX
  \usepackage[T1]{fontenc}
  \usepackage[utf8]{inputenc}
  \usepackage{textcomp} % provide euro and other symbols
\else % if luatex or xetex
  \usepackage{unicode-math}
  \defaultfontfeatures{Scale=MatchLowercase}
  \defaultfontfeatures[\rmfamily]{Ligatures=TeX,Scale=1}
\fi
% Use upquote if available, for straight quotes in verbatim environments
\IfFileExists{upquote.sty}{\usepackage{upquote}}{}
\IfFileExists{microtype.sty}{% use microtype if available
  \usepackage[]{microtype}
  \UseMicrotypeSet[protrusion]{basicmath} % disable protrusion for tt fonts
}{}
\makeatletter
\@ifundefined{KOMAClassName}{% if non-KOMA class
  \IfFileExists{parskip.sty}{%
    \usepackage{parskip}
  }{% else
    \setlength{\parindent}{0pt}
    \setlength{\parskip}{6pt plus 2pt minus 1pt}}
}{% if KOMA class
  \KOMAoptions{parskip=half}}
\makeatother
\usepackage{xcolor}
\IfFileExists{xurl.sty}{\usepackage{xurl}}{} % add URL line breaks if available
\IfFileExists{bookmark.sty}{\usepackage{bookmark}}{\usepackage{hyperref}}
\hypersetup{
  pdftitle={Fermionic Mediators, Static Potentials, and Contact/Boundary-Condition Limits},
  pdfauthor={Alejandro Rivero},
  hidelinks,
  pdfcreator={LaTeX via pandoc}}
\urlstyle{same} % disable monospaced font for URLs
\setlength{\emergencystretch}{3em} % prevent overfull lines
\providecommand{\tightlist}{%
  \setlength{\itemsep}{0pt}\setlength{\parskip}{0pt}}
\setcounter{secnumdepth}{-\maxdimen} % remove section numbering
\ifLuaTeX
  \usepackage{selnolig}  % disable illegal ligatures
\fi

\title{Fermionic Mediators, Static Potentials, and
Contact/Boundary-Condition Limits}
\author{Alejandro Rivero}
\date{2026}

\begin{document}
\maketitle
\begin{abstract}
The textbook derivation of a static potential from ``field exchange''
uses a bosonic mediator linearly sourced by a commuting classical
density, yielding an effective action quadratic in the source and (in a
static limit) a central Yukawa/Coulomb potential. This derivation does
not transplant verbatim to fermionic fields: the linear source terms for
fermions require Grassmann-valued sources, so there is no ordinary
commuting classical source whose elimination produces a classical
potential in the same way. This note isolates the precise obstruction
and records the robust infrared replacement: when a microscopic
description reduces to local operators at low resolution, the effective
interaction is encoded by contact terms (delta kernels and their
derivatives) or, equivalently, boundary-condition/self-adjoint-extension
data, with renormalization-group running when the contact limit is
singular.

This is a dependent note aligned with the broader
refinement-compatibility program: contact terms are diagonal-support
kernels, and their scale dependence is a compatibility condition rather
than an afterthought.
\end{abstract}

\hypertarget{purpose-and-scope}{%
\section{1. Purpose and scope}\label{purpose-and-scope}}

This note answers a narrowly phrased question: what can it mean for a
\textbf{fermionic} field to ``generate a (central) potential'' in the
same sense that a massive bosonic field generates a Yukawa potential?

We keep the scope bounded: 1. state the bosonic sourcing \(\Rightarrow\)
potential mechanism (derivation-first, brief), 2. state the fermionic
obstruction precisely (Grassmann sources), 3. give one explicit IR
matching witness: \textbf{local operators \(\Rightarrow\)
contact/derivative-contact kernels}, 4. connect contact kernels to
related point-interaction/RG witnesses.

We do \textbf{not} claim that fermions cannot affect forces; we only
isolate which parts of the ``classical source \(\Rightarrow\)
potential'' story fail, and what the correct replacement statement is at
low resolution.

\hypertarget{what-a-field-generates-a-potential-means-in-the-bosonic-source-story}{%
\section{2. What ``a field generates a potential'' means in the bosonic
source
story}\label{what-a-field-generates-a-potential-means-in-the-bosonic-source-story}}

The archetypal construction is a bosonic mediator \(\varphi\) linearly
coupled to a commuting source \(J(x)\): \[
S[\varphi;J]=\int d^Dx\left(\frac12\,\varphi\,K\,\varphi + J\,\varphi\right),
\qquad K=(\Box+m^2)\ \text{(example)}.
\] Integrating out \(\varphi\) (Gaussian elimination) yields an
effective action quadratic in the source, \[
S_{\mathrm{eff}}[J]=-\frac12\int d^Dx\,d^Dy\; J(x)\,K^{-1}(x,y)\,J(y),
\] so the static, nonrelativistic limit of \(K^{-1}\) produces a central
potential (Yukawa for \(m\neq 0\), Coulomb-type when \(m=0\)).

The key structural ingredient is that the source is an ordinary
commuting function (a classical background density).

\hypertarget{fermionic-fields-linear-sources-are-grassmann-so-the-classical-source-story-does-not-transplant}{%
\section{3. Fermionic fields: linear sources are Grassmann, so the
classical-source story does not
transplant}\label{fermionic-fields-linear-sources-are-grassmann-so-the-classical-source-story-does-not-transplant}}

For a Dirac fermion \(\Psi\), the generating functional with sources is
written with \textbf{Grassmann-valued} sources \(\eta,\bar\eta\): \[
Z[\bar\eta,\eta]
=\int D\bar\Psi\,D\Psi\;
\exp\!\left(
i\int d^Dx\;\bar\Psi\,(i\gamma^\mu\partial_\mu-m)\Psi
+i\int d^Dx\;(\bar\eta\,\Psi+\bar\Psi\,\eta)
\right).
\] An explicit statement of this form, including that \(\eta,\bar\eta\)
are Grassmann-valued, is recorded in
{[}Floerchinger2024QFT1Lecture21{]}.

Two immediate consequences follow.

\texttt{Remark\ 3.1\ (Obstruction\ statement).} The bosonic derivation
``choose a commuting classical source \(J\), integrate out the field,
and read off a classical potential'' does not directly apply to
fermions, because the linear source terms that couple to \(\Psi\)
require Grassmann sources rather than commuting c-number densities.
Therefore, ``fermion exchange generates a classical potential between
commuting sources'' is not a well-posed transplant of the bosonic story.

This does \textbf{not} mean fermions are irrelevant: fermions can and do
affect effective interactions through loop effects, through bosonic
composite modes (bilinears), and through low-energy EFT operators. The
point is that the meaning of ``generates a potential'' must be stated
through one of these controlled mechanisms.

\hypertarget{the-controlled-alternative-fermion-loops-modify-bosonic-propagators}{%
\subsection{3.1 The controlled alternative: fermion loops modify bosonic
propagators}\label{the-controlled-alternative-fermion-loops-modify-bosonic-propagators}}

The standard example is vacuum polarization in quantum electrodynamics.
A closed electron--positron loop inserted into the photon propagator
gives a momentum-dependent correction to the effective electromagnetic
coupling, \[
\alpha_{\mathrm{eff}}(q^2)
=
\frac{\alpha}{1-\Pi(q^2)},
\qquad
\Pi(q^2)
=
-\frac{\alpha}{3\pi}\ln\!\left(\frac{q^2}{\mu^2}\right)+\cdots,
\] where \(\Pi(q^2)\) is the vacuum polarization function (the photon
self-energy from a fermion one-loop diagram). At low momentum transfer
(\(|q|\ll m_e\)), the loop correction is analytic in \(q^2\) and
generates precisely the local operators \(C_0 + C_2 q^2 + \cdots\)
discussed in Section 4 below.

The structural point: fermions affect forces, but the path from
``fermion field'' to ``effective interaction'' runs through a quantum
loop (not through a tree-level Gaussian elimination of a classical
source), and the low-energy residue takes the form of local/contact
operators.

\texttt{Example\ 3.2\ (Uehling\ potential:\ the\ coordinate-space\ face\ of\ vacuum\ polarization).}
The momentum-dependent coupling above translates, via Fourier transform,
into a coordinate-space correction to the Coulomb potential --- the
Uehling potential
\(V_{\mathrm{Uehl}}(r)=-({Z_1 Z_2\alpha}/{r})\cdot({2\alpha}/{3\pi})\int_1^\infty du\;{(1+1/(2u^2))\sqrt{1-1/u^2}}\,{u^{-1}}\,e^{-2m_e r\, u}\).
At short distances (\(r\ll 1/m_e\)) the integral yields a logarithmic
correction \(\propto\ln(1/(m_e r))\), reflecting the running coupling
and matching the analytic \(C_0+C_2 q^2+\cdots\) expansion of Section 4;
at long distances (\(r\gg 1/m_e\)) it is exponentially suppressed
\(\propto e^{-2m_e r}\), confirming that the fermion decouples below its
mass threshold. The dominant observable consequence is the
vacuum-polarization contribution to the hydrogen Lamb shift
(\(\approx 27\) MHz of the total \(\approx 1058\) MHz \(2S\)--\(2P\)
splitting), which probes the modified potential at nuclear distances
where the \(S\)-wave wavefunction satisfies \(|\psi(0)|^2\neq 0\).

\texttt{Remark\ 3.3\ (Schwinger\ pair\ production\ and\ the\ Euler–Heisenberg\ Lagrangian:\ the\ non-perturbative\ complement).}
The perturbative vacuum-polarization story of Section 3.1 has a
non-perturbative counterpart. In a constant electric field
\(\mathcal{E}\), the QED vacuum is unstable against spontaneous
electron--positron pair creation at a rate per unit volume
\(\Gamma/V\propto m^2(\mathcal{E}/\mathcal{E}_{\mathrm{cr}})^2\exp(-\pi\mathcal{E}_{\mathrm{cr}}/\mathcal{E})\),
where
\(\mathcal{E}_{\mathrm{cr}}=m^2c^3/(e\hbar)\approx 1.3\times 10^{18}\)
V/m is the Schwinger critical field (Schwinger, 1951). The essential
singularity \(\exp(-\pi\mathcal{E}_{\mathrm{cr}}/\mathcal{E})\) makes
this invisible to all orders of perturbation theory: the rate vanishes
faster than any power of \(\alpha\) as \(\mathcal{E}\to 0\). The real
part of the same one-loop effective action gives the Euler--Heisenberg
Lagrangian
\(\mathcal{L}_{\mathrm{EH}}=\mathcal{L}_{\mathrm{Maxwell}}+(2\alpha^2/45m^4)[(\mathcal{E}^2-\mathcal{B}^2)^2+7(\mathcal{E}\cdot\mathcal{B})^2]+\cdots\)
(Euler and Heisenberg, 1936) --- dimension-8 contact operators in the
electromagnetic field, the QFT counterpart of the contact expansion in
Section 4. Fermions thus affect forces both perturbatively (running
coupling, Uehling potential) and non-perturbatively (pair creation,
photon--photon scattering), with the contact-operator tower organizing
both regimes.

\hypertarget{ir-replacement-local-operators-rightarrow-contact-kernels-boundary-condition-data}{%
\section{\texorpdfstring{4. IR replacement: local operators
\(\Rightarrow\) contact kernels / boundary-condition
data}{4. IR replacement: local operators \textbackslash Rightarrow contact kernels / boundary-condition data}}\label{ir-replacement-local-operators-rightarrow-contact-kernels-boundary-condition-data}}

At low resolution, integrating out heavy degrees of freedom typically
produces local operators. In a two-body, nonrelativistic sector, this
appears as an amplitude expansion analytic in momentum transfer \(q\):
\[
\mathcal A(q)=C_0+C_2 q^2+O(q^4).
\]

The coordinate-space interaction associated to such an analytic
expansion is distributional and diagonal-supported. The invariant core
is a Fourier-transform identity: \[
\int \frac{d^d q}{(2\pi)^d}\,e^{iq\cdot r}= \delta^{(d)}(r),
\qquad
\int \frac{d^d q}{(2\pi)^d}\,q^2\,e^{iq\cdot r}= -\nabla^2\delta^{(d)}(r).
\]

\texttt{Derivation\ 4.1\ (Contact\ expansion\ gives\ \textbackslash{}(\textbackslash{}delta\textbackslash{})\ and\ derivative\ contacts).}
Interpreting the low-energy interaction kernel as the inverse Fourier
transform of \(\mathcal A(q)\) (Born-level language, up to overall
convention-dependent factors), we obtain \[
V(r)\ \propto\ \int \frac{d^d q}{(2\pi)^d}\,e^{iq\cdot r}\,\mathcal A(q)
\ \propto\
C_0\,\delta^{(d)}(r)\;-\;C_2\,\nabla^2\delta^{(d)}(r)\;+\;\cdots.
\] Thus locality at low energy naturally becomes \textbf{contact data}:
delta kernels and their derivatives supported at a point (or on the
diagonal, in kernel language).

In singular cases (notably \(\delta\) interactions in \(d\ge 2\) in
various channels), this contact data is not simply a fixed number: it is
defined by a renormalization condition and can generate RG-invariant
scales and bound states {[}Jackiw1991DeltaPotentials{]}
{[}ManuelTarrach1994PertRenQM{]}.

\texttt{Remark\ 4.2\ (Connection\ to\ the\ effective\ range\ expansion).}
In scattering theory the s-wave amplitude is parametrized by the
effective range expansion (ERE)
\(k\cot\delta_0(k) = -1/a + r_0 k^2/2 + O(k^4)\), where \(a\) is the
scattering length and \(r_0\) the effective range. The contact expansion
of Derivation 4.1 is the momentum-space counterpart: at Born level,
\(C_0\) determines \(a\), \(C_2\) determines \(r_0\), and each higher
\(C_{2n}\) maps to a shape parameter. Examples 5.1 and 5.2 below are the
leading-order case \(C_2 = 0\) (zero effective range, \(r_0 = 0\)), for
which the full amplitude \(f_0(k) = -a/(1+ika)\) depends on a single
parameter --- the scattering length.

\texttt{Remark\ 4.3\ (Huang-Yang\ pseudopotential:\ regularization\ built\ into\ the\ operator).}
The Huang-Yang pseudopotential
\(V(r)=(4\pi\hbar^2 a/m)\,\delta^{(3)}(r)\,(\partial/\partial r)(r\,\cdot)\)
builds the renormalization condition of Example 5.2 directly into the
operator definition. The differential operator
\((\partial/\partial r)(r\,\cdot)\) extracts the regular part of the
wavefunction at the origin --- for a function with the s-wave
asymptotics \(\psi(r)\sim A(1/r-1/a)+\text{(regular)}\), it yields the
finite value \(-A/a\) rather than the divergent \(A/r\) --- and thereby
automatically implements the scattering-length boundary condition
without explicit cutoff manipulation. This is a ``smart'' contact
kernel: the prescription for handling the \(r=0\) singularity is not an
external regularization step but is part of the definition of the
interaction, with different values of \(a\) selecting different
self-adjoint extensions.

\hypertarget{boundary-condition-reading-point-interactions}{%
\section{5. Boundary-condition reading (point
interactions)}\label{boundary-condition-reading-point-interactions}}

Point-supported interactions can be encoded as self-adjoint extension /
boundary-condition data rather than as ordinary functions \(V(r)\). This
is the natural operator-theoretic counterpart of ``diagonal-support
kernels.'' For standard references and pedagogical framing, see
{[}BonneauFarautValent2001SAE{]} and the delta-potential discussion in
{[}Jackiw1991DeltaPotentials{]}.

This viewpoint matches the controlled-refinement perspective: when a
continuum description is defined as a refinement limit, UV data can
survive in the limit precisely as boundary-condition parameters (contact
terms), with RG flow expressing compatibility across resolutions.

\texttt{Example\ 5.1\ (Contact\ coupling\ generates\ a\ scale:\ 2D\ delta\ potential).}
In two spatial dimensions, a contact interaction
\(V(r)=g_0\,\delta^{(2)}(r)\) with bare coupling \(g_0\) and UV cutoff
\(\Lambda\) leads, after a standard loop integral, to a \(T\)-matrix
with the structure \[
T(k)^{-1}=\frac{1}{g_0}+\frac{m}{\pi\hbar^2}\ln\!\left(\frac{\Lambda}{k}\right),
\] which diverges as \(\Lambda\to\infty\) unless \(g_0\) is tuned.
Define a renormalized coupling at reference scale \(\mu\) by absorbing
the \(\ln\Lambda\) divergence; cutoff independence then gives the beta
function \(\beta(g_R)=\mu\,dg_R/d\mu=(m/\pi\hbar^2)\,g_R^2\). This is a
quadratic beta function of the same form as the toy logarithmic model in
the cornerstone (Section 8.3), with solution producing a dynamically
generated scale \(\mu_\ast=\mu\,e^{\pi\hbar^2/(mg_R)}\). For attractive
coupling (\(g_R<0\)) this scale is below the reference scale and sets
the bound-state energy: \(E=-\hbar^2\mu_\ast^2/(2m)\).

The structural lesson: the ``contact'' limit of the effective
interaction is not a number (coupling constant) but a flow --- a
scale-dependent parameter whose RG trajectory is part of the definition.
This is ``uncuttable'' in the sense of the companion note: the continuum
theory requires the refinement rule (cutoff removal + beta function) and
not merely a single-cutoff value.

\texttt{Example\ 5.2\ (3D\ contact\ interaction:\ scattering\ length).}
In three spatial dimensions, the same contact interaction
\(V(r)=g_0\,\delta^{(3)}(r)\) with UV cutoff \(\Lambda\) produces a
linearly divergent loop integral (compared to the logarithmic divergence
in \(d=2\)). After resummation, the s-wave scattering amplitude takes
the standard effective-range form with zero effective range: \[
f_0(k) = \frac{-a}{1+ika},
\] where the scattering length \(a\) is defined by absorbing the
\(\Lambda\)-dependent bare coupling into a single physical parameter via
a renormalization condition of the form
\(1/g_0 \propto \Lambda + \text{(finite part depending on } a\text{)}\).
When \(a>0\), a pole at \(k=i/a\) gives a bound state with energy
\(E=-\hbar^2/(2ma^2)\) {[}AlbeverioGesztesyHoeghKrohnHolden2005{]}. The
comparison with Example 5.1 highlights how the divergence character
changes with dimension --- logarithmic (\(d=2\)) versus linear (\(d=3\))
--- while the structural lesson is identical: the ``coupling constant''
of a contact interaction is not a bare number but a
renormalization-group datum, defined only through a refinement rule
(cutoff removal + physical matching condition).

\texttt{Remark\ 5.3\ (Unitarity\ limit:\ universality\ at\ the\ RG\ fixed\ point).}
When the scattering length diverges (\(|a|\to\infty\)), the contact
coupling sits at a non-trivial RG fixed point and the theory becomes
scale-invariant: no microscopic length survives beyond the interparticle
spacing. Thermodynamic ratios become universal --- for a
spin-\(\tfrac12\) Fermi gas the ground-state energy is
\(E=\xi\,E_{\mathrm{FG}}\) with Bertsch parameter \(\xi\approx 0.37\),
independent of the short-range physics that produced the large
scattering length. This fixed point controls the BEC--BCS crossover in
cold atomic gases, where a magnetic Feshbach resonance tunes \(a\)
through \(\pm\infty\), providing a laboratory realization of the
structural lesson in Examples 5.1--5.2: the contact coupling is not a
number but a flow, and the fixed point of that flow generates
universality.

\texttt{Remark\ 5.4\ (Dimensional\ dependence:\ from\ UV-finite\ to\ cutoff-dependent\ extension\ data).}
The same contact operator \(\delta^{(d)}(r)\) requires qualitatively
different control data across dimensions. In \(d=1\), the delta
potential defines a self-adjoint extension of the free Hamiltonian whose
boundary-condition parameter (the coupling \(g\)) is UV-finite --- no
cutoff dependence, no RG flow, just a number. In \(d=2\) (Example 5.1),
the extension parameter diverges logarithmically with the cutoff,
producing a running coupling and dimensional transmutation. In \(d=3\)
(Example 5.2), the divergence is linear. The critical dimension is
\(d=2\), where the contact coupling is classically marginal; above it,
the coupling is classically relevant and power-law subtractions are
needed. In all dimensions the interaction requires self-adjoint
extension theory (it is never a ``plain operator perturbation''), but
the extension datum transitions from UV-finite (\(d=1\)) to
cutoff-dependent (\(d\ge 2\)) --- and it is this transition that makes
renormalization group flow part of the definition.

\texttt{Remark\ 5.5\ (Efimov\ effect:\ 3-body\ limit\ cycle\ from\ 2-body\ contact).}
At the unitarity limit of Remark 5.3, the 3-body sector exhibits a
qualitatively new RG phenomenon: the running 3-body coupling is
log-periodic in the cutoff with period \(\pi/s_0\) (where
\(s_0\approx 1.006\) for identical bosons), corresponding to a geometric
energy scaling factor \(e^{\pi/s_0}\approx 22.7\). This RG limit cycle
--- as opposed to the 2-body fixed point --- produces an infinite tower
of 3-body bound states with geometrically spaced energies
\(E_n\propto e^{-2\pi n/s_0}\), a signature of discrete scale invariance
first predicted by Efimov (1970) and observed in cesium cold atoms by
Kraemer et al.~(2006). Crucially, the absolute position of the tower
requires an additional 3-body parameter beyond the scattering length
\(a\) --- a new piece of self-adjoint extension data not determined by
2-body physics, illustrating that the contact/extension hierarchy grows
richer with particle number.

\texttt{Remark\ 5.6\ (Feshbach\ resonances:\ laboratory\ realization\ of\ the\ contact-coupling\ RG\ flow).}
The theoretical framework of Examples 5.1--5.2 has a direct experimental
counterpart: a magnetic Feshbach resonance occurs when a bound state in
a closed hyperfine channel is Zeeman-tuned into near-degeneracy with the
open scattering threshold, producing a scattering length
\(a(B)=a_{\mathrm{bg}}(1-\Delta B/(B-B_0))\) that diverges as
\(B\to B_0\) (Chin et al., 2010). Sweeping \(B\) through the resonance
traces the full one-parameter family of self-adjoint extensions: from
weakly attractive (\(a<0\), BCS pairing) through the scale-invariant
unitarity fixed point of Remark 5.3 (\(|a|=\infty\)) to the BEC regime
of tightly bound dimers (\(a>0\)), thereby realizing the BEC--BCS
crossover as a continuous traversal of the RG flow. The Efimov states of
Remark 5.5 have been observed near such Feshbach resonances in cesium
(Kraemer et al., 2006). Structurally, the magnetic field \(B\) is an
external knob that tunes the renormalized contact coupling at a fixed
energy scale --- the experimentalist's version of the beta function
controlling the coupling flow in Examples 5.1--5.2.

\texttt{Remark\ 5.7\ (Casimir\ effect:\ macroscopic\ force\ from\ boundary\ conditions\ alone).}
The self-adjoint-extension/boundary-condition paradigm of Examples
5.1--5.2 generates a measurable macroscopic force in the Casimir effect
(Casimir, 1948). Two perfectly conducting parallel plates separated by
distance \(d\) impose boundary conditions that restrict the
electromagnetic vacuum modes to a discrete set between the plates. The
regulated difference of zero-point energies --- computed via
zeta-function regularization, exponential cutoff, or dimensional
methods, all yielding the same result --- gives an attractive force per
unit area \(F/A=-\pi^2\hbar c/(240\,d^4)\), experimentally verified to
percent-level precision (Lamoreaux, 1997; Mohideen and Roy, 1998). No
classical source \(J(x)\) is needed: the force arises purely from how
the boundary conditions modify the vacuum fluctuation spectrum. The
regularization-scheme independence of the result echoes the cutoff
independence of the renormalized observables in Examples 5.1--5.2: the
physically meaningful Casimir energy is the refinement limit (regulator
removed), not any single regulated approximant. This is the
boundary-condition paradigm of Section 5 elevated from a mathematical
framework to a laboratory-observable phenomenon.

\hypertarget{outlook-emergent-bosonic-composites}{%
\section{6. Outlook: emergent bosonic
composites}\label{outlook-emergent-bosonic-composites}}

\texttt{Remark\ 6.1\ (Cooper\ pairing:\ fermion\ bilinears\ as\ emergent\ bosonic\ mediators).}
The Grassmann-source obstruction of Section 3 is bypassed when fermion
bilinears condense into bosonic composite modes. The canonical example
is BCS superconductivity (Bardeen, Cooper, and Schrieffer, 1957): any
attractive interaction in the Cooper channel between fermions near a
Fermi surface produces a pairing instability with a non-perturbative gap
\(\Delta\propto\omega_D\exp(-1/(N(0)|V|))\), where \(N(0)\) is the
density of states and \(|V|\) the interaction strength. The resulting
condensate --- described by the bosonic order parameter
\(\langle\psi_\downarrow\psi_\uparrow\rangle\) --- gives mass to the
photon via the Anderson--Higgs mechanism, with the effective photon mass
\(m_A\propto\sqrt{n_s\,e^2}\) set by the superfluid density \(n_s\)
(proportional to \(|\Delta|^2\) at zero temperature). This generates the
Meissner effect: magnetic fields decay exponentially inside the
superconductor with penetration depth \(\lambda=1/m_A\). In the
contact-interaction limit, the BCS gap equation requires the same
renormalization as Examples 5.1--5.2: the bare coupling diverges with
the cutoff, and the physical gap is determined by the renormalized
scattering length. The BEC--BCS crossover of Remark 5.6 then
interpolates continuously between overlapping Cooper pairs (BCS) and
tightly bound bosonic molecules (BEC), demonstrating that the ``emergent
boson'' spans the full range from collective many-body mode to genuine
two-body bound state.

\hypertarget{references}{%
\section{References}\label{references}}

\begin{enumerate}
\def\labelenumi{\arabic{enumi}.}
\tightlist
\item
  {[}ManuelTarrach1994PertRenQM{]} Cristina Manuel and Rolf Tarrach,
  ``Perturbative Renormalization in Quantum Mechanics,'' \emph{Physics
  Letters B} 328 (1994), 113--118. arXiv:\texttt{hep-th/9309013} (v1, 2
  Sep 1993). DOI \texttt{10.1016/0370-2693(94)90437-5}.
\item
  {[}BonneauFarautValent2001SAE{]} Guy Bonneau, Jacques Faraut, and
  Galliano Valent, ``Self-adjoint extensions of operators and the
  teaching of quantum mechanics,'' \emph{American Journal of Physics} 69
  (2001), 322--331. arXiv:\texttt{quant-ph/0103153}. DOI
  \texttt{10.1119/1.1328351}.
\item
  {[}Jackiw1991DeltaPotentials{]} R. Jackiw, ``Delta-function potentials
  in two- and three-dimensional quantum mechanics,'' MIT-CTP-1937 (Jan
  1991). Reprinted in \emph{M.A.B. Bég Memorial Volume} (World
  Scientific, 1991), pp.~25--42. OA mirror:
  \url{https://www.physics.smu.edu/scalise/P6335fa21/notes/Jackiw.pdf}.
\item
  {[}Floerchinger2024QFT1Lecture21{]} Stefan Floerchinger, ``Quantum
  field theory 1, lecture 21'' (updated 11 Jun 2024). (States the
  Dirac-fermion generating functional \(Z[\bar\eta,\eta]\) with
  Grassmann-valued sources.) OA: lecture webpage.
\item
  {[}AlbeverioGesztesyHoeghKrohnHolden2005{]} S. Albeverio, F. Gesztesy,
  R. Høegh-Krohn, and H. Holden, \emph{Solvable Models in Quantum
  Mechanics}, 2nd ed., AMS Chelsea Publishing, 2005. ISBN
  \texttt{978-0-8218-3624-4}. (Canonical reference for point
  interactions in quantum mechanics; self-adjoint extensions, delta
  potentials.)
\item
  {[}Casimir1948{]} H. B. G. Casimir, ``On the attraction between two
  perfectly conducting plates,'' \emph{Proc. K. Ned. Akad. Wet.} 51
  (1948), 793--795. (Original prediction of the Casimir effect; used in
  Remark 5.7.)
\item
  {[}Lamoreaux1997{]} S. K. Lamoreaux, ``Demonstration of the Casimir
  force in the 0.6 to 6 μm range,'' \emph{Physical Review Letters} 78
  (1997), 5--8. DOI \texttt{10.1103/PhysRevLett.78.5}. (First precision
  Casimir measurement; used in Remark 5.7.)
\item
  {[}Chin2010{]} Cheng Chin, Rudolf Grimm, Paul Julienne, and Eite
  Tiesinga, ``Feshbach resonances in ultracold gases,'' \emph{Reviews of
  Modern Physics} 82 (2010), 1225--1286. DOI
  \texttt{10.1103/RevModPhys.82.1225}. (Comprehensive review of Feshbach
  resonances; used in Remark 5.6.)
\item
  {[}BCS1957{]} John Bardeen, Leon N. Cooper, and J. Robert Schrieffer,
  ``Theory of superconductivity,'' \emph{Physical Review} 108 (1957),
  1175--1204. DOI \texttt{10.1103/PhysRev.108.1175}. (BCS theory; used
  in Remark 6.1.)
\end{enumerate}

\end{document}
