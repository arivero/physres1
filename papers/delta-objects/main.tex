% Options for packages loaded elsewhere
\PassOptionsToPackage{unicode}{hyperref}
\PassOptionsToPackage{hyphens}{url}
\documentclass[
]{article}
\usepackage{xcolor}
\usepackage{amsmath,amssymb}
\setcounter{secnumdepth}{-\maxdimen} % remove section numbering
\usepackage{iftex}
\ifPDFTeX
  \usepackage[T1]{fontenc}
  \usepackage[utf8]{inputenc}
  \usepackage{textcomp} % provide euro and other symbols
\else % if luatex or xetex
  \usepackage{unicode-math} % this also loads fontspec
  \defaultfontfeatures{Scale=MatchLowercase}
  \defaultfontfeatures[\rmfamily]{Ligatures=TeX,Scale=1}
\fi
\usepackage{lmodern}
\ifPDFTeX\else
  % xetex/luatex font selection
\fi
% Use upquote if available, for straight quotes in verbatim environments
\IfFileExists{upquote.sty}{\usepackage{upquote}}{}
\IfFileExists{microtype.sty}{% use microtype if available
  \usepackage[]{microtype}
  \UseMicrotypeSet[protrusion]{basicmath} % disable protrusion for tt fonts
}{}
\makeatletter
\@ifundefined{KOMAClassName}{% if non-KOMA class
  \IfFileExists{parskip.sty}{%
    \usepackage{parskip}
  }{% else
    \setlength{\parindent}{0pt}
    \setlength{\parskip}{6pt plus 2pt minus 1pt}}
}{% if KOMA class
  \KOMAoptions{parskip=half}}
\makeatother
\setlength{\emergencystretch}{3em} % prevent overfull lines
\providecommand{\tightlist}{%
  \setlength{\itemsep}{0pt}\setlength{\parskip}{0pt}}
\usepackage{bookmark}
\IfFileExists{xurl.sty}{\usepackage{xurl}}{} % add URL line breaks if available
\urlstyle{same}
\hypersetup{
  pdftitle={Delta Objects as Half-Density Kernels: Identity{,} Stationary-Set Concentration{,} and Point Interactions},
  hidelinks,
  pdfcreator={LaTeX via pandoc}}

\title{Delta Objects as Half-Density Kernels: Identity, Stationary-Set
Concentration, and Point Interactions}
\author{}
\date{}

\begin{document}
\maketitle
\begin{abstract}
Three seemingly different uses of the Dirac delta share one geometric
meaning when amplitudes are treated as \textbf{half-densities}: 1. the
delta as the Schwartz kernel of the identity operator, 2. the delta as a
density supported on stationary points (\(\delta(\nabla f)\)), 3. the
delta as a rank-one kernel defining a point interaction
(\(g|0\rangle\langle0|\)).

In each case, the amplitude-level object carries \textbf{square-root
Jacobian} weights (half-density weights), while the corresponding
``probability''/density-level object carries the unsquared Jacobians.
This note collects the finite-dimensional identities and scaling
computations that make this pattern explicit, and isolates where a
physical length scale may enter when one insists on scalar
representatives.
\end{abstract}

This note is a companion to \texttt{paper/main.md}. Statements are kept
finite-dimensional unless explicitly labeled heuristic.

\section{1. Half-densities and kernels (coordinate
free)}\label{half-densities-and-kernels-coordinate-free}

Let \(M\) be a \(d\)-dimensional manifold and \(|\Omega|^{1/2}\) the
half-density bundle. An operator
\(K:\Gamma_c(|\Omega|^{1/2})\to \Gamma(|\Omega|^{1/2})\) has a natural
Schwartz kernel \[
\mathsf K\in \mathcal D'(M\times M;\;|\Omega|^{1/2}\boxtimes|\Omega|^{1/2}),
\] so that \[
(K\psi)(x)=\int_M \mathsf K(x,y)\,\psi(y),
\] is coordinate invariant: \(\mathsf K(x,y)\psi(y)\) is a density in
\(y\) valued in a half-density at \(x\).

Scalarizing kernels (writing \(\int dy\) with a scalar integrand)
implicitly chooses a reference density/half-density; the half-density
formalism keeps this choice explicit.

\section{2. Delta as the identity kernel (and near-diagonal
scaling)}\label{delta-as-the-identity-kernel-and-near-diagonal-scaling}

The identity operator on half-densities has Schwartz kernel \[
\mathsf K_{\mathrm{Id}}(x,y)=\delta^{(d)}(x-y)\,|dx|^{1/2}|dy|^{1/2}.
\]

\subsection{\texorpdfstring{Worked scaling computation (the \(d/2\)
exponent)}{Worked scaling computation (the d/2 exponent)}}\label{worked-scaling-computation-the-d2-exponent}

Introduce near-diagonal coordinates \(y=x+\varepsilon v\). Then
\(\delta^{(d)}(x-y)=\delta^{(d)}(\varepsilon v)=\varepsilon^{-d}\delta^{(d)}(v)\)
and \(|dy|^{1/2}=\varepsilon^{d/2}|dv|^{1/2}\), so \[
\mathsf K_{\mathrm{Id}}(x,x+\varepsilon v)
=\varepsilon^{-d/2}\,\delta^{(d)}(v)\,|dx|^{1/2}|dv|^{1/2}.
\] Thus the universal \(\varepsilon^{-d/2}\) normalization exponent is
already present in the identity delta kernel, once kernels are treated
as half-densities.

\section{\texorpdfstring{3. Delta on the stationary set: \u03b4(\u2207f)
and determinant
weights}{3. Delta on the stationary set: 3b4(207f) and determinant weights}}\label{delta-on-the-stationary-set-3b4207f-and-determinant-weights}

\subsection{\texorpdfstring{3.1 One-dimensional identity
(\u03b4(f'))}{3.1 One-dimensional identity (3b4(f'))}}\label{one-dimensional-identity-3b4f}

Let \(f:\mathbb R\to\mathbb R\) have finitely many nondegenerate
critical points \(x_i\) (so \(f'(x_i)=0\), \(f''(x_i)\neq 0\)). Then, as
distributions, \[
\delta(f'(x))=\sum_i \frac{\delta(x-x_i)}{|f''(x_i)|}.
\] So \(\delta(f')\,dx\) is a density supported at stationary points
with weights \(1/|f''|\).

\subsection{\texorpdfstring{3.1a \u03b4(f') versus \u03b4': delta of a
derivative vs derivative of
delta}{3.1a 3b4(f') versus 3b4': delta of a derivative vs derivative of delta}}\label{a-3b4f-versus-3b4-delta-of-a-derivative-vs-derivative-of-delta}

The notation \(\delta(f')\) above means: apply the Dirac delta
distribution \(\delta(\cdot)\) to the \textbf{function} \(f'(x)\),
thereby localizing to the stationary set \(f'(x)=0\). It should not be
confused with \(\delta'\), the \textbf{distributional derivative} of
\(\delta\), defined by duality: \[
\langle \delta',\varphi\rangle := -\langle \delta,\varphi'\rangle = -\varphi'(0).
\] So \(\delta'\) is the distribution that probes derivatives of test
functions at a point (``value of the derivative at zero'', up to sign),
whereas \(\delta(f')\) is a stationary-set localization distribution.

\subsection{\texorpdfstring{3.1b \u03b4' from point splitting
(difference quotient of shifted
deltas)}{3.1b 3b4' from point splitting (difference quotient of shifted deltas)}}\label{b-3b4-from-point-splitting-difference-quotient-of-shifted-deltas}

The distribution \(\delta'\) can be realized as a regulated
point-splitting limit. Let \(\varepsilon\to 0\) and consider the shifted
delta \(\delta(x+\varepsilon)\). For any test function \(\varphi\), \[
\left\langle \frac{\delta(\,\cdot+\varepsilon)-\delta}{\varepsilon},\varphi\right\rangle
=\frac{\varphi(-\varepsilon)-\varphi(0)}{\varepsilon}
\xrightarrow[\varepsilon\to 0]{} -\varphi'(0)
=\langle \delta',\varphi\rangle.
\] Hence, in the sense of distributions, \[
\frac{\delta(x+\varepsilon)-\delta(x)}{\varepsilon}\xrightarrow[\varepsilon\to 0]{}\delta'(x).
\]

This gives a clean dictionary item for ``probing the derivative at a
point'': \[
f'(0)=\langle -\delta', f\rangle.
\] For the parallel smooth-function toy model (``difference quotient as
divergence + subtraction'') and further remarks, see
\texttt{blackboards/2026-02-10-difference-quotients-counterterms-and-delta-prime.md}.

\subsection{\texorpdfstring{3.2 Multi-dimensional identity
(\u03b4(\u2207f))}{3.2 Multi-dimensional identity (3b4(207f))}}\label{multi-dimensional-identity-3b4207f}

Let \(f:\mathbb R^n\to\mathbb R\) have finitely many nondegenerate
critical points \(x_i\) (so \(\nabla f(x_i)=0\) and
\(\det(\mathrm{Hess}\,f)(x_i)\neq 0\)). Then \[
\delta^{(n)}(\nabla f(x))
=\sum_i \frac{\delta^{(n)}(x-x_i)}{|\det(\mathrm{Hess}\,f)(x_i)|}.
\]

\subsection{3.3 Stationary phase and square-root weights (amplitudes vs
densities)}\label{stationary-phase-and-square-root-weights-amplitudes-vs-densities}

For the oscillatory integral \[
I(\hbar)=\int_{\mathbb R^n} e^{\frac{i}{\hbar}f(x)}\,a(x)\,dx,\qquad \hbar\to 0^+,
\] stationary phase gives amplitude contributions weighted by \[
\frac{1}{\sqrt{|\det(\mathrm{Hess}\,f)(x_i)|}},
\] up to a universal \(\hbar\)-dependent factor and a signature phase.
Squaring amplitude weights produces the density weights in
\(\delta^{(n)}(\nabla f)\). This is the finite-dimensional prototype of
the slogan: \textbf{amplitudes are half-densities; probabilities are
densities.}

\subsection{\texorpdfstring{3.4 Extremals in weak form: where \u03b4 and
\u03b4' appear in
Euler\u2013Lagrange}{3.4 Extremals in weak form: where 3b4 and 3b4' appear in Euler013Lagrange}}\label{extremals-in-weak-form-where-3b4-and-3b4-appear-in-euler013lagrange}

For an action \(S[q]=\int L(q,\dot q,t)\,dt\), the extremal condition is
naturally distributional: for test variations \(\eta(t)\) of compact
support, \[
\delta S[q;\eta]=\int \Bigl(\frac{\partial L}{\partial q}-\frac{d}{dt}\frac{\partial L}{\partial \dot q}\Bigr)\eta(t)\,dt.
\] If \(\delta S[q;\eta]=0\) for all \(\eta\), then the
Euler\u2013Lagrange expression vanishes as a distribution. Approximating
\(\eta\) by bump functions converging to \(\delta(t-t_\ast)\) localizes
the equation at \(t_\ast\) under regularity.

When \(\partial L/\partial \dot q\) has jumps (corners/impulses), the
distributional derivative produces delta terms automatically; more
generally, point-supported singularities are encoded by delta kernels
and their derivatives (\(\delta,\delta',\ldots\)), depending on
distributional order. For a short dictionary, see
\texttt{blackboards/2026-02-10-distribution-theory-for-extremals.md}.

\section{4. Delta at a point: point interactions as rank-one
kernels}\label{delta-at-a-point-point-interactions-as-rank-one-kernels}

A point interaction is naturally the rank-one operator \[
V=g\,|0\rangle\langle0|.
\] In the half-density kernel calculus this is written as the
bi-half-density distribution supported at \((0,0)\): \[
\mathsf K_V(x,y)=g\;\delta^{(d)}(x)\,\delta^{(d)}(y)\,|dx|^{1/2}|dy|^{1/2}.
\] This is the ``projector-like delta'' object underlying contact
interactions.

\section{5. Where scales enter upon scalarization (and why RG invariants
are natural
candidates)}\label{where-scales-enter-upon-scalarization-and-why-rg-invariants-are-natural-candidates}

Half-density kernels are canonical; scalar representatives are not.
Choosing a reference half-density \(\sigma_\ast\) identifies any
half-density \(\psi\) with a scalar \(f\) via \(\psi=f\,\sigma_\ast\).
If one insists that scalar representatives be dimensionless, then
\(\sigma_\ast\) must carry a \(\text{length}^{d/2}\) constant.

In marginal cases (notably the 2D point interaction), renormalization
generates an RG-invariant scale \(\kappa_\ast\) (dimensional
transmutation). This suggests a conditional identification: if one adds
a universality hypothesis that scalarization scales must be built from
physical invariants, then RG-invariant scales are natural candidates to
supply the missing \(\text{length}^{d/2}\) factors required by
scalarization.

This note treats that identification as an organizing perspective, not
as a theorem.

\section{6. Outlook}\label{outlook}

\begin{enumerate}
\def\labelenumi{\arabic{enumi}.}
\tightlist
\item
  Relate the determinant weights in \(\delta(\nabla f)\) to the mixed
  Hessian determinants (Van Vleck type) that appear after eliminating
  intermediate variables in time slicing (Schur complement template).
\item
  Clarify which parts of the ``functional delta \(\delta(\delta S)\)''
  story survive rigorous regularization and which remain heuristic.
\end{enumerate}

\end{document}
