% Options for packages loaded elsewhere
\PassOptionsToPackage{unicode}{hyperref}
\PassOptionsToPackage{hyphens}{url}
%
\documentclass[
]{article}
\usepackage{amsmath,amssymb}
\usepackage{lmodern}
\usepackage{iftex}
\ifPDFTeX
  \usepackage[T1]{fontenc}
  \usepackage[utf8]{inputenc}
  \usepackage{textcomp} % provide euro and other symbols
\else % if luatex or xetex
  \usepackage{unicode-math}
  \defaultfontfeatures{Scale=MatchLowercase}
  \defaultfontfeatures[\rmfamily]{Ligatures=TeX,Scale=1}
\fi
% Use upquote if available, for straight quotes in verbatim environments
\IfFileExists{upquote.sty}{\usepackage{upquote}}{}
\IfFileExists{microtype.sty}{% use microtype if available
  \usepackage[]{microtype}
  \UseMicrotypeSet[protrusion]{basicmath} % disable protrusion for tt fonts
}{}
\makeatletter
\@ifundefined{KOMAClassName}{% if non-KOMA class
  \IfFileExists{parskip.sty}{%
    \usepackage{parskip}
  }{% else
    \setlength{\parindent}{0pt}
    \setlength{\parskip}{6pt plus 2pt minus 1pt}}
}{% if KOMA class
  \KOMAoptions{parskip=half}}
\makeatother
\usepackage{xcolor}
\IfFileExists{xurl.sty}{\usepackage{xurl}}{} % add URL line breaks if available
\IfFileExists{bookmark.sty}{\usepackage{bookmark}}{\usepackage{hyperref}}
\hypersetup{
  pdftitle={Delta Objects as Half-Density Kernels: Identity, Stationary-Set Concentration, and Point Interactions},
  pdfauthor={Alejandro Rivero},
  hidelinks,
  pdfcreator={LaTeX via pandoc}}
\urlstyle{same} % disable monospaced font for URLs
\usepackage[normalem]{ulem}
% Avoid problems with \sout in headers with hyperref
\pdfstringdefDisableCommands{\renewcommand{\sout}{}}
\setlength{\emergencystretch}{3em} % prevent overfull lines
\providecommand{\tightlist}{%
  \setlength{\itemsep}{0pt}\setlength{\parskip}{0pt}}
\setcounter{secnumdepth}{-\maxdimen} % remove section numbering
\ifLuaTeX
  \usepackage{selnolig}  % disable illegal ligatures
\fi

\title{Delta Objects as Half-Density Kernels: Identity, Stationary-Set
Concentration, and Point Interactions}
\author{Alejandro Rivero}
\date{2026}

\begin{document}
\maketitle
\begin{abstract}
Three seemingly different uses of the Dirac delta share one geometric
meaning when amplitudes are treated as \textbf{half-densities}: 1. the
delta as the Schwartz kernel of the identity operator, 2. the delta as a
density supported on stationary points (\(\delta(\nabla f)\)), 3. the
delta as a rank-one kernel defining a point interaction
(\(g|0\rangle\langle0|\)).

In each case, the amplitude-level object carries \textbf{square-root
Jacobian} weights (half-density weights), while the corresponding
``probability''/density-level object carries the unsquared Jacobians.
This note collects the finite-dimensional identities and scaling
computations that make this pattern explicit, and isolates where a
physical length scale may enter when one insists on scalar
representatives.
\end{abstract}

This note is a companion to the cornerstone manuscript. Statements are
kept finite-dimensional unless explicitly labeled heuristic.

\hypertarget{half-densities-and-kernels-coordinate-free}{%
\section{1. Half-densities and kernels (coordinate
free)}\label{half-densities-and-kernels-coordinate-free}}

Let \(M\) be a \(d\)-dimensional manifold and \(|\Omega|^{1/2}\) the
half-density bundle {[}BatesWeinstein1997{]}. An operator
\(K:\Gamma_c(|\Omega|^{1/2})\to \Gamma(|\Omega|^{1/2})\) has a natural
Schwartz kernel {[}Hormander2003{]} \[
\mathsf K\in \mathcal D'(M\times M;\;|\Omega|^{1/2}\boxtimes|\Omega|^{1/2}),
\] so that \[
(K\psi)(x)=\int_M \mathsf K(x,y)\,\psi(y),
\] is coordinate invariant: \(\mathsf K(x,y)\psi(y)\) is a density in
\(y\) valued in a half-density at \(x\).

Scalarizing kernels (writing \(\int dy\) with a scalar integrand)
implicitly chooses a reference density/half-density; the half-density
formalism keeps this choice explicit.

\hypertarget{delta-as-the-identity-kernel-and-near-diagonal-scaling}{%
\section{2. Delta as the identity kernel (and near-diagonal
scaling)}\label{delta-as-the-identity-kernel-and-near-diagonal-scaling}}

The identity operator on half-densities has Schwartz kernel \[
\mathsf K_{\mathrm{Id}}(x,y)=\delta^{(d)}(x-y)\,|dx|^{1/2}|dy|^{1/2}.
\]

\hypertarget{worked-scaling-computation-the-d2-exponent}{%
\subsection{\texorpdfstring{Worked scaling computation (the \(d/2\)
exponent)}{Worked scaling computation (the d/2 exponent)}}\label{worked-scaling-computation-the-d2-exponent}}

Introduce near-diagonal coordinates \(y=x+\varepsilon v\). Then
\(\delta^{(d)}(x-y)=\delta^{(d)}(\varepsilon v)=\varepsilon^{-d}\delta^{(d)}(v)\)
and \(|dy|^{1/2}=\varepsilon^{d/2}|dv|^{1/2}\), so \[
\mathsf K_{\mathrm{Id}}(x,x+\varepsilon v)
=\varepsilon^{-d/2}\,\delta^{(d)}(v)\,|dx|^{1/2}|dv|^{1/2}.
\] Thus the universal \(\varepsilon^{-d/2}\) normalization exponent is
already present in the identity delta kernel, once kernels are treated
as half-densities.

\hypertarget{delta-on-the-stationary-set-ux3b4f-and-determinant-weights}{%
\section{3. Delta on the stationary set: δ(∇f) and determinant
weights}\label{delta-on-the-stationary-set-ux3b4f-and-determinant-weights}}

\hypertarget{one-dimensional-identity-ux3b4f}{%
\subsection{3.1 One-dimensional identity
(δ(f'))}\label{one-dimensional-identity-ux3b4f}}

Let \(f:\mathbb R\to\mathbb R\) have finitely many nondegenerate
critical points \(x_i\) (so \(f'(x_i)=0\), \(f''(x_i)\neq 0\)). Then, as
distributions, \[
\delta(f'(x))=\sum_i \frac{\delta(x-x_i)}{|f''(x_i)|}.
\] So \(\delta(f')\,dx\) is a density supported at stationary points
with weights \(1/|f''|\).

\hypertarget{a-ux3b4f-versus-ux3b4-delta-of-a-derivative-vs-derivative-of-delta}{%
\subsection{3.1a δ(f') versus δ': delta of a derivative vs derivative of
delta}\label{a-ux3b4f-versus-ux3b4-delta-of-a-derivative-vs-derivative-of-delta}}

The notation \(\delta(f')\) above means: apply the Dirac delta
distribution \(\delta(\cdot)\) to the \textbf{function} \(f'(x)\),
thereby localizing to the stationary set \(f'(x)=0\). It should not be
confused with \(\delta'\), the \textbf{distributional derivative} of
\(\delta\), defined by duality: \[
\langle \delta',\varphi\rangle := -\langle \delta,\varphi'\rangle = -\varphi'(0).
\] So \(\delta'\) is the distribution that probes derivatives of test
functions at a point (``value of the derivative at zero'', up to sign),
whereas \(\delta(f')\) is a stationary-set localization distribution.

\hypertarget{b-ux3b4-from-point-splitting-difference-quotient-of-shifted-deltas}{%
\subsection{3.1b δ' from point splitting (difference quotient of shifted
deltas)}\label{b-ux3b4-from-point-splitting-difference-quotient-of-shifted-deltas}}

The distribution \(\delta'\) can be realized as a regulated
point-splitting limit. Let \(\varepsilon\to 0\) and consider the shifted
delta \(\delta(x+\varepsilon)\). For any test function \(\varphi\), \[
\left\langle \frac{\delta(\,\cdot+\varepsilon)-\delta}{\varepsilon},\varphi\right\rangle
=\frac{\varphi(-\varepsilon)-\varphi(0)}{\varepsilon}
\xrightarrow[\varepsilon\to 0]{} -\varphi'(0)
=\langle \delta',\varphi\rangle.
\] Hence, in the sense of distributions, \[
\frac{\delta(x+\varepsilon)-\delta(x)}{\varepsilon}\xrightarrow[\varepsilon\to 0]{}\delta'(x).
\]

This gives a clean dictionary item for ``probing the derivative at a
point'': \[
f'(0)=\langle -\delta', f\rangle.
\] For the parallel smooth-function toy model (``difference quotient as
divergence + subtraction'') and further remarks, see the companion
notes.

\hypertarget{multi-dimensional-identity-ux3b4f}{%
\subsection{3.2 Multi-dimensional identity
(δ(∇f))}\label{multi-dimensional-identity-ux3b4f}}

Let \(f:\mathbb R^n\to\mathbb R\) have finitely many nondegenerate
critical points \(x_i\) (so \(\nabla f(x_i)=0\) and
\(\det(\mathrm{Hess}\,f)(x_i)\neq 0\)). Then \[
\delta^{(n)}(\nabla f(x))
=\sum_i \frac{\delta^{(n)}(x-x_i)}{|\det(\mathrm{Hess}\,f)(x_i)|}.
\]

\hypertarget{stationary-phase-and-square-root-weights-amplitudes-vs-densities}{%
\subsection{3.3 Stationary phase and square-root weights (amplitudes vs
densities)}\label{stationary-phase-and-square-root-weights-amplitudes-vs-densities}}

For the oscillatory integral \[
I(\hbar)=\int_{\mathbb R^n} e^{\frac{i}{\hbar}f(x)}\,a(x)\,dx,\qquad \hbar\to 0^+,
\] stationary phase gives amplitude contributions weighted by \[
\frac{1}{\sqrt{|\det(\mathrm{Hess}\,f)(x_i)|}},
\] up to a universal \(\hbar\)-dependent factor and a signature phase.
Squaring amplitude weights produces the density weights in
\(\delta^{(n)}(\nabla f)\). This is the finite-dimensional prototype of
the slogan: \textbf{amplitudes are half-densities; probabilities are
densities.}

\hypertarget{extremals-in-weak-form-where-ux3b4-and-ux3b4-appear-in-eulerlagrange}{%
\subsection{3.4 Extremals in weak form: where δ and δ' appear in
Euler--Lagrange}\label{extremals-in-weak-form-where-ux3b4-and-ux3b4-appear-in-eulerlagrange}}

For an action \(S[q]=\int L(q,\dot q,t)\,dt\), the extremal condition is
naturally distributional: for test variations \(\eta(t)\) of compact
support, \[
\delta S[q;\eta]=\int \Bigl(\frac{\partial L}{\partial q}-\frac{d}{dt}\frac{\partial L}{\partial \dot q}\Bigr)\eta(t)\,dt.
\] If \(\delta S[q;\eta]=0\) for all \(\eta\), then the Euler--Lagrange
expression vanishes as a distribution. Approximating \(\eta\) by bump
functions converging to \(\delta(t-t_\ast)\) localizes the equation at
\(t_\ast\) under regularity.

When \(\partial L/\partial \dot q\) has jumps (corners/impulses), the
distributional derivative produces delta terms automatically; more
generally, point-supported singularities are encoded by delta kernels
and their derivatives (\(\delta,\delta',\ldots\)), depending on
distributional order.

\hypertarget{van-vleck-determinant-the-propagator-instance-of-the-square-root-hessian}{%
\subsection{3.5 Van Vleck determinant: the propagator instance of the
square-root
Hessian}\label{van-vleck-determinant-the-propagator-instance-of-the-square-root-hessian}}

The square-root Hessian weight of Section 3.3 has a distinguished
physical instance: the Van Vleck determinant
{[}VanVleck1928Correspondence{]} {[}Morette1951{]} in the semiclassical
propagator.

For the short-time quantum propagator between positions \(q_i\) and
\(q_f\) with time interval \(\Delta t\), stationary-phase evaluation of
the path integral gives \[
K(q_f,q_i;\Delta t)
\;\propto\;
\sqrt{D(q_f,q_i;\Delta t)}\;e^{(i/\hbar)\,S_{\mathrm{cl}}(q_f,q_i;\Delta t)},
\] where \(S_{\mathrm{cl}}\) is the classical action on the extremal
path and \[
D(q_f,q_i;\Delta t)
\;:=\;
\left|\det\!\left(-\frac{\partial^2 S_{\mathrm{cl}}}{\partial q_f^a\,\partial q_i^b}\right)\right|
\] is the Van Vleck determinant --- a \emph{mixed} Hessian (derivatives
at the two endpoints of the classical path), as opposed to the full
Hessian of \(f\) that appears in \(\delta(\nabla f)\). Despite this
difference, it arises by the same stationary-phase mechanism:
square-root Hessian weights at the amplitude level, confirming the
``amplitudes are half-densities'' pattern.

\texttt{Example\ 3.5a\ (Free\ particle).} For the free particle in \(d\)
dimensions, \(S_{\mathrm{cl}}=m|q_f-q_i|^2/(2\Delta t)\), so \[
D = (m/\Delta t)^d,\qquad \sqrt{D}=(m/\Delta t)^{d/2},
\] reproducing the \((\Delta t)^{-d/2}\) normalization of Section 2.

\texttt{Example\ 3.5b\ (Harmonic\ oscillator).} For the harmonic
oscillator (\(V=\tfrac12 m\omega^2 q^2\)) in \(d=1\), the classical
action between \(q_i\) and \(q_f\) in time \(\Delta t\) is
\(S_{\mathrm{cl}}=\frac{m\omega}{2\sin\omega\Delta t}\bigl[(q_f^2+q_i^2)\cos\omega\Delta t - 2q_f q_i\bigr]\),
giving \[
D = \left|\frac{m\omega}{\sin\omega\Delta t}\right|,\qquad \sqrt{D}=\sqrt{\frac{m\omega}{|\sin\omega\Delta t|}}.
\] As \(\omega\Delta t\to 0\),
\(\sin\omega\Delta t\approx\omega\Delta t\), recovering the
free-particle result \(\sqrt{D}\to\sqrt{m/\Delta t}\). At
\(\omega\Delta t=\pi\) (half-period), \(\sin\omega\Delta t\to 0\) and
\(\sqrt{D}\to\infty\): this is the familiar caustic (focal point) where
the semiclassical approximation breaks down because the classical flow
focuses all initial momenta onto a single final point.

\hypertarget{delta-at-a-point-point-interactions-as-rank-one-kernels}{%
\section{4. Delta at a point: point interactions as rank-one
kernels}\label{delta-at-a-point-point-interactions-as-rank-one-kernels}}

A point interaction {[}AlbeverioGesztesyHoeghKrohnHolden2005{]} is
naturally the rank-one operator \[
V=g\,|0\rangle\langle0|.
\] In the half-density kernel calculus this is written as the
bi-half-density distribution supported at \((0,0)\): \[
\mathsf K_V(x,y)=g\;\delta^{(d)}(x)\,\delta^{(d)}(y)\,|dx|^{1/2}|dy|^{1/2}.
\] This is the ``projector-like delta'' object underlying contact
interactions.

\hypertarget{where-scales-enter-upon-scalarization-and-why-rg-invariants-are-natural-candidates}{%
\section{5. Where scales enter upon scalarization (and why RG invariants
are natural
candidates)}\label{where-scales-enter-upon-scalarization-and-why-rg-invariants-are-natural-candidates}}

Half-density kernels are canonical; scalar representatives are not.
Choosing a reference half-density \(\sigma_\ast\) identifies any
half-density \(\psi\) with a scalar \(f\) via \(\psi=f\,\sigma_\ast\).
If one insists that scalar representatives be dimensionless, then
\(\sigma_\ast\) must carry a \(\text{length}^{d/2}\) constant.

In marginal cases (notably the 2D point interaction), renormalization
generates an RG-invariant scale \(\kappa_\ast\) (dimensional
transmutation). This suggests a conditional identification: if one adds
a universality hypothesis that scalarization scales must be built from
physical invariants, then RG-invariant scales are natural candidates to
supply the missing \(\text{length}^{d/2}\) factors required by
scalarization.

This note treats that identification as an organizing perspective, not
as a theorem.

\hypertarget{outlook}{%
\section{6. Outlook}\label{outlook}}

\begin{enumerate}
\def\labelenumi{\arabic{enumi}.}
\tightlist
\item
  \sout{Relate determinant weights to Van Vleck type.} Addressed:
  Section 3.5 makes the connection explicit.
\item
  Clarify which parts of the ``functional delta \(\delta(\delta S)\)''
  story survive rigorous regularization and which remain heuristic.
\item
  Extend the half-density treatment to spacetime (Lorentzian)
  propagators and distributional kernels in field theory.
\end{enumerate}

\hypertarget{references}{%
\section{References}\label{references}}

\begin{enumerate}
\def\labelenumi{\arabic{enumi}.}
\tightlist
\item
  {[}VanVleck1928Correspondence{]} J. H. Van Vleck, ``The Correspondence
  Principle in the Statistical Interpretation of Quantum Mechanics,''
  \emph{Proceedings of the National Academy of Sciences of the United
  States of America} 14(2) (1928), 178--188. DOI
  \texttt{10.1073/pnas.14.2.178}.
\item
  {[}BatesWeinstein1997{]} Sean Bates and Alan Weinstein, ``Lectures on
  the Geometry of Quantization,'' Berkeley Mathematics Lecture Notes,
  vol.~8, AMS, 1997. ISBN \texttt{978-0-8218-0798-9}. OA:
  \url{https://math.berkeley.edu/~alanw/GofQ.pdf}. (Canonical reference
  for half-density formalism in geometric quantization; half-density
  kernels and composition.)
\item
  {[}Hormander2003{]} Lars Hörmander, \emph{The Analysis of Linear
  Partial Differential Operators I: Distribution Theory and Fourier
  Analysis}, 2nd ed., Springer, 2003. DOI
  \texttt{10.1007/978-3-642-61497-2}. (Schwartz kernel theorem;
  distributional calculus for PDE Green functions.)
\item
  {[}AlbeverioGesztesyHoeghKrohnHolden2005{]} S. Albeverio, F. Gesztesy,
  R. Høegh-Krohn, and H. Holden, \emph{Solvable Models in Quantum
  Mechanics}, 2nd ed., AMS Chelsea Publishing, 2005. ISBN
  \texttt{978-0-8218-3624-4}. (Canonical reference for point
  interactions in quantum mechanics; self-adjoint extensions, delta
  potentials.)
\item
  {[}Morette1951{]} C. Morette, ``On the Definition and Approximation of
  Feynman's Path Integrals,'' \emph{Phys. Rev.} \textbf{81}, 848--852
  (1951). DOI \texttt{10.1103/PhysRev.81.848}. (Van Vleck determinant in
  path integral; semiclassical expansion weights.)
\end{enumerate}

\end{document}
