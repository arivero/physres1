% Options for packages loaded elsewhere
\PassOptionsToPackage{unicode}{hyperref}
\PassOptionsToPackage{hyphens}{url}
\documentclass[
]{article}
\usepackage{xcolor}
\usepackage{amsmath,amssymb}
\setcounter{secnumdepth}{-\maxdimen} % remove section numbering
\usepackage{iftex}
\ifPDFTeX
  \usepackage[T1]{fontenc}
  \usepackage[utf8]{inputenc}
  \usepackage{textcomp} % provide euro and other symbols
\else % if luatex or xetex
  \usepackage{unicode-math} % this also loads fontspec
  \defaultfontfeatures{Scale=MatchLowercase}
  \defaultfontfeatures[\rmfamily]{Ligatures=TeX,Scale=1}
\fi
\usepackage{lmodern}
\ifPDFTeX\else
  % xetex/luatex font selection
\fi
% Use upquote if available, for straight quotes in verbatim environments
\IfFileExists{upquote.sty}{\usepackage{upquote}}{}
\IfFileExists{microtype.sty}{% use microtype if available
  \usepackage[]{microtype}
  \UseMicrotypeSet[protrusion]{basicmath} % disable protrusion for tt fonts
}{}
\makeatletter
\@ifundefined{KOMAClassName}{% if non-KOMA class
  \IfFileExists{parskip.sty}{%
    \usepackage{parskip}
  }{% else
    \setlength{\parindent}{0pt}
    \setlength{\parskip}{6pt plus 2pt minus 1pt}}
}{% if KOMA class
  \KOMAoptions{parskip=half}}
\makeatother
\setlength{\emergencystretch}{3em} % prevent overfull lines
\providecommand{\tightlist}{%
  \setlength{\itemsep}{0pt}\setlength{\parskip}{0pt}}
\usepackage{bookmark}
\IfFileExists{xurl.sty}{\usepackage{xurl}}{} % add URL line breaks if available
\urlstyle{same}
\hypersetup{
  pdftitle={Action--Angle Indeterminacy in Central Potentials: A Referee-Safe Witness},
  hidelinks,
  pdfcreator={LaTeX via pandoc}}

\title{Action--Angle Indeterminacy in Central Potentials: A Referee-Safe
Witness}
\author{}
\date{}

\begin{document}
\maketitle
\begin{abstract}
``Action--angle indeterminacy'' should not be read as a force-range
heuristic (in the style of energy--time slogans), but as a clean
conjugacy statement: sharpening an action variable broadens the
conjugate angle variable. For central potentials the safest, most
explicit instance is the azimuthal pair \((\phi,L_z)\): an \(L_z\)
eigenstate has \(\phi\)-dependence \(e^{im\phi}\), hence a uniform
azimuthal probability distribution; conversely, any state localized in
\(\phi\) must involve a broad superposition of angular-momentum modes
(Fourier on the circle). This note records that witness and explains its
foundations-level message: classical orbit-phase/orientation pictures
correspond to semiclassical packets/superpositions rather than single
stationary eigenstates.
\end{abstract}

\section{1. Purpose and scope}\label{purpose-and-scope}

This dependent note isolates one specific ``action--angle
indeterminacy'' statement that is both explicit and referee-safe in a
central potential: \textbf{\(\phi\) is delocalized in an \(L_z\)
eigenstate}, and conversely \textbf{localizing \(\phi\) requires a
superposition over many \(m\)} modes.

We deliberately keep the scope bounded. We do \textbf{not} enter the
self-adjoint ``angle operator'' debate; instead we use the standard
circle/Fourier structure and the unitary phase variable \(e^{i\phi}\).
We also do \textbf{not} make any claims about the range of forces or
potentials; the point here is about \textbf{which variables can be
simultaneously sharp} in stationary states.

\section{\texorpdfstring{2. The safe conjugate pair on the circle:
\(\phi\) and
\(L_z\)}{2. The safe conjugate pair on the circle: \textbackslash phi and L\_z}}\label{the-safe-conjugate-pair-on-the-circle-phi-and-l_z}

In spherical coordinates the azimuthal angle is periodic,
\(\phi\sim\phi+2\pi\). The generator of rotations about the \(z\)-axis
is \[
L_z=-i\hbar\,\frac{\partial}{\partial\phi}.
\] The periodicity makes the naive commutator \([\phi,L_z]=i\hbar\)
subtle if one insists on an everywhere-defined self-adjoint \(\phi\)
operator. A standard way to stay on safe ground is to use the unitary
``phase'' variable \[
E := e^{i\phi}.
\] Acting on \(2\pi\)-periodic wavefunctions, \(E\) is well-defined and
satisfies the canonical shift relation \[
[L_z,E]=\hbar\,E,
\] which already captures the operational content: sharp \(L_z\) implies
maximal delocalization in the conjugate angle.

\section{\texorpdfstring{3. Central potentials: \(L_z\) eigenstates have
uniform \(\phi\)
distribution}{3. Central potentials: L\_z eigenstates have uniform \textbackslash phi distribution}}\label{central-potentials-l_z-eigenstates-have-uniform-phi-distribution}

For a central potential (or any Hamiltonian commuting with \(L_z\)), one
may choose simultaneous eigenstates of \(L_z\). In the standard
separation of variables, the azimuthal dependence of an angular-momentum
eigenstate is the Fourier mode \(e^{im\phi}\) with integer \(m\) (for
example in the spherical-harmonic factor
\(Y_{\ell m}(\theta,\phi)\propto P_{\ell m}(\cos\theta)e^{im\phi}\))
{[}TongQMLectures{]}.

Thus an \(L_z\) eigenstate may be written as \[
\psi(r,\theta,\phi)=F(r,\theta)\,e^{im\phi},
\qquad m\in\mathbb Z,
\] and therefore \[
|\psi(r,\theta,\phi)|^2 = |F(r,\theta)|^2,
\] independent of \(\phi\). In particular, the marginal distribution of
\(\phi\) is uniform on \([0,2\pi)\). This is the minimal ``angle
indeterminacy'' witness for central potentials.

\section{\texorpdfstring{4. Fourier tradeoff: localizing \(\phi\) forces
a broad
\(m\)-superposition}{4. Fourier tradeoff: localizing \textbackslash phi forces a broad m-superposition}}\label{fourier-tradeoff-localizing-phi-forces-a-broad-m-superposition}

Any square-integrable \(2\pi\)-periodic function admits a Fourier series
\[
\psi(\phi)=\sum_{m\in\mathbb Z} c_m e^{im\phi},
\qquad
\sum_{m\in\mathbb Z}|c_m|^2<\infty.
\] If only one Fourier mode is present (sharp \(m\), hence sharp
\(L_z\)), then \(|\psi(\phi)|^2\) is constant; conversely, a state that
is peaked in \(\phi\) necessarily uses many Fourier modes (broad
\(m\)-support).

\texttt{Example\ 4.1\ (Dirichlet-kernel\ packet).} The normalized
superposition of modes \(-M\le m\le M\), \[
\psi_M(\phi)=\frac{1}{\sqrt{2\pi(2M+1)}}\sum_{m=-M}^{M} e^{im\phi},
\] is peaked near \(\phi=0\) with an angular width that scales like
\(1/M\), while its \(m\)-distribution is spread across
\(\{-M,\dots,M\}\). This makes the ``sharpening \(\phi\) \(\Rightarrow\)
broadening \(L_z\)'' tradeoff completely explicit without invoking any
disputed angle-operator formalism.

\section{5. Foundations message: orbit pictures require
packets/superpositions}\label{foundations-message-orbit-pictures-require-packetssuperpositions}

This witness supports a simple interpretive guardrail for central-force
intuition: a single stationary eigenstate (even when it carries
classical-sounding quantum numbers) is typically \textbf{not} a
localized classical orbit with a definite phase/orientation. Variables
like the azimuthal phase \(\phi\) (and, in more structured integrable
cases, other angle variables on the invariant torus) become localized
only in \textbf{coherent superpositions} of many stationary modes.

In other words, ``classical orbit pictures'' correspond to semiclassical
packets and stationary-phase concentration, not to exact eigenstates
that are sharp in the conserved actions.

\section{6. Outlook (kept minimal)}\label{outlook-kept-minimal}

Beyond the \((\phi,L_z)\) sector, integrable central problems admit a
fuller action--angle description (with a radial action and additional
angle variables on the invariant torus). EBK/WKB quantization makes the
same structural point: the more sharply the actions are specified, the
less information remains in the conjugate phases. Hardening that broader
story into a standalone foundations claim would require a separate study
cycle to avoid conflating (i) action--angle existence/global issues with
(ii) semiclassical quantization conditions.

\section{References}\label{references}

\begin{enumerate}
\def\labelenumi{\arabic{enumi}.}
\tightlist
\item
  {[}TongQMLectures{]} David Tong, ``Quantum Mechanics'' (lecture notes,
  no DOI). OA: lecture-note PDF. (Contains
  \(Y_{l,m}(\theta,\phi)=P_{l,m}(\cos\theta)e^{im\phi}\) as simultaneous
  eigenstates of \(L^2\) and \(L_z\).)
\end{enumerate}

\end{document}
