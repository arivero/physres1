% Options for packages loaded elsewhere
\PassOptionsToPackage{unicode}{hyperref}
\PassOptionsToPackage{hyphens}{url}
%
\documentclass[
]{article}
\usepackage{amsmath,amssymb}
\usepackage{lmodern}
\usepackage{iftex}
\ifPDFTeX
  \usepackage[T1]{fontenc}
  \usepackage[utf8]{inputenc}
  \usepackage{textcomp} % provide euro and other symbols
\else % if luatex or xetex
  \usepackage{unicode-math}
  \defaultfontfeatures{Scale=MatchLowercase}
  \defaultfontfeatures[\rmfamily]{Ligatures=TeX,Scale=1}
\fi
% Use upquote if available, for straight quotes in verbatim environments
\IfFileExists{upquote.sty}{\usepackage{upquote}}{}
\IfFileExists{microtype.sty}{% use microtype if available
  \usepackage[]{microtype}
  \UseMicrotypeSet[protrusion]{basicmath} % disable protrusion for tt fonts
}{}
\makeatletter
\@ifundefined{KOMAClassName}{% if non-KOMA class
  \IfFileExists{parskip.sty}{%
    \usepackage{parskip}
  }{% else
    \setlength{\parindent}{0pt}
    \setlength{\parskip}{6pt plus 2pt minus 1pt}}
}{% if KOMA class
  \KOMAoptions{parskip=half}}
\makeatother
\usepackage{xcolor}
\IfFileExists{xurl.sty}{\usepackage{xurl}}{} % add URL line breaks if available
\IfFileExists{bookmark.sty}{\usepackage{bookmark}}{\usepackage{hyperref}}
\hypersetup{
  pdftitle={Action--Angle Indeterminacy in Central Potentials: A Referee-Safe Witness},
  pdfauthor={Alejandro Rivero},
  hidelinks,
  pdfcreator={LaTeX via pandoc}}
\urlstyle{same} % disable monospaced font for URLs
\setlength{\emergencystretch}{3em} % prevent overfull lines
\providecommand{\tightlist}{%
  \setlength{\itemsep}{0pt}\setlength{\parskip}{0pt}}
\setcounter{secnumdepth}{-\maxdimen} % remove section numbering
\ifLuaTeX
  \usepackage{selnolig}  % disable illegal ligatures
\fi

\title{Action--Angle Indeterminacy in Central Potentials: A Referee-Safe
Witness}
\author{Alejandro Rivero}
\date{2026}

\begin{document}
\maketitle
\begin{abstract}
``Action--angle indeterminacy'' should not be read as a force-range
heuristic (in the style of energy--time slogans), but as a clean
conjugacy statement: sharpening an action variable broadens the
conjugate angle variable. For central potentials the safest, most
explicit instance is the azimuthal pair \((\phi,L_z)\): an \(L_z\)
eigenstate has \(\phi\)-dependence \(e^{im\phi}\), hence a uniform
azimuthal probability distribution; conversely, any state localized in
\(\phi\) must involve a broad superposition of angular-momentum modes
(Fourier on the circle). This note records that witness and explains its
foundations-level message: classical orbit-phase/orientation pictures
correspond to semiclassical packets/superpositions rather than single
stationary eigenstates.
\end{abstract}

\hypertarget{purpose-and-scope}{%
\section{1. Purpose and scope}\label{purpose-and-scope}}

This dependent note isolates one specific ``action--angle
indeterminacy'' statement that is both explicit and referee-safe in a
central potential: \textbf{\(\phi\) is delocalized in an \(L_z\)
eigenstate}, and conversely \textbf{localizing \(\phi\) requires a
superposition over many \(m\)} modes.

We deliberately keep the scope bounded. We do \textbf{not} enter the
self-adjoint ``angle operator'' debate; instead we use the standard
circle/Fourier structure and the unitary phase variable \(e^{i\phi}\).
We also do \textbf{not} make any claims about the range of forces or
potentials; the point here is about \textbf{which variables can be
simultaneously sharp} in stationary states.

\hypertarget{the-safe-conjugate-pair-on-the-circle-phi-and-l_z}{%
\section{\texorpdfstring{2. The safe conjugate pair on the circle:
\(\phi\) and
\(L_z\)}{2. The safe conjugate pair on the circle: \textbackslash phi and L\_z}}\label{the-safe-conjugate-pair-on-the-circle-phi-and-l_z}}

In spherical coordinates the azimuthal angle is periodic,
\(\phi\sim\phi+2\pi\). The generator of rotations about the \(z\)-axis
is \[
L_z=-i\hbar\,\frac{\partial}{\partial\phi}.
\] The periodicity makes the naive commutator \([\phi,L_z]=i\hbar\)
subtle if one insists on an everywhere-defined self-adjoint \(\phi\)
operator. A standard way to stay on safe ground is to use the unitary
``phase'' variable \[
E := e^{i\phi}.
\] Acting on \(2\pi\)-periodic wavefunctions, \(E\) is well-defined and
satisfies the canonical shift relation \[
[L_z,E]=\hbar\,E,
\] which already captures the operational content: sharp \(L_z\) implies
maximal delocalization in the conjugate angle.

\hypertarget{central-potentials-l_z-eigenstates-have-uniform-phi-distribution}{%
\section{\texorpdfstring{3. Central potentials: \(L_z\) eigenstates have
uniform \(\phi\)
distribution}{3. Central potentials: L\_z eigenstates have uniform \textbackslash phi distribution}}\label{central-potentials-l_z-eigenstates-have-uniform-phi-distribution}}

For a central potential (or any Hamiltonian commuting with \(L_z\)), one
may choose simultaneous eigenstates of \(L_z\). In the standard
separation of variables, the azimuthal dependence of an angular-momentum
eigenstate is the Fourier mode \(e^{im\phi}\) with integer \(m\) (for
example in the spherical-harmonic factor
\(Y_{\ell m}(\theta,\phi)\propto P_{\ell m}(\cos\theta)e^{im\phi}\))
{[}TongQMLectures{]}.

Thus an \(L_z\) eigenstate may be written as \[
\psi(r,\theta,\phi)=F(r,\theta)\,e^{im\phi},
\qquad m\in\mathbb Z,
\] and therefore \[
|\psi(r,\theta,\phi)|^2 = |F(r,\theta)|^2,
\] independent of \(\phi\). In particular, the marginal distribution of
\(\phi\) is uniform on \([0,2\pi)\). This is the minimal ``angle
indeterminacy'' witness for central potentials.

\hypertarget{fourier-tradeoff-localizing-phi-forces-a-broad-m-superposition}{%
\section{\texorpdfstring{4. Fourier tradeoff: localizing \(\phi\) forces
a broad
\(m\)-superposition}{4. Fourier tradeoff: localizing \textbackslash phi forces a broad m-superposition}}\label{fourier-tradeoff-localizing-phi-forces-a-broad-m-superposition}}

Any square-integrable \(2\pi\)-periodic function admits a Fourier series
\[
\psi(\phi)=\sum_{m\in\mathbb Z} c_m e^{im\phi},
\qquad
\sum_{m\in\mathbb Z}|c_m|^2<\infty.
\] If only one Fourier mode is present (sharp \(m\), hence sharp
\(L_z\)), then \(|\psi(\phi)|^2\) is constant; conversely, a state that
is peaked in \(\phi\) necessarily uses many Fourier modes (broad
\(m\)-support).

\texttt{Example\ 4.1\ (Dirichlet-kernel\ packet).} The normalized
superposition of modes \(-M\le m\le M\), \[
\psi_M(\phi)=\frac{1}{\sqrt{2\pi(2M+1)}}\sum_{m=-M}^{M} e^{im\phi},
\] is peaked near \(\phi=0\) with an angular width that scales like
\(1/M\), while its \(m\)-distribution is spread across
\(\{-M,\dots,M\}\). This makes the ``sharpening \(\phi\) \(\Rightarrow\)
broadening \(L_z\)'' tradeoff completely explicit without invoking any
disputed angle-operator formalism.

The Fourier tradeoff above can be made into a sharp quantitative bound
using only the self-adjoint observables \(\cos\phi\) and \(\sin\phi\):

\texttt{Proposition\ 4.2\ (Circular\ uncertainty\ relation).} For any
state on the circle, define the circular concentration
\(R=|\langle e^{i\phi}\rangle|\in[0,1]\). Adding the Robertson
inequalities for the two self-adjoint pairs \((L_z,\cos\phi)\) and
\((L_z,\sin\phi)\) --- using \([L_z,\cos\phi]=i\hbar\sin\phi\) and
\([L_z,\sin\phi]=-i\hbar\cos\phi\) --- and the identity
\(\mathrm{Var}(\cos\phi)+\mathrm{Var}(\sin\phi)=1-R^2\), gives \[
\mathrm{Var}(L_z)\cdot(1-R^2)\ge\frac{\hbar^2}{4}\,R^2.
\] When \(R=0\) (uniform distribution, as in an \(L_z\) eigenstate) the
bound is trivial. As \(R\to1\) (sharply localized angle) the bound
forces \(\mathrm{Var}(L_z)\to\infty\): angular localization requires
spreading across many \(m\)-modes. This quantifies the Fourier tradeoff
above without invoking a self-adjoint angle operator.

\texttt{Example\ 4.3\ (Verifying\ the\ bound\ for\ the\ Dirichlet-kernel\ packet).}
For the state \(\psi_M\) of Example 4.1, the circular concentration is
\(R=\langle e^{i\phi}\rangle = 2M/(2M+1)\) (by orthogonality, only the
\(2M\) consecutive pairs \((m,m+1)\) with both in \(\{-M,\dots,M\}\)
contribute). The angular-momentum variance is
\(\mathrm{Var}(L_z)=\hbar^2 M(M+1)/3\) (using
\(\sum_{m=1}^M m^2 = M(M+1)(2M+1)/6\) and \(\langle L_z\rangle=0\) by
symmetry). The ratio of the left-hand side to the right-hand side of the
bound in Proposition 4.2 is \[
\frac{\mathrm{Var}(L_z)\,(1-R^2)}{(\hbar^2/4)\,R^2}
=\frac{(M+1)(4M+1)}{3M},
\] which equals \(10/3\approx 3.3\) at \(M=1\) and grows as \(4M/3\) for
large \(M\). The inequality is satisfied with increasing slack: the
Dirichlet kernel is far from a minimum-uncertainty state for the
circular relation. Physically, narrower angular packets (\(R\to 1\))
require disproportionately more angular-momentum spread than the bound
demands.

\texttt{Remark\ 4.4\ (Near-optimal\ angular\ localization:\ the\ von\ Mises\ state).}
The rectangular Fourier profile of the Dirichlet kernel wastes
angular-momentum variance on sidelobes, driving the ratio LHS/RHS to
\(4M/3\). The angular analog of a Gaussian --- the von Mises
wavefunction \(\psi(\phi)\propto \exp(\kappa\cos\phi)\) --- has Fourier
coefficients \(c_m\propto I_m(\kappa)\) (modified Bessel functions) that
decay smoothly. For large \(\kappa\) the coefficients are approximately
Gaussian in \(m\) with width \(\sqrt{\kappa}\), giving
\(\mathrm{Var}(L_z)\approx\hbar^2\kappa/2\), while the circular
concentration satisfies \(1-R^2\approx 1/(2\kappa)\) (since the
probability \(|\psi|^2\propto\exp(2\kappa\cos\phi)\) is a von Mises
distribution with parameter \(2\kappa\)). The ratio
\(\mathrm{Var}(L_z)(1-R^2)/[(\hbar^2/4)R^2]\to 1\) as
\(\kappa\to\infty\): the von Mises state asymptotically saturates the
bound in Proposition 4.2.

\hypertarget{foundations-message-orbit-pictures-require-packetssuperpositions}{%
\section{5. Foundations message: orbit pictures require
packets/superpositions}\label{foundations-message-orbit-pictures-require-packetssuperpositions}}

This witness supports a simple interpretive guardrail for central-force
intuition: a single stationary eigenstate (even when it carries
classical-sounding quantum numbers) is typically \textbf{not} a
localized classical orbit with a definite phase/orientation. Variables
like the azimuthal phase \(\phi\) (and, in more structured integrable
cases, other angle variables on the invariant torus) become localized
only in \textbf{coherent superpositions} of many stationary modes.

In other words, ``classical orbit pictures'' correspond to semiclassical
packets and stationary-phase concentration, not to exact eigenstates
that are sharp in the conserved actions.

\texttt{Remark\ 5.1\ (Temporal\ coherence\ and\ quantum\ revivals).} The
superpositions that localize an angle variable also have a temporal
constraint: for anharmonic spectra (\(d^2E/dm^2\neq 0\)), the packet
disperses on a timescale
\(t_{\mathrm{disp}}\sim\hbar/(|d^2E/dm^2|\,\Delta m)\) and reforms at
the revival time \(t_{\mathrm{rev}}\sim 2\pi\hbar/|d^2E/dm^2|\). Only
for a linear spectrum (\(d^2E/dm^2=0\)) does the packet rotate rigidly
like a classical orbit for all time. Thus classical orbit pictures
require not only spatial localization (many \(m\)-modes, Section 4) but
also approximate spectral linearity for temporal coherence.

\hypertarget{a-second-witness-the-harmonic-oscillator}{%
\section{6. A second witness: the harmonic
oscillator}\label{a-second-witness-the-harmonic-oscillator}}

The same structure appears in the simplest one-dimensional integrable
system.

\texttt{Example\ 6.1\ (Harmonic\ oscillator:\ Fock\ states\ vs\ coherent\ states).}
For a harmonic oscillator of frequency \(\omega\), define the classical
action variable \(J=E/\omega\). The quantum Fock states \(|n\rangle\)
are the action eigenstates (\(J_n=(n+\tfrac12)\hbar\)), and their
phase-space (Husimi) distribution is a ring centered at the origin ---
the orbit phase \(\theta\) is uniformly delocalized, exactly as \(\phi\)
is delocalized in an \(L_z\) eigenstate. Conversely, a coherent state \[
|\alpha\rangle
=e^{-|\alpha|^2/2}\sum_{n=0}^{\infty}\frac{\alpha^n}{\sqrt{n!}}\,|n\rangle,
\qquad \alpha=|\alpha|\,e^{i\theta_0},
\] is the closest quantum analog of a classical orbit with definite
amplitude \(|\alpha|\) and phase \(\theta_0\). Its Fock-state weights
follow a Poisson distribution with mean \(\bar n=|\alpha|^2\), so
localizing the phase to width \(\Delta\theta\sim 1/|\alpha|\) requires
spreading the action over \(\Delta n\sim |\alpha|\) modes. The tradeoff
is the same as in Section 4: sharp action implies delocalized phase, and
vice versa.

\texttt{Example\ 6.2\ (Hydrogen\ atom:\ three\ action–angle\ pairs).} In
the hydrogen atom, the \(n^2\)-fold degeneracy (\(E_n\) depending only
on the principal quantum number \(n\)) reflects an enhanced \(SO(4)\)
symmetry {[}Sakurai2020{]}. Semiclassically, the bound orbits lie on a
three-torus with action integrals quantized by \((n,\ell,m)\). A
stationary eigenstate \(|n,\ell,m\rangle\) is sharp in all three actions
and therefore delocalized in all three conjugate angles: the azimuthal
phase \(\phi\) is uniform (Section 3), the in-plane orbit orientation
has no preferred direction (the Runge--Lenz vector has vanishing
expectation value, since it connects states of different \(\ell\)), and
the radial probability \(|R_{n\ell}(r)|^2\) is time-independent --- the
sharp radial action leaves the conjugate radial phase uniformly
delocalized. A classical Keplerian ellipse with definite eccentricity,
orientation, and timing requires a coherent superposition over ranges of
\((n,\ell,m)\), just as a coherent state in Example 6.1 requires
superposing many Fock states.

\texttt{Remark\ 6.3\ (EBK\ quantization\ on\ the\ invariant\ torus).}
For a classically integrable system with \(d\) degrees of freedom, the
Arnold--Liouville theorem provides \(d\) action variables
\(I_k=(2\pi)^{-1}\oint_{\gamma_k}p\cdot dq\), integrated around the
independent cycles \(\gamma_k\) of the invariant \(d\)-torus. The EBK
(Einstein--Brillouin--Keller) quantization condition requires \[
I_k=\left(n_k+\frac{\alpha_k}{4}\right)\hbar,\qquad n_k\in\mathbb{Z}_{\ge0},
\] where \(\alpha_k\) is the Maslov index of the \(k\)-th cycle
(counting caustic/turning-point contributions). The integer quantum
numbers \(n_k\) select the torus; the conjugate angle variables
\(\theta_k\in[0,2\pi)\) are uniformly distributed on that torus and
carry no quantum-number information. This is the semiclassical
counterpart of the fully quantum statement: stationary eigenstates
(sharp actions) have delocalized angles. Examples 6.1 and 6.2 are the
exact quantum versions of this principle for the \(d=1\) and \(d=3\)
cases.

\texttt{Remark\ 6.4\ (Husimi\ function:\ visualizing\ action–angle\ states\ in\ phase\ space).}
The Husimi \(Q\)-function
\(Q(\alpha)=\langle\alpha|\hat\rho|\alpha\rangle/\pi\) assigns a
non-negative quasiprobability to each phase-space point \(\alpha\),
using coherent states as the reference frame. For a Fock state
\(|n\rangle\), \(Q(\alpha)=e^{-|\alpha|^2}|\alpha|^{2n}/(\pi\,n!)\) ---
a ring at radius \(|\alpha|=\sqrt{n}\), uniform in the phase angle:
sharp action, fully delocalized angle. For a coherent state
\(|\alpha_0\rangle\), \(Q(\alpha)=e^{-|\alpha-\alpha_0|^2}/\pi\) --- a
Gaussian blob centered at \(\alpha_0\), simultaneously localizing both
action and angle to uncertainty-limited width. The ring-versus-blob
distinction is the phase-space portrait of the Fourier tradeoff in
Section 4, with the Husimi function providing a literal (non-negative)
probability picture that the Wigner function's sign changes would
obscure.

\hypertarget{outlook-kept-minimal}{%
\section{7. Outlook (kept minimal)}\label{outlook-kept-minimal}}

The preceding witnesses illustrate the action--angle tradeoff in systems
with one, two, and three action--angle pairs, and Remark 6.3 shows that
EBK quantization makes the same structural point in general: the more
sharply the actions are specified, the less information remains in the
conjugate phases.

\texttt{Remark\ 7.1\ (Boundary\ at\ integrability\ breaking).} The
action--angle framework of Sections 3--6 presupposes the existence of
global action variables, hence applies exactly to integrable systems.
For nearly integrable Hamiltonians, the KAM theorem guarantees
persistence of most invariant tori (those with sufficiently irrational
frequency ratios), on which the framework remains valid. In fully
chaotic systems, the absence of conserved actions replaces the
structured Fourier tradeoff with eigenstate thermalization --- a more
drastic delocalization where individual energy eigenstates appear
thermal for local observables, with no residual action--angle structure
to organize the uncertainty.

\hypertarget{references}{%
\section{References}\label{references}}

\begin{enumerate}
\def\labelenumi{\arabic{enumi}.}
\tightlist
\item
  {[}TongQMLectures{]} David Tong, ``Quantum Mechanics'' (lecture notes,
  no DOI). OA: lecture-note PDF. (Contains
  \(Y_{l,m}(\theta,\phi)=P_{l,m}(\cos\theta)e^{im\phi}\) as simultaneous
  eigenstates of \(L^2\) and \(L_z\).)
\item
  {[}Sakurai2020{]} J. J. Sakurai and Jim Napolitano, \emph{Modern
  Quantum Mechanics}, 3rd ed., Cambridge University Press, 2020. ISBN
  \texttt{978-1-108-47322-4}. (Standard treatment of angular momentum,
  spherical harmonics, and quantum measurement.)
\end{enumerate}

\end{document}
