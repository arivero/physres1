% Options for packages loaded elsewhere
\PassOptionsToPackage{unicode}{hyperref}
\PassOptionsToPackage{hyphens}{url}
%
\documentclass[
]{article}
\usepackage{amsmath,amssymb}
\usepackage{lmodern}
\usepackage{iftex}
\ifPDFTeX
  \usepackage[T1]{fontenc}
  \usepackage[utf8]{inputenc}
  \usepackage{textcomp} % provide euro and other symbols
\else % if luatex or xetex
  \usepackage{unicode-math}
  \defaultfontfeatures{Scale=MatchLowercase}
  \defaultfontfeatures[\rmfamily]{Ligatures=TeX,Scale=1}
\fi
% Use upquote if available, for straight quotes in verbatim environments
\IfFileExists{upquote.sty}{\usepackage{upquote}}{}
\IfFileExists{microtype.sty}{% use microtype if available
  \usepackage[]{microtype}
  \UseMicrotypeSet[protrusion]{basicmath} % disable protrusion for tt fonts
}{}
\makeatletter
\@ifundefined{KOMAClassName}{% if non-KOMA class
  \IfFileExists{parskip.sty}{%
    \usepackage{parskip}
  }{% else
    \setlength{\parindent}{0pt}
    \setlength{\parskip}{6pt plus 2pt minus 1pt}}
}{% if KOMA class
  \KOMAoptions{parskip=half}}
\makeatother
\usepackage{xcolor}
\IfFileExists{xurl.sty}{\usepackage{xurl}}{} % add URL line breaks if available
\IfFileExists{bookmark.sty}{\usepackage{bookmark}}{\usepackage{hyperref}}
\hypersetup{
  pdftitle={Action--Angle Indeterminacy in Central Potentials: A Referee-Safe Witness},
  pdfauthor={Alejandro Rivero},
  hidelinks,
  pdfcreator={LaTeX via pandoc}}
\urlstyle{same} % disable monospaced font for URLs
\setlength{\emergencystretch}{3em} % prevent overfull lines
\providecommand{\tightlist}{%
  \setlength{\itemsep}{0pt}\setlength{\parskip}{0pt}}
\setcounter{secnumdepth}{-\maxdimen} % remove section numbering
\ifLuaTeX
  \usepackage{selnolig}  % disable illegal ligatures
\fi

\title{Action--Angle Indeterminacy in Central Potentials: A Referee-Safe
Witness}
\author{Alejandro Rivero}
\date{2026}

\begin{document}
\maketitle
\begin{abstract}
``Action--angle indeterminacy'' should not be read as a force-range
heuristic (in the style of energy--time slogans), but as a clean
conjugacy statement: sharpening an action variable broadens the
conjugate angle variable. For central potentials the safest, most
explicit instance is the azimuthal pair \((\phi,L_z)\): an \(L_z\)
eigenstate has \(\phi\)-dependence \(e^{im\phi}\), hence a uniform
azimuthal probability distribution; conversely, any state localized in
\(\phi\) must involve a broad superposition of angular-momentum modes
(Fourier on the circle). This note records that witness and explains its
foundations-level message: classical orbit-phase/orientation pictures
correspond to semiclassical packets/superpositions rather than single
stationary eigenstates.
\end{abstract}

\hypertarget{purpose-and-scope}{%
\section{1. Purpose and scope}\label{purpose-and-scope}}

This dependent note isolates one specific ``action--angle
indeterminacy'' statement that is both explicit and referee-safe in a
central potential: \textbf{\(\phi\) is delocalized in an \(L_z\)
eigenstate}, and conversely \textbf{localizing \(\phi\) requires a
superposition over many \(m\)} modes.

We deliberately keep the scope bounded. We do \textbf{not} enter the
self-adjoint ``angle operator'' debate; instead we use the standard
circle/Fourier structure and the unitary phase variable \(e^{i\phi}\).
We also do \textbf{not} make any claims about the range of forces or
potentials; the point here is about \textbf{which variables can be
simultaneously sharp} in stationary states.

\hypertarget{the-safe-conjugate-pair-on-the-circle-phi-and-l_z}{%
\section{\texorpdfstring{2. The safe conjugate pair on the circle:
\(\phi\) and
\(L_z\)}{2. The safe conjugate pair on the circle: \textbackslash phi and L\_z}}\label{the-safe-conjugate-pair-on-the-circle-phi-and-l_z}}

In spherical coordinates the azimuthal angle is periodic,
\(\phi\sim\phi+2\pi\). The generator of rotations about the \(z\)-axis
is \[
L_z=-i\hbar\,\frac{\partial}{\partial\phi}.
\] The periodicity makes the naive commutator \([\phi,L_z]=i\hbar\)
subtle if one insists on an everywhere-defined self-adjoint \(\phi\)
operator. A standard way to stay on safe ground is to use the unitary
``phase'' variable \[
E := e^{i\phi}.
\] Acting on \(2\pi\)-periodic wavefunctions, \(E\) is well-defined and
satisfies the canonical shift relation \[
[L_z,E]=\hbar\,E,
\] which already captures the operational content: sharp \(L_z\) implies
maximal delocalization in the conjugate angle.

\texttt{Remark\ 2.1\ (Number–phase\ pair:\ the\ oscillator\ counterpart).}
The same structure appears for the harmonic-oscillator number--phase
pair \((N,\theta)\). The number operator \(N=\hat a^\dagger\hat a\) has
non-negative integer spectrum, and the oscillation phase \(\theta\) is
periodic --- the same mathematical setting as \((L_z,\phi)\). The
Susskind--Glogower operator
\(\hat E_-=\sum_{n=0}^\infty|n\rangle\langle n+1|=\hat a\,(N+1)^{-1/2}\)
(Susskind and Glogower, 1964) plays the role of \(E=e^{i\phi}\): it
satisfies \([N,\hat E_-]=-\hat E_-\) (lowering \(n\) by one) and its
adjoint \(\hat E_+\) satisfies \([N,\hat E_+]=+\hat E_+\). However,
\(\hat E_+\hat E_-=I-|0\rangle\langle 0|\): the vacuum projection spoils
exact unitarity because the spectrum of \(N\) is bounded below, and
there is no state below \(|0\rangle\) to shift into. This is the
Fock-space manifestation of the same obstruction that prevents a
self-adjoint angle operator on the circle. The consequence: a Fock state
\(|n\rangle\) has a completely uniform phase distribution (the ring of
Remark 6.4), just as an \(L_z\) eigenstate has uniform \(\phi\) (Section
3), and localizing the oscillator phase requires a broad superposition
over number states --- exactly the coherent-state construction of
Example 6.1.

\hypertarget{central-potentials-l_z-eigenstates-have-uniform-phi-distribution}{%
\section{\texorpdfstring{3. Central potentials: \(L_z\) eigenstates have
uniform \(\phi\)
distribution}{3. Central potentials: L\_z eigenstates have uniform \textbackslash phi distribution}}\label{central-potentials-l_z-eigenstates-have-uniform-phi-distribution}}

For a central potential (or any Hamiltonian commuting with \(L_z\)), one
may choose simultaneous eigenstates of \(L_z\). In the standard
separation of variables, the azimuthal dependence of an angular-momentum
eigenstate is the Fourier mode \(e^{im\phi}\) with integer \(m\) (for
example in the spherical-harmonic factor
\(Y_{\ell m}(\theta,\phi)\propto P_{\ell m}(\cos\theta)e^{im\phi}\))
{[}TongQMLectures{]}.

Thus an \(L_z\) eigenstate may be written as \[
\psi(r,\theta,\phi)=F(r,\theta)\,e^{im\phi},
\qquad m\in\mathbb Z,
\] and therefore \[
|\psi(r,\theta,\phi)|^2 = |F(r,\theta)|^2,
\] independent of \(\phi\). In particular, the marginal distribution of
\(\phi\) is uniform on \([0,2\pi)\). This is the minimal ``angle
indeterminacy'' witness for central potentials.

\texttt{Remark\ 3.1\ (Real\ spherical\ harmonics:\ directional\ lobes\ from\ the\ minimal\ m-superposition).}
The complex spherical harmonics \(Y_{\ell,m}\) have definite \(m\) and
therefore uniform \(\phi\)-dependence (the main result above). The
``real'' spherical harmonics used in chemistry ---
\(p_x\propto\sin\theta\cos\phi\), \(p_y\propto\sin\theta\sin\phi\),
\(d_{xy}\propto\sin^2\theta\sin 2\phi\), etc. --- are the real and
imaginary parts of \(Y_{\ell,m}\), hence equal-weight superpositions of
\(m\) and \(-m\). This minimal two-mode superposition already breaks
azimuthal uniformity: the probability density acquires \(\cos^2(m\phi)\)
or \(\sin^2(m\phi)\) angular lobes, at the cost of angular-momentum
indeterminacy \(\mathrm{Var}(L_z)=m^2\hbar^2\). The directional orbital
shapes of atomic and molecular physics are thus the simplest instance of
the Fourier tradeoff quantified in Section 4.

\hypertarget{fourier-tradeoff-localizing-phi-forces-a-broad-m-superposition}{%
\section{\texorpdfstring{4. Fourier tradeoff: localizing \(\phi\) forces
a broad
\(m\)-superposition}{4. Fourier tradeoff: localizing \textbackslash phi forces a broad m-superposition}}\label{fourier-tradeoff-localizing-phi-forces-a-broad-m-superposition}}

Any square-integrable \(2\pi\)-periodic function admits a Fourier series
\[
\psi(\phi)=\sum_{m\in\mathbb Z} c_m e^{im\phi},
\qquad
\sum_{m\in\mathbb Z}|c_m|^2<\infty.
\] If only one Fourier mode is present (sharp \(m\), hence sharp
\(L_z\)), then \(|\psi(\phi)|^2\) is constant; conversely, a state that
is peaked in \(\phi\) necessarily uses many Fourier modes (broad
\(m\)-support).

\texttt{Example\ 4.1\ (Dirichlet-kernel\ packet).} The normalized
superposition of modes \(-M\le m\le M\), \[
\psi_M(\phi)=\frac{1}{\sqrt{2\pi(2M+1)}}\sum_{m=-M}^{M} e^{im\phi},
\] is peaked near \(\phi=0\) with an angular width that scales like
\(1/M\), while its \(m\)-distribution is spread across
\(\{-M,\dots,M\}\). This makes the ``sharpening \(\phi\) \(\Rightarrow\)
broadening \(L_z\)'' tradeoff completely explicit without invoking any
disputed angle-operator formalism.

The Fourier tradeoff above can be made into a sharp quantitative bound
using only the self-adjoint observables \(\cos\phi\) and \(\sin\phi\):

\texttt{Proposition\ 4.2\ (Circular\ uncertainty\ relation).} For any
state on the circle, define the circular concentration
\(R=|\langle e^{i\phi}\rangle|\in[0,1]\). Adding the Robertson
inequalities for the two self-adjoint pairs \((L_z,\cos\phi)\) and
\((L_z,\sin\phi)\) --- using \([L_z,\cos\phi]=i\hbar\sin\phi\) and
\([L_z,\sin\phi]=-i\hbar\cos\phi\) --- and the identity
\(\mathrm{Var}(\cos\phi)+\mathrm{Var}(\sin\phi)=1-R^2\), gives \[
\mathrm{Var}(L_z)\cdot(1-R^2)\ge\frac{\hbar^2}{4}\,R^2.
\] When \(R=0\) (uniform distribution, as in an \(L_z\) eigenstate) the
bound is trivial. As \(R\to1\) (sharply localized angle) the bound
forces \(\mathrm{Var}(L_z)\to\infty\): angular localization requires
spreading across many \(m\)-modes. This quantifies the Fourier tradeoff
above without invoking a self-adjoint angle operator.

\texttt{Example\ 4.3\ (Verifying\ the\ bound\ for\ the\ Dirichlet-kernel\ packet).}
For the state \(\psi_M\) of Example 4.1, the circular concentration is
\(R=\langle e^{i\phi}\rangle = 2M/(2M+1)\) (by orthogonality, only the
\(2M\) consecutive pairs \((m,m+1)\) with both in \(\{-M,\dots,M\}\)
contribute). The angular-momentum variance is
\(\mathrm{Var}(L_z)=\hbar^2 M(M+1)/3\) (using
\(\sum_{m=1}^M m^2 = M(M+1)(2M+1)/6\) and \(\langle L_z\rangle=0\) by
symmetry). The ratio of the left-hand side to the right-hand side of the
bound in Proposition 4.2 is \[
\frac{\mathrm{Var}(L_z)\,(1-R^2)}{(\hbar^2/4)\,R^2}
=\frac{(M+1)(4M+1)}{3M},
\] which equals \(10/3\approx 3.3\) at \(M=1\) and grows as \(4M/3\) for
large \(M\). The inequality is satisfied with increasing slack: the
Dirichlet kernel is far from a minimum-uncertainty state for the
circular relation. Physically, narrower angular packets (\(R\to 1\))
require disproportionately more angular-momentum spread than the bound
demands.

\texttt{Remark\ 4.4\ (Near-optimal\ angular\ localization:\ the\ von\ Mises\ state).}
The rectangular Fourier profile of the Dirichlet kernel wastes
angular-momentum variance on sidelobes, driving the ratio LHS/RHS to
\(4M/3\). The angular analog of a Gaussian --- the von Mises
wavefunction \(\psi(\phi)\propto \exp(\kappa\cos\phi)\) --- has Fourier
coefficients \(c_m\propto I_m(\kappa)\) (modified Bessel functions) that
decay smoothly. For large \(\kappa\) the coefficients are approximately
Gaussian in \(m\) with width \(\sqrt{\kappa}\), giving
\(\mathrm{Var}(L_z)\approx\hbar^2\kappa/2\), while the circular
concentration satisfies \(1-R^2\approx 1/(2\kappa)\) (since the
probability \(|\psi|^2\propto\exp(2\kappa\cos\phi)\) is a von Mises
distribution with parameter \(2\kappa\)). The ratio
\(\mathrm{Var}(L_z)(1-R^2)/[(\hbar^2/4)R^2]\to 1\) as
\(\kappa\to\infty\): the von Mises state asymptotically saturates the
bound in Proposition 4.2.

\hypertarget{foundations-message-orbit-pictures-require-packetssuperpositions}{%
\section{5. Foundations message: orbit pictures require
packets/superpositions}\label{foundations-message-orbit-pictures-require-packetssuperpositions}}

This witness supports a simple interpretive guardrail for central-force
intuition: a single stationary eigenstate (even when it carries
classical-sounding quantum numbers) is typically \textbf{not} a
localized classical orbit with a definite phase/orientation. Variables
like the azimuthal phase \(\phi\) (and, in more structured integrable
cases, other angle variables on the invariant torus) become localized
only in \textbf{coherent superpositions} of many stationary modes.

In other words, ``classical orbit pictures'' correspond to semiclassical
packets and stationary-phase concentration, not to exact eigenstates
that are sharp in the conserved actions.

\texttt{Remark\ 5.1\ (Temporal\ coherence\ and\ quantum\ revivals).} The
superpositions that localize an angle variable also have a temporal
constraint: for anharmonic spectra (\(d^2E/dm^2\neq 0\)), the packet
disperses on a timescale
\(t_{\mathrm{disp}}\sim\hbar/(|d^2E/dm^2|\,\Delta m)\) and reforms at
the revival time \(t_{\mathrm{rev}}\sim 2\pi\hbar/|d^2E/dm^2|\). Only
for a linear spectrum (\(d^2E/dm^2=0\)) does the packet rotate rigidly
like a classical orbit for all time. Thus classical orbit pictures
require not only spatial localization (many \(m\)-modes, Section 4) but
also approximate spectral linearity for temporal coherence.

\texttt{Remark\ 5.2\ (Decoherence\ selects\ the\ localized\ packets).}
Environment-induced decoherence provides the dynamical mechanism that
selects the coherent packets of Section 4 over the sharp-action
eigenstates of Section 3. For a harmonic oscillator coupled to a thermal
bath through position, coherent states minimize the rate of entanglement
with the environment and emerge as the preferred ``pointer states'' ---
the states most robust against decoherence {[}ZurekHabibPaz1993{]}. Fock
states, by contrast, decohere rapidly: superpositions of well-separated
number states lose coherence on timescales much shorter than the thermal
relaxation time, because the position operator (through which the
environment couples) does not commute with the number operator.
Classical orbit pictures thus emerge not only from semiclassical
wavepacket structure (Sections 4--5) but from the environment's
dynamical selection of those packets as the robust states.

\texttt{Remark\ 5.3\ (Energy–time\ uncertainty:\ the\ Mandelstam–Tamm\ temporal\ analog).}
The action--angle tradeoff of Sections 3--4 has a temporal counterpart
that avoids the well-known difficulty of defining a self-adjoint time
operator (Pauli's theorem for bounded-below Hamiltonians). Mandelstam
and Tamm (1945) define the ``evolution time'' of an observable \(A\) as
\(\Delta t_A:=\Delta A/|d\langle A\rangle/dt|\) --- the time for
\(\langle A\rangle\) to change by one standard deviation. The Robertson
inequality for \((H,A)\) then gives
\(\Delta E\cdot\Delta t_A\ge\hbar/2\): sharp energy implies slow
evolution of every observable, just as sharp \(L_z\) implies uniform
\(\phi\) (Section 3). An energy eigenstate (\(\Delta E=0\)) has
\(d\langle A\rangle/dt=0\) for all \(A\) --- a completely static state,
the temporal version of the azimuthally uniform \(L_z\) eigenstate.
Conversely, rapid evolution requires a broad energy superposition, just
as angular localization requires many \(m\)-modes (Section 4). For the
coherent state of Example 6.1 below, this bound is saturated:
\(\Delta E\cdot\Delta t_x=\hbar/2\), confirming the coherent state as a
minimum-uncertainty state for both the spatial and temporal versions of
the tradeoff.

\hypertarget{a-second-witness-the-harmonic-oscillator}{%
\section{6. A second witness: the harmonic
oscillator}\label{a-second-witness-the-harmonic-oscillator}}

The same structure appears in the simplest one-dimensional integrable
system.

\texttt{Example\ 6.1\ (Harmonic\ oscillator:\ Fock\ states\ vs\ coherent\ states).}
For a harmonic oscillator of frequency \(\omega\), define the classical
action variable \(J=E/\omega\). The quantum Fock states \(|n\rangle\)
are the action eigenstates (\(J_n=(n+\tfrac12)\hbar\)), and their
phase-space (Husimi) distribution is a ring centered at the origin ---
the orbit phase \(\theta\) is uniformly delocalized, exactly as \(\phi\)
is delocalized in an \(L_z\) eigenstate. Conversely, a coherent state \[
|\alpha\rangle
=e^{-|\alpha|^2/2}\sum_{n=0}^{\infty}\frac{\alpha^n}{\sqrt{n!}}\,|n\rangle,
\qquad \alpha=|\alpha|\,e^{i\theta_0},
\] is the closest quantum analog of a classical orbit with definite
amplitude \(|\alpha|\) and phase \(\theta_0\). Its Fock-state weights
follow a Poisson distribution with mean \(\bar n=|\alpha|^2\), so
localizing the phase to width \(\Delta\theta\sim 1/|\alpha|\) requires
spreading the action over \(\Delta n\sim |\alpha|\) modes. The tradeoff
is the same as in Section 4: sharp action implies delocalized phase, and
vice versa.

\texttt{Example\ 6.2\ (Hydrogen\ atom:\ three\ action–angle\ pairs).} In
the hydrogen atom, the \(n^2\)-fold degeneracy (\(E_n\) depending only
on the principal quantum number \(n\)) reflects an enhanced \(SO(4)\)
symmetry {[}Sakurai2020{]}. Semiclassically, the bound orbits lie on a
three-torus with action integrals quantized by \((n,\ell,m)\). A
stationary eigenstate \(|n,\ell,m\rangle\) is sharp in all three actions
and therefore delocalized in all three conjugate angles: the azimuthal
phase \(\phi\) is uniform (Section 3), the in-plane orbit orientation
has no preferred direction (the Runge--Lenz vector has vanishing
expectation value, since it connects states of different \(\ell\)), and
the radial probability \(|R_{n\ell}(r)|^2\) is time-independent --- the
sharp radial action leaves the conjugate radial phase uniformly
delocalized. A classical Keplerian ellipse with definite eccentricity,
orientation, and timing requires a coherent superposition over ranges of
\((n,\ell,m)\), just as a coherent state in Example 6.1 requires
superposing many Fock states.

\texttt{Remark\ 6.3\ (EBK\ quantization\ on\ the\ invariant\ torus).}
For a classically integrable system with \(d\) degrees of freedom, the
Arnold--Liouville theorem provides \(d\) action variables
\(I_k=(2\pi)^{-1}\oint_{\gamma_k}p\cdot dq\), integrated around the
independent cycles \(\gamma_k\) of the invariant \(d\)-torus. The EBK
(Einstein--Brillouin--Keller) quantization condition requires \[
I_k=\left(n_k+\frac{\alpha_k}{4}\right)\hbar,\qquad n_k\in\mathbb{Z}_{\ge0},
\] where \(\alpha_k\) is the Maslov index of the \(k\)-th cycle
(counting caustic/turning-point contributions). The integer quantum
numbers \(n_k\) select the torus; the conjugate angle variables
\(\theta_k\in[0,2\pi)\) are uniformly distributed on that torus and
carry no quantum-number information. This is the semiclassical
counterpart of the fully quantum statement: stationary eigenstates
(sharp actions) have delocalized angles. Examples 6.1 and 6.2 are the
exact quantum versions of this principle for the \(d=1\) and \(d=3\)
cases.

\texttt{Remark\ 6.4\ (Husimi\ function:\ visualizing\ action–angle\ states\ in\ phase\ space).}
The Husimi \(Q\)-function
\(Q(\alpha)=\langle\alpha|\hat\rho|\alpha\rangle/\pi\) assigns a
non-negative quasiprobability to each phase-space point \(\alpha\),
using coherent states as the reference frame. For a Fock state
\(|n\rangle\), \(Q(\alpha)=e^{-|\alpha|^2}|\alpha|^{2n}/(\pi\,n!)\) ---
a ring at radius \(|\alpha|=\sqrt{n}\), uniform in the phase angle:
sharp action, fully delocalized angle. For a coherent state
\(|\alpha_0\rangle\), \(Q(\alpha)=e^{-|\alpha-\alpha_0|^2}/\pi\) --- a
Gaussian blob centered at \(\alpha_0\), simultaneously localizing both
action and angle to uncertainty-limited width. The ring-versus-blob
distinction is the phase-space portrait of the Fourier tradeoff in
Section 4, with the Husimi function providing a literal (non-negative)
probability picture that the Wigner function's sign changes would
obscure.

\texttt{Remark\ 6.5\ (Squeezed\ states:\ continuous\ interpolation\ between\ ring\ and\ blob).}
The Fock ring and coherent blob are not the only options; squeezed
states of the form \(D(\alpha_0)S(\xi)|0\rangle\) (displacement followed
by squeezing) produce an elliptical Husimi contour at \(\alpha_0\), with
the squeeze parameter \(r=|\xi|\) controlling the eccentricity. When the
ellipse is aligned radially (amplitude squeezing), action uncertainty is
reduced at the expense of angle uncertainty --- approaching Fock-ring
character. When aligned tangentially (phase squeezing), angle
uncertainty is reduced at the expense of action uncertainty --- giving a
better approximation to a classical orbit with definite phase. The full
family, parametrized by \(r\) and the squeeze angle, interpolates
continuously between the extremes of Examples 6.1 and 6.4 while
saturating the Heisenberg bound \(\Delta X_1\,\Delta X_2=\hbar/2\) (for
the dimensionless quadratures \(X_1,X_2\) of the oscillator mode) at
every point.

\texttt{Remark\ 6.6\ (Wigner\ function:\ sub-Planck\ structure\ beneath\ the\ Husimi\ portrait).}
The Husimi function is the Wigner function convolved with a Gaussian of
width \(\sqrt{\hbar}\): this smoothing guarantees non-negativity (Remark
6.4) but erases interference fringes. For a Fock state \(|n\rangle\),
the Wigner function
\(W_n(x,p)=((-1)^n/\pi\hbar)\,L_n(2H/\hbar\omega)\,e^{-2H/\hbar\omega}\)
(where \(H=(p^2+\omega^2 x^2)/2\) and \(L_n\) is a Laguerre polynomial)
exhibits alternating-sign rings --- structure finer than the minimal
uncertainty cell \(\Delta x\,\Delta p\sim\hbar\) that the non-negative
Husimi ring completely hides. The Wigner portrait thus complements
Remarks 6.4--6.5: Husimi shows where the state ``is'' in phase space,
while Wigner reveals the quantum coherences that make ``where'' an
incomplete description.

\texttt{Remark\ 6.7\ (Quantum\ state\ tomography:\ reconstructing\ the\ phase-space\ portrait).}
The Husimi and Wigner functions of Remarks 6.4--6.6 are not merely
theoretical constructs: they can be reconstructed from experimental data
via quantum state tomography. For an oscillator mode, measuring the
rotated quadrature \(X_\theta=x\cos\theta+p\sin\theta\) at angle
\(\theta\) yields a marginal distribution \(P_\theta(s)\) that is a
Radon projection of the Wigner function. Collecting \(P_\theta\) for all
\(\theta\in[0,\pi)\) and inverting the Radon transform (filtered
back-projection) recovers the full \(W(x,p)\). This was first
demonstrated by Smithey, Beck, Raymer, and Faridani (1993), who
reconstructed the Wigner function of a squeezed vacuum state using
optical homodyne detection, where the local oscillator phase selects the
measured quadrature angle. The ring-versus-blob distinction of Remark
6.4 is thus directly observable: Fock-state tomograms show the
uniform-phase ring, while coherent-state tomograms show the localized
Gaussian blob --- the action-angle tradeoff of Sections 3--4 made
experimentally visible.

\texttt{Remark\ 6.8\ (Spin\ coherent\ states:\ the\ angular-momentum\ analog\ of\ coherent-state\ localization).}
The harmonic-oscillator coherent states of Example 6.1 have a direct
angular-momentum counterpart. The spin coherent state
\(|j,\hat n\rangle = R(\hat n)|j,j\rangle\) --- obtained by rotating the
highest-weight state to point along \(\hat n=(\theta_0,\phi_0)\)
(Radcliffe, 1971; Perelomov, 1972) --- has \(m\)-coefficients that
follow a binomial distribution with mean \(\bar m=j\cos\theta_0\) and
variance \(\mathrm{Var}(m)=(j/2)\sin^2\theta_0\). The expectation value
of the angular-momentum vector is
\(\langle\mathbf J\rangle=j\hbar\,\hat n\), with angular uncertainty
\(\Delta\theta\sim 1/\sqrt{2j}\) that saturates the SU(2) Robertson
bound \(\Delta J_1\,\Delta J_2\ge(\hbar/2)|\langle J_3\rangle|\). As
\(j\to\infty\), the spin coherent states become delta-concentrated on
the classical phase space \(S^2\) (Lieb, 1973) --- the angular-momentum
analog of the \(|\alpha|\to\infty\) classical limit of Example 6.1. The
ring-versus-blob picture of Remark 6.4 thus extends from the flat
(oscillator) phase space to the sphere: \(|j,m\rangle\) eigenstates are
azimuthal rings with uniform \(\phi\), while spin coherent states are
directional blobs that spread over \(\sim\sqrt{j}\) magnetic sub-levels.

\texttt{Remark\ 6.9\ (Bargmann\ representation:\ action–angle\ duality\ made\ algebraic).}
The Bargmann--Segal representation (Bargmann, 1961) maps the oscillator
Hilbert space to the space of holomorphic functions on \(\mathbb{C}\)
with Gaussian measure \(d\mu=\pi^{-1}e^{-|z|^2}d^2z\). Under this map,
Fock states become monomials --- \(|n\rangle\mapsto z^n/\sqrt{n!}\) ---
whose rotational symmetry \(|z^n|^2=|z|^{2n}\) reflects the uniform
phase distribution of Remark 6.4's ring. Coherent states become
reproducing kernels \(K_\alpha(z)=e^{\bar\alpha z}\), which extract the
value of a holomorphic function at \(\alpha\) --- the algebraic
expression of ``evaluation at a phase-space point,'' i.e., maximal
localization. Creation and annihilation become multiplication and
differentiation: \(\hat a^\dagger\mapsto z\), \(\hat a\mapsto d/dz\).
The Husimi function of Remark 6.4 is then the squared modulus of the
Bargmann representative (weighted by the Gaussian measure), so the
ring-versus-blob distinction of Remarks 6.4--6.8 is literally monomials
versus peaked exponentials. For spin-\(j\) systems (Schwinger's
oscillator construction), the Bargmann space truncates to polynomials of
degree \(\le 2j\), and spin coherent states become evaluation points on
\(\mathbb{CP}^1\cong S^2\) --- the finite-dimensional counterpart of the
full oscillator duality.

\hypertarget{outlook-kept-minimal}{%
\section{7. Outlook (kept minimal)}\label{outlook-kept-minimal}}

The preceding witnesses illustrate the action--angle tradeoff in systems
with one, two, and three action--angle pairs, and Remark 6.3 shows that
EBK quantization makes the same structural point in general: the more
sharply the actions are specified, the less information remains in the
conjugate phases.

\texttt{Remark\ 7.1\ (Boundary\ at\ integrability\ breaking).} The
action--angle framework of Sections 3--6 presupposes the existence of
global action variables, hence applies exactly to integrable systems.
For nearly integrable Hamiltonians, the KAM theorem guarantees
persistence of most invariant tori (those with sufficiently irrational
frequency ratios), on which the framework remains valid. In fully
chaotic systems, the absence of conserved actions replaces the
structured Fourier tradeoff with eigenstate thermalization --- a more
drastic delocalization where individual energy eigenstates appear
thermal for local observables, with no residual action--angle structure
to organize the uncertainty.

\texttt{Remark\ 7.2\ (Spectral\ statistics:\ from\ Poisson\ to\ random-matrix\ universality).}
The action--angle structure of Sections 3--6 also organizes the energy
spectrum. Berry and Tabor (1977) showed that for a generic integrable
system (incommensurable classical frequencies), EBK quantization on
independent tori produces energy levels whose nearest-neighbor spacings
follow a Poisson distribution \(P(s)=e^{-s}\) --- uncorrelated, with no
level repulsion. Bohigas, Giannoni, and Schmit (1984) conjectured (and
verified numerically for the Sinai billiard) that classically chaotic
systems instead display the level repulsion characteristic of random
matrix theory: GOE statistics for time-reversal invariant systems, GUE
when time-reversal is broken. The spectral transition --- Poisson to
Wigner-Dyson --- mirrors the phase-space transition of Remark 7.1:
independent torus quantization gives way to the global eigenstate
entanglement that produces both level repulsion and eigenstate
thermalization.

\texttt{Remark\ 7.3\ (Ehrenfest\ time:\ temporal\ boundary\ of\ the\ classical\ orbit\ picture).}
The Ehrenfest theorem --- \(d\langle x\rangle/dt=\langle p\rangle/m\),
\(d\langle p\rangle/dt=-\langle V'(x)\rangle\) --- gives exact classical
dynamics only when the potential is at most quadratic (so that
\(\langle V'(x)\rangle=V'(\langle x\rangle)\)). For anharmonic systems,
the wavepacket's finite width introduces corrections that grow with
time, and the classical-orbit picture breaks down after the Ehrenfest
time \(t_E\). For integrable systems, the characteristic quantum
timescales --- both the dispersal time and the revival time
\(t_{\mathrm{rev}}\sim 2\pi\hbar/|d^2E/dm^2|\) of Remark 5.1 --- are
algebraic in \(\hbar\). For classically chaotic systems with maximal
Lyapunov exponent \(\lambda\), exponential trajectory divergence
compresses this to \(t_E\sim(1/\lambda)\ln(1/\hbar)\) --- logarithmic in
\(1/\hbar\), hence far shorter (Berman and Zaslavsky, 1978). This
dramatic contrast --- algebraic versus logarithmic --- is the temporal
counterpart of the phase-space and spectral transitions described in
Remarks 7.1 and 7.2: the coherent packets of Section 5 remain classical
for polynomially long times in integrable systems, but only
logarithmically long in chaotic ones.

\texttt{Remark\ 7.4\ (Gutzwiller\ trace\ formula:\ periodic\ orbits\ as\ the\ non-integrable\ spectral\ bridge).}
The EBK quantization of Remark 6.3 connects eigenvalues to classical
tori in integrable systems. For non-integrable (chaotic) systems,
invariant tori do not exist, and the Gutzwiller trace formula (1971)
replaces them with isolated periodic orbits: the oscillatory part of the
density of states is
\(g_{\mathrm{osc}}(E)\approx\sum_\gamma A_\gamma\cos(S_\gamma(E)/\hbar-\mu_\gamma\pi/2)\),
where the sum runs over all periodic orbits \(\gamma\) (primitive and
repeated), \(S_\gamma=\oint_\gamma p\cdot dq\) is the classical action,
\(\mu_\gamma\) is the Maslov index, and the amplitude
\(A_\gamma\propto T_\gamma/|\det(M_\gamma-I)|^{1/2}\) involves the orbit
period \(T_\gamma\) and the stability determinant of the monodromy
matrix \(M_\gamma\) restricted to the transverse directions. For
marginally stable orbits (\(M_\gamma\) has eigenvalue 1), this amplitude
diverges --- these are the torus orbits of integrable systems, and the
trace formula degenerates to EBK quantization. This formula also
underlies the spectral statistics of Remark 7.2: the Poisson statistics
of integrable systems follow from the incommensurate action phases on
independent tori, while the random-matrix (GOE/GUE) statistics of
chaotic systems emerge from the exponentially proliferating periodic
orbits (their number grows as \(\sim e^{hT}/T\) with topological entropy
\(h\)) through semiclassical sum rules (Berry, 1985). The Gutzwiller
formula is generically asymptotic (\(\hbar\to 0\)), but for surfaces of
constant negative curvature, the Selberg trace formula provides an exact
identity between the Laplacian spectrum and closed geodesics.

\texttt{Remark\ 7.5\ (Geometric\ quantization:\ the\ action-angle\ framework\ made\ systematic).}
The action-angle structure underlying Sections 2--6 has a systematic
mathematical home in geometric quantization (Kostant and Souriau, 1970).
Given a symplectic manifold \((M,\omega)\), the prequantization
condition requires \([\omega/(2\pi\hbar)]\) to be an integral cohomology
class --- precisely the EBK quantization condition of Remark 6.3,
expressed as integrality of the Chern class of a line bundle \(L\to M\).
A polarization --- a choice of Lagrangian foliation --- then reduces the
prequantum Hilbert space to the physical one. For integrable systems
with action-angle coordinates \((I,\theta)\), the real (action)
polarization gives wave functions depending on \(I\) alone, yielding the
number/Fock states of Example 6.1 --- the ``rings'' of Remark 6.4 ---
while the complex (Kähler) polarization gives holomorphic wave
functions, recovering the Bargmann representation of Remark 6.9 --- the
``blobs.'' The metaplectic correction (Blattner, Kostant, Sternberg)
accounts for the half-density structure of the cornerstone manuscript,
adding the \(\tfrac12\hbar\omega\) zero-point energy that naive
geometric quantization misses. For spin-\(j\) systems, geometric
quantization of \(S^2\cong\mathbb{CP}^1\) with symplectic form
proportional to the area form produces the spin coherent states of
Remark 6.8, with the prequantization condition forcing
\(2j\in\mathbb{Z}\) --- the integrality of spin as a topological
constraint.

\hypertarget{references}{%
\section{References}\label{references}}

\begin{enumerate}
\def\labelenumi{\arabic{enumi}.}
\tightlist
\item
  {[}TongQMLectures{]} David Tong, ``Quantum Mechanics'' (lecture notes,
  no DOI). OA: lecture-note PDF. (Contains
  \(Y_{l,m}(\theta,\phi)=P_{l,m}(\cos\theta)e^{im\phi}\) as simultaneous
  eigenstates of \(L^2\) and \(L_z\).)
\item
  {[}Sakurai2020{]} J. J. Sakurai and Jim Napolitano, \emph{Modern
  Quantum Mechanics}, 3rd ed., Cambridge University Press, 2020. ISBN
  \texttt{978-1-108-47322-4}. (Standard treatment of angular momentum,
  spherical harmonics, and quantum measurement.)
\item
  {[}ZurekHabibPaz1993{]} W. H. Zurek, S. Habib, and J. P. Paz,
  ``Coherent States via Decoherence,'' \emph{Physical Review Letters} 70
  (1993), 1187--1190. DOI \texttt{10.1103/PhysRevLett.70.1187}. (Shows
  coherent states minimize entropy production under environmental
  coupling, emerging as preferred pointer states.)
\item
  {[}Bargmann1961{]} Valentine Bargmann, ``On a Hilbert space of
  analytic functions and an associated integral transform, Part I,''
  \emph{Communications on Pure and Applied Mathematics} 14 (1961),
  187--214. DOI \texttt{10.1002/cpa.3160140303}. (Bargmann--Segal
  holomorphic representation; used in Remark 6.9.)
\item
  {[}Perelomov1972{]} A. M. Perelomov, ``Coherent states for arbitrary
  Lie group,'' \emph{Communications in Mathematical Physics} 26 (1972),
  222--236. DOI \texttt{10.1007/BF01645091}. (Generalized coherent
  states including spin coherent states; used in Remark 6.8.)
\item
  {[}MandelstamTamm1945{]} L. I. Mandelstam and I. E. Tamm, ``The
  uncertainty relation between energy and time in non-relativistic
  quantum mechanics,'' \emph{Journal of Physics (USSR)} 9 (1945),
  249--254. (Energy--time uncertainty relation; used in Remark 5.3.)
\item
  {[}BerryTabor1977{]} M. V. Berry and M. Tabor, ``Level clustering in
  the regular spectrum,'' \emph{Proceedings of the Royal Society A} 356
  (1977), 375--394. DOI \texttt{10.1098/rspa.1977.0140}. (Poisson
  statistics for integrable systems; used in Remark 7.2.)
\end{enumerate}

\end{document}
