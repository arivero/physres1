% Options for packages loaded elsewhere
\PassOptionsToPackage{unicode}{hyperref}
\PassOptionsToPackage{hyphens}{url}
\documentclass[
]{article}
\usepackage{xcolor}
\usepackage{amsmath,amssymb}
\setcounter{secnumdepth}{-\maxdimen} % remove section numbering
\usepackage{iftex}
\ifPDFTeX
  \usepackage[T1]{fontenc}
  \usepackage[utf8]{inputenc}
  \usepackage{textcomp} % provide euro and other symbols
\else % if luatex or xetex
  \usepackage{unicode-math} % this also loads fontspec
  \defaultfontfeatures{Scale=MatchLowercase}
  \defaultfontfeatures[\rmfamily]{Ligatures=TeX,Scale=1}
\fi
\usepackage{lmodern}
\ifPDFTeX\else
  % xetex/luatex font selection
\fi
% Use upquote if available, for straight quotes in verbatim environments
\IfFileExists{upquote.sty}{\usepackage{upquote}}{}
\IfFileExists{microtype.sty}{% use microtype if available
  \usepackage[]{microtype}
  \UseMicrotypeSet[protrusion]{basicmath} % disable protrusion for tt fonts
}{}
\makeatletter
\@ifundefined{KOMAClassName}{% if non-KOMA class
  \IfFileExists{parskip.sty}{%
    \usepackage{parskip}
  }{% else
    \setlength{\parindent}{0pt}
    \setlength{\parskip}{6pt plus 2pt minus 1pt}}
}{% if KOMA class
  \KOMAoptions{parskip=half}}
\makeatother
\setlength{\emergencystretch}{3em} % prevent overfull lines
\providecommand{\tightlist}{%
  \setlength{\itemsep}{0pt}\setlength{\parskip}{0pt}}
\usepackage{bookmark}
\IfFileExists{xurl.sty}{\usepackage{xurl}}{} % add URL line breaks if available
\urlstyle{same}
\hypersetup{
  hidelinks,
  pdfcreator={LaTeX via pandoc}}

\author{}
\date{}

\begin{document}

\section{Planck Area from Half-Density Normalization
(Draft)}\label{planck-area-from-half-density-normalization-draft}

\subsection{Abstract}\label{abstract}

Half-densities are the natural ``coordinate-free integrands'' for
composing kernels without choosing a background measure. But a
half-density still carries physical units: in \(d\) dimensions its
normalization involves a scale with units of \(\text{length}^{d/2}\). In
\(d=4\), this is an \emph{area}. This note develops a programmatic
argument that the need to normalize composition kernels as
half-densities forces the introduction of a universal area scale, and
that identifying this scale with the Planck area is both natural and, in
certain Newtonian/gravitational settings, reciprocally recoverable from
a minimal-areal-speed principle {[}RiveroAreal{]} {[}RiveroSimple{]}.

\subsection{1. Purpose and Status}\label{purpose-and-status}

This is a dependent follow-up to \texttt{paper/main.md}. It is not yet a
finished paper; its goal is to isolate one technical point that is only
implicit in the main manuscript: the role of half-densities (and their
scaling) in making composition laws coordinate-invariant \emph{and}
dimensionally well-defined.

Claims below are labeled as \texttt{Proposition} (math-precise under
hypotheses) or \texttt{Heuristic} (programmatic bridge).

\subsection{2. Half-Densities and Composition
Kernels}\label{half-densities-and-composition-kernels}

Let \(M\) be a \(d\)-dimensional manifold. A (positive) density is a
section of \(|\Lambda^d T^\ast M|\), and a half-density is a section of
\(|\Lambda^d T^\ast M|^{1/2}\).

The key operational point is: when a kernel is a half-density in its
integration variable, composition of kernels does not depend on an
arbitrary choice of coordinate measure.

\texttt{Heuristic\ H1.1\ (Why\ half-densities).} If \(K_1(x,y)\) and
\(K_2(y,z)\) are chosen so that their product in the intermediate
variable \(y\) is a density, then \(\int_M K_1(x,y)K_2(y,z)\) is
coordinate-invariant without fixing a preferred \(dy\). This matches the
structural role of kernel composition used in \texttt{paper/main.md}
(Section 6).

\texttt{Derivation\ D1.1\ (Coordinate\ invariance\ of\ half-density\ pairing\ and\ composition).}
In a local chart \(y=(y^1,\ldots,y^d)\), write a half-density as
\(\psi(y)=\varphi(y)\,|dy|^{1/2}\). Under a change of variables
\(y=y(y')\), one has
\(|dy|^{1/2}=|\det(\partial y/\partial y')|^{1/2}|dy'|^{1/2}\), so the
coefficient transforms as
\(\varphi'(y')=\varphi(y(y'))\,|\det(\partial y/\partial y')|^{1/2}\).

Hence the product of two half-densities is a density:
\(\psi_1\psi_2=(\varphi_1\varphi_2)\,|dy|\), and its integral is
chart-independent: \(\int_M \psi_1\psi_2\) is well-defined without
choosing a background measure beyond the density bundle itself.

Kernel composition is the same mechanism: if \(K_1(x,y)\) and
\(K_2(y,z)\) are half-densities in \(y\), then \(K_1K_2\) is a density
in \(y\) and \(\int_M K_1K_2\) is coordinate invariant.

\subsection{3. Dimensional Analysis: Normalizing a Half-Density Requires
a
Scale}\label{dimensional-analysis-normalizing-a-half-density-requires-a-scale}

A density on \(M\) carries the units of \(\text{length}^d\) once
physical units are assigned to coordinates. A half-density therefore
carries units \(\text{length}^{d/2}\).

\texttt{Proposition\ P1.1\ (Scale\ required\ for\ numerical\ normalization).}
Any attempt to map a half-density \(\psi\in|\Lambda^d T^\ast M|^{1/2}\)
to a dimensionless numerical amplitude requires choosing a reference
scale \(\ell_\ast\) with units \(\text{length}\) (equivalently
\(\ell_\ast^{d/2}\) with units \(\text{length}^{d/2}\)) to fix
normalization conventions.

In \(d=4\), \(\ell_\ast^{d/2}=\ell_\ast^2\) is an \emph{area}. Thus, in
four dimensions, half-density normalization is naturally controlled by a
fundamental area scale.

\texttt{Derivation\ D1.2\ (Dilation\ makes\ the\ \textbackslash{}(\textbackslash{}text\{length\}\^{}\{d/2\}\textbackslash{})\ scale\ explicit).}
On \(\mathbb R^d\), consider a dilation \(y\mapsto y'=a y\) with
\(a>0\). Then \(|dy'|=a^d|dy|\), so \(|dy'|^{1/2}=a^{d/2}|dy|^{1/2}\).
Thus even in flat space, half-density normalization is inherently tied
to a \(\text{length}^{d/2}\) scaling weight. In \(d=4\), the ``scale you
must insert to make half-densities numerically comparable'' naturally
carries units of area.

\texttt{Heuristic\ H1.2\ (Reciprocity\ claim).} If one accepts
``composition kernels live as half-densities'' as the right invariant
setup for quantum amplitudes, and also insists on a \emph{universal}
normalization convention (no background structures), then a universal
area scale is forced in \(d=4\). A natural identification is the Planck
area \(L_P^2\).

\subsection{4. Stationary Phase Produces Half-Density Prefactors
(Short-Time
Kernel)}\label{stationary-phase-produces-half-density-prefactors-short-time-kernel}

The main manuscript uses stationary phase to explain why classical
extremals dominate refinement limits. Here we add the complementary
kernel-level fact: stationary phase does not only pick the extremal; it
also produces a determinant prefactor that transforms as a half-density,
i.e.~the object needed for coordinate-free kernel composition.

\texttt{Derivation\ D1.4\ (Van\ Vleck\ prefactor\ is\ a\ bi-half-density).}
Let \(S_{\mathrm{cl}}(x,z;t)\) be the classical action as a function of
endpoints and time, treated as a generating function. The standard
short-time/stationary-phase approximation to the propagator has the form

\[
K(x,z;t)
\approx
\frac{1}{(2\pi i\hbar)^{d/2}}
\left|\det\!\left(-\frac{\partial^2 S_{\mathrm{cl}}}{\partial x\,\partial z}\right)\right|^{1/2}
\exp\!\left(\frac{i}{\hbar}S_{\mathrm{cl}}(x,z;t)\right).
\]

Under a change of coordinates \(x=x(x')\), \(z=z(z')\), the mixed
Hessian transforms by the chain rule, and its determinant acquires
Jacobian factors:

\[
\det\!\left(-\frac{\partial^2 S_{\mathrm{cl}}}{\partial x'\,\partial z'}\right)
=
\det\!\left(\frac{\partial x}{\partial x'}\right)
\det\!\left(\frac{\partial z}{\partial z'}\right)
\det\!\left(-\frac{\partial^2 S_{\mathrm{cl}}}{\partial x\,\partial z}\right).
\]

Taking square roots shows that the prefactor transforms with
\(|\det(\partial x/\partial x')|^{1/2}|\det(\partial z/\partial z')|^{1/2}\),
i.e.~exactly as a half-density factor at each endpoint. Thus the
stationary-phase prefactor is naturally interpreted as making \(K\) a
half-density in each variable, so that kernel composition does not
depend on a background measure choice.

\texttt{Proposition\ P1.2\ (Reference\ half-density\ fixes\ normalization\ conventions).}
Given a chart \(x\) on \(M\), any nowhere-vanishing reference
half-density can be written as
\(\sigma_\ast(x)=\ell_\ast^{d/2}\,|dx|^{1/2}\), where \(\ell_\ast\) is a
chosen length scale. A half-density \(\psi(x)=\varphi(x)\,|dx|^{1/2}\)
is then converted into a scalar coefficient by
\(\varphi_\ast(x)=\psi(x)/\sigma_\ast(x)\). Therefore a universal
convention for turning half-densities into scalar amplitudes requires
choosing a universal \(\ell_\ast^{d/2}\) scale.

In \(d=4\), \(\ell_\ast^{d/2}=\ell_\ast^2\) is an area, so a universal
normalization convention for half-densities in four dimensions is
equivalent to choosing a universal area scale.

\texttt{Heuristic\ H1.4\ (Where\ Planck\ area\ can\ enter,\ minimally).}
If one further insists that \(\ell_\ast\) be fixed by universal
constants rather than background geometric data, then in a relativistic
setting the only available \(4\)D area scale built from \((\hbar,c,G)\)
is the Planck area \(L_P^2\). The claim pursued in this follow-up draft
is that this is not merely dimensional bookkeeping: it interacts with
refinement/composition in a way that can be physically anchored (Section
5).

\subsection{\texorpdfstring{5. A Gravitational Anchor: Minimal Areal
Speed and the \(D=4\)
Cancellation}{5. A Gravitational Anchor: Minimal Areal Speed and the D=4 Cancellation}}\label{a-gravitational-anchor-minimal-areal-speed-and-the-d4-cancellation}

Rivero's ``Planck areal speed'' observation gives a concrete route by
which Planck-scale discreteness reappears at Compton scales in
inverse-square gravity {[}RiveroAreal{]} {[}RiveroSimple{]}.

\texttt{Heuristic\ H1.3\ (Areal-speed\ selection).} In \(3+1\) Newtonian
gravity (inverse-square), imposing a discrete areal-speed/area-time
condition at a Planck scale can yield characteristic radii proportional
to a reduced Compton length, with Newton's constant canceling when
expressed in Planck units. This is a nontrivial indication that ``a
universal area scale'' can be operationally meaningful at low energies
in \(D=4\).

\texttt{Derivation\ D1.3\ (Inverse-square\ circular\ orbit\ +\ Planck\ areal\ speed\ \textbackslash{}(\textbackslash{}Rightarrow\textbackslash{})\ Compton\ radius).}
For a circular orbit under an inverse-square central force
\(F(r)=K/r^2\) (with coupling \(K>0\)), the centripetal balance is
\(m v^2/r = K/r^2\). The areal speed is \(\dot A = \tfrac12 r v\), so
\(v = 2\dot A/r\). Substituting into the force balance gives

\[
m\left(\frac{2\dot A}{r}\right)^2=\frac{K}{r}
\quad\Longrightarrow\quad
r=\frac{4m\,\dot A^2}{K}.
\]

For Newtonian gravity between a source mass \(M\) and test mass \(m\),
\(K=GMm\), hence

\[
r=\frac{4\dot A^2}{GM},
\]

independent of the test mass \(m\). If one now imposes
\(\dot A = k\,\dot A_P\), where Rivero's Planck areal speed is
\(\dot A_P = cL_P\) {[}RiveroAreal{]}, then using \(L_P^2 = G\hbar/c^3\)
yields

\[
r
=\frac{4k^2(cL_P)^2}{GM}
=\frac{4k^2(G\hbar/c)}{GM}
=4k^2\,\frac{\hbar}{cM}.
\]

Thus \(r\) becomes a multiple of the reduced Compton length
\(L_M=\hbar/(cM)\), with Newton's constant canceled out. In particular,
\(k=\tfrac12\) gives \(r=L_M\). This is the ``Planck area per Planck
time \(\Rightarrow\) Compton scale'' cancellation highlighted in
{[}RiveroAreal{]} and summarized in {[}RiveroSimple{]}.

\subsection{6. Interface with the Main
Paper}\label{interface-with-the-main-paper}

The main manuscript argues that: 1. classical dynamics are recovered
from quantum composition by stationary-phase concentration, and 2.
refinement across scales forces RG-style consistency conditions when
naive limits diverge.

This draft adds a complementary ingredient: the kernel side is most
naturally formulated in half-density language, and stationary phase
produces the bi-half-density prefactor directly. A universal convention
for turning those half-densities into scalar amplitudes then requires a
\(\text{length}^{d/2}\) scale; in \(d=4\) this is an area scale.

\subsection{7. Open Problems (Needed for a Real
Paper)}\label{open-problems-needed-for-a-real-paper}

\begin{enumerate}
\def\labelenumi{\arabic{enumi}.}
\tightlist
\item
  Make the half-density normalization argument precise for a concrete
  groupoid or kernel model (tangent-groupoid or short-time propagator
  model).
\item
  Show how the area scale enters stationary-phase prefactors and how
  this interacts with RG scaling.
\item
  General-dimension analysis: clarify what replaces ``area'' in odd
  dimensions and whether a universal normalization is still defensible.
\item
  Identify minimal hypotheses under which ``need of half-density scale
  \(\Rightarrow\) Planck area'' is more than dimensional bookkeeping.
\end{enumerate}

\end{document}
