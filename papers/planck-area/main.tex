% Options for packages loaded elsewhere
\PassOptionsToPackage{unicode}{hyperref}
\PassOptionsToPackage{hyphens}{url}
\documentclass[
]{article}
\usepackage{xcolor}
\usepackage{amsmath,amssymb}
\setcounter{secnumdepth}{-\maxdimen} % remove section numbering
\usepackage{iftex}
\ifPDFTeX
  \usepackage[T1]{fontenc}
  \usepackage[utf8]{inputenc}
  \usepackage{textcomp} % provide euro and other symbols
\else % if luatex or xetex
  \usepackage{unicode-math} % this also loads fontspec
  \defaultfontfeatures{Scale=MatchLowercase}
  \defaultfontfeatures[\rmfamily]{Ligatures=TeX,Scale=1}
\fi
\usepackage{lmodern}
\ifPDFTeX\else
  % xetex/luatex font selection
\fi
% Use upquote if available, for straight quotes in verbatim environments
\IfFileExists{upquote.sty}{\usepackage{upquote}}{}
\IfFileExists{microtype.sty}{% use microtype if available
  \usepackage[]{microtype}
  \UseMicrotypeSet[protrusion]{basicmath} % disable protrusion for tt fonts
}{}
\makeatletter
\@ifundefined{KOMAClassName}{% if non-KOMA class
  \IfFileExists{parskip.sty}{%
    \usepackage{parskip}
  }{% else
    \setlength{\parindent}{0pt}
    \setlength{\parskip}{6pt plus 2pt minus 1pt}}
}{% if KOMA class
  \KOMAoptions{parskip=half}}
\makeatother
\setlength{\emergencystretch}{3em} % prevent overfull lines
\providecommand{\tightlist}{%
  \setlength{\itemsep}{0pt}\setlength{\parskip}{0pt}}
\usepackage{bookmark}
\IfFileExists{xurl.sty}{\usepackage{xurl}}{} % add URL line breaks if available
\urlstyle{same}
\hypersetup{
  hidelinks,
  pdfcreator={LaTeX via pandoc}}

\author{}
\date{}

\begin{document}

\section{Planck Area from Half-Density Normalization
(Draft)}\label{planck-area-from-half-density-normalization-draft}

\subsection{Abstract}\label{abstract}

Half-densities are the natural ``coordinate-free integrands'' for
composing kernels without choosing a background measure. But choosing a
\emph{universal} convention for turning half-density objects into
dimensionless numerical amplitudes introduces a \(\text{length}^{d/2}\)
scale. In \(d=4\), this is an \emph{area}. This note sharpens the
hypothesis ladder needed for the claim ``half-density normalization
selects a universal area scale'', and isolates a simple
dimension-matching condition under which the Planck area appears without
fractional powers of couplings. A gravitational anchor based on a
minimal-areal-speed principle is recorded as a separate heuristic thread
{[}RiveroAreal{]} {[}RiveroSimple{]}.

\subsection{1. Purpose and Status}\label{purpose-and-status}

This is a dependent follow-up to \texttt{paper/main.md}. It is not yet a
finished paper; its goal is to isolate one technical point that is only
implicit in the main manuscript: the role of half-densities (and their
scaling) in making composition laws coordinate-invariant \emph{and}
dimensionally well-defined.

Claims below are labeled as \texttt{Proposition} (math-precise under
hypotheses) or \texttt{Heuristic} (programmatic bridge).

\subsection{2. Half-Densities and Composition
Kernels}\label{half-densities-and-composition-kernels}

Let \(M\) be a \(d\)-dimensional manifold. A (positive) density is a
section of \(|\Lambda^d T^\ast M|\), and a half-density is a section of
\(|\Lambda^d T^\ast M|^{1/2}\).

The key operational point is: when a kernel is a half-density in its
integration variable, composition of kernels does not depend on an
arbitrary choice of coordinate measure.

\texttt{Heuristic\ H1.1\ (Why\ half-densities).} If \(K_1(x,y)\) and
\(K_2(y,z)\) are chosen so that their product in the intermediate
variable \(y\) is a density, then \(\int_M K_1(x,y)K_2(y,z)\) is
coordinate-invariant without fixing a preferred \(dy\). This matches the
structural role of kernel composition used in \texttt{paper/main.md}
(Section 6).

\texttt{Derivation\ D1.1\ (Coordinate\ invariance\ of\ half-density\ pairing\ and\ composition).}
In a local chart \(y=(y^1,\ldots,y^d)\), write a half-density as
\(\psi(y)=\varphi(y)\,|dy|^{1/2}\). Under a change of variables
\(y=y(y')\), one has
\(|dy|^{1/2}=|\det(\partial y/\partial y')|^{1/2}|dy'|^{1/2}\), so the
coefficient transforms as
\(\varphi'(y')=\varphi(y(y'))\,|\det(\partial y/\partial y')|^{1/2}\).

Hence the product of two half-densities is a density:
\(\psi_1\psi_2=(\varphi_1\varphi_2)\,|dy|\), and its integral is
chart-independent: \(\int_M \psi_1\psi_2\) is well-defined without
choosing a background measure beyond the density bundle itself.

Kernel composition is the same mechanism: if \(K_1(x,y)\) and
\(K_2(y,z)\) are half-densities in \(y\), then \(K_1K_2\) is a density
in \(y\) and \(\int_M K_1K_2\) is coordinate invariant.

\subsection{3. Dimensional Analysis: Normalizing a Half-Density Requires
a
Scale}\label{dimensional-analysis-normalizing-a-half-density-requires-a-scale}

A density on \(M\) carries the units of \(\text{length}^d\) once
physical units are assigned to coordinates. A half-density therefore
carries units \(\text{length}^{d/2}\).

\texttt{Proposition\ P1.1\ (No\ canonical\ “half-density\ =\ function”\ identification).}
There is no canonical identification of a half-density
\(\psi\in|\Lambda^d T^\ast M|^{1/2}\) with an ordinary scalar function
\(f\) on \(M\). Choosing such an identification is equivalent to
choosing a nowhere-vanishing reference half-density \(\sigma_\ast\)
(equivalently a positive density \(\rho_\ast=\sigma_\ast^2\)) and
writing \(\psi=f\,\sigma_\ast\).

\texttt{Derivation\ D1.2\ (Dilation\ makes\ the\ \textbackslash{}(\textbackslash{}text\{length\}\^{}\{d/2\}\textbackslash{})\ weight\ explicit).}
On \(\mathbb R^d\), consider a dilation \(y\mapsto y'=a y\) with
\(a>0\). Then \(|dy'|=a^d|dy|\), so \(|dy'|^{1/2}=a^{d/2}|dy|^{1/2}\).
Thus even in flat space, half-densities carry an inherent
\(\text{length}^{d/2}\) scaling weight.

\texttt{Proposition\ P1.2\ (Universal\ *dimensionless*\ amplitudes\ force\ a\ \textbackslash{}(\textbackslash{}text\{length\}\^{}\{d/2\}\textbackslash{})\ constant).}
If one imposes the extra requirement that the scalar representative
\(f\) in \(\psi=f\,\sigma_\ast\) be dimensionless in physical units,
then the reference half-density \(\sigma_\ast\) must carry all of the
\(\text{length}^{d/2}\) dimension. In particular, a \emph{constant}
(field-independent) choice of \(\sigma_\ast\) is equivalent to choosing
a universal \(\text{length}^{d/2}\) scale.

In \(d=4\), this universal \(\text{length}^{d/2}\) scale is a universal
\emph{area} scale.

\texttt{Heuristic\ H1.2\ (Reciprocity\ claim).} Half-densities alone do
not force a particular scale: the forced fact is that converting
half-density objects into scalar numerical amplitudes requires extra
structure (a reference half-density). The ``universal area scale'' claim
begins only after adding two further hypotheses: 1. the reference
\(\sigma_\ast\) is taken to be \emph{constant} (no dependence on
background metric/fields), and 2. the constant is required to be fixed
by universal constants/couplings of the theory.

Under these hypotheses, \(d=4\) is the unique dimension in which the
needed \(\text{length}^{d/2}\) constant can be supplied by the
gravitational coupling without fractional powers (Derivation D1.3).

\texttt{Derivation\ D1.3\ (Dimension\ match:\ why\ \textbackslash{}(d=4\textbackslash{})\ is\ singled\ out\ by\ gravity).}
In \(d\) spacetime dimensions, the Einstein--Hilbert action
\(\frac{1}{16\pi G_d}\int d^d x\,\sqrt{|g|}\,R\) shows that (in
\(c=\hbar=1\) units) Newton's constant has dimension
\([G_d]=\text{length}^{d-2}\). If the universal half-density
normalization constant is required to be built from \(G_d\) without
fractional powers, then its dimension must match \(\text{length}^{d/2}\)
with exponent \(1\), i.e. \(\text{length}^{d/2}=\text{length}^{d-2}\),
which holds if and only if \(d=4\). In that case \(G_4\) itself has
dimension of area, and the corresponding area scale is the Planck area
\(L_P^2\sim \hbar G_4/c^3\).

\subsubsection{3.1 Hypotheses as Separate Knobs (What Is Forced vs
Chosen)}\label{hypotheses-as-separate-knobs-what-is-forced-vs-chosen}

The discussion above mixes three different kinds of statements: 1.
\textbf{Geometric facts} (what half-densities are, how they compose, how
they scale), 2. \textbf{Representational choices} (how one turns
half-density objects into scalar numbers), 3.
\textbf{Universality/selection principles} (what choices are allowed if
we demand ``background-free'' and ``built from couplings'').

To study these separately, it is useful to keep the hypotheses explicit.

\texttt{Hypothesis\ H2.1\ (Half-density\ formulation).} Quantum kernels
are treated as bi-half-densities so that composition in intermediate
variables is coordinate invariant (Section 2 and Derivation D1.4).

\texttt{Hypothesis\ H2.2\ (Scalarization\ by\ a\ reference\ half-density).}
To interpret half-density amplitudes as scalar numerical functions, we
pick a nowhere-vanishing reference half-density \(\sigma_\ast\) and
write \(\psi=f\,\sigma_\ast\) (Proposition P1.1).

\texttt{Hypothesis\ H2.3\ (Dimensionless\ scalar\ representative).} The
scalar representative \(f\) is required to be dimensionless in physical
units (Proposition P1.2). This forces \(\sigma_\ast\) to carry the full
\(\text{length}^{d/2}\) weight.

\texttt{Hypothesis\ H2.4\ (Background-free\ constancy).} The reference
\(\sigma_\ast\) is taken to be constant/field-independent, rather than
determined by background geometry (e.g.~a Riemannian volume
\(|g|^{1/4}|dx|^{1/2}\)) or by dynamical fields (e.g.~a dilaton-like
factor). This is the first point where a \emph{universal constant}
enters.

\texttt{Hypothesis\ H2.5\ (Analyticity\ /\ no\ fractional\ powers\ of\ couplings).}
If the universal constant is required to be built from the theory's
couplings without fractional powers, then dimensional analysis becomes a
\emph{dimension sieve} rather than a tautology. Derivation D1.3 is the
gravity instance: ``use \(G_d\) without fractional powers'' singles out
\(d=4\).

\texttt{Heuristic\ H2.6\ (Where\ “special\ dimensions”\ can\ appear).}
Special dimensions do not come from half-densities alone (Hypothesis
H2.1). They appear only after adding a selection principle like
H2.4--H2.5: the requirement that the scalarization choice be universal,
background-free, and coupling-built in a restricted (e.g.~analytic) way.

\subsubsection{3.2 What Changes When a Hypothesis Is
Relaxed?}\label{what-changes-when-a-hypothesis-is-relaxed}

This subsection records the main ``branches'' that need separate study.

\begin{enumerate}
\def\labelenumi{\arabic{enumi}.}
\tightlist
\item
  \textbf{Drop H2.3 (allow dimensionful \(f\)).} Then no universal
  \(\text{length}^{d/2}\) constant is forced; the dimensional weight can
  be carried by the scalar representative itself (as in the usual
  statement ``wavefunctions have dimension \(\text{length}^{-d/2}\)'').
\item
  \textbf{Drop H2.4 (allow background geometry).} Then \(\sigma_\ast\)
  can be chosen from a metric (or other structure), and the ``universal
  constant'' is replaced by background-dependent normalization.
\item
  \textbf{Drop H2.5 (allow fractional powers).} Then in any \(d>2\) one
  can build a \(\text{length}^{d/2}\) constant from gravity via
  \(G_d^{\,d/(2(d-2))}\) (in \(c=\hbar=1\) units), so \(d=4\) is no
  longer singled out; instead, \(d=4\) is simply the unique case where
  the exponent is an integer.
\item
  \textbf{Change ``which coupling supplies the scale''.} Using other
  dimensionful couplings (cosmological constant, string tension, gauge
  couplings in various dimensions, etc.) yields different
  ``special-dimension'' sieves. This is conceptually aligned with the
  observation that some dimensions are singled out by other structures
  (division algebras, special holonomy, supersymmetry), but those
  filters are separate from the half-density story and should not be
  conflated.
\end{enumerate}

\subsection{4. Stationary Phase Produces Half-Density Prefactors
(Short-Time
Kernel)}\label{stationary-phase-produces-half-density-prefactors-short-time-kernel}

The main manuscript uses stationary phase to explain why classical
extremals dominate refinement limits. Here we add the complementary
kernel-level fact: stationary phase does not only pick the extremal; it
also produces a determinant prefactor that transforms as a half-density,
i.e.~the object needed for coordinate-free kernel composition.

\texttt{Derivation\ D1.4\ (Van\ Vleck\ prefactor\ is\ a\ bi-half-density).}
Let \(S_{\mathrm{cl}}(x,z;t)\) be the classical action as a function of
endpoints and time, treated as a generating function. The standard
short-time/stationary-phase approximation to the propagator has the form

\[
K(x,z;t)
\approx
\frac{1}{(2\pi i\hbar)^{d/2}}
\left|\det\!\left(-\frac{\partial^2 S_{\mathrm{cl}}}{\partial x\,\partial z}\right)\right|^{1/2}
\exp\!\left(\frac{i}{\hbar}S_{\mathrm{cl}}(x,z;t)\right).
\]

Under a change of coordinates \(x=x(x')\), \(z=z(z')\), the mixed
Hessian transforms by the chain rule, and its determinant acquires
Jacobian factors:

\[
\det\!\left(-\frac{\partial^2 S_{\mathrm{cl}}}{\partial x'\,\partial z'}\right)
=
\det\!\left(\frac{\partial x}{\partial x'}\right)
\det\!\left(\frac{\partial z}{\partial z'}\right)
\det\!\left(-\frac{\partial^2 S_{\mathrm{cl}}}{\partial x\,\partial z}\right).
\]

Taking square roots shows that the prefactor transforms with
\(|\det(\partial x/\partial x')|^{1/2}|\det(\partial z/\partial z')|^{1/2}\),
i.e.~exactly as a half-density factor at each endpoint. Thus the
stationary-phase prefactor is naturally interpreted as making \(K\) a
half-density in each variable, so that kernel composition does not
depend on a background measure choice.

\texttt{Heuristic\ H1.4\ (Where\ Planck\ area\ can\ enter,\ minimally).}
Derivation D1.3 isolates one minimal route by which a Planck-scale
quantity can enter: if the theory supplies a single universal coupling
with dimension of length (Newton's constant) and one demands that the
half-density normalization constant be built from that coupling
\emph{without fractional powers}, then \(d=4\) is singled out and the
resulting constant has the dimension of an area, naturally identified
with the Planck area \(L_P^2\sim \hbar G_4/c^3\).

\subsection{\texorpdfstring{5. A Gravitational Anchor: Minimal Areal
Speed and the \(D=4\)
Cancellation}{5. A Gravitational Anchor: Minimal Areal Speed and the D=4 Cancellation}}\label{a-gravitational-anchor-minimal-areal-speed-and-the-d4-cancellation}

Rivero's ``Planck areal speed'' observation gives a concrete route by
which Planck-scale discreteness reappears at Compton scales in
inverse-square gravity {[}RiveroAreal{]} {[}RiveroSimple{]}.

\texttt{Heuristic\ H1.3\ (Areal-speed\ selection).} In \(3+1\) Newtonian
gravity (inverse-square), imposing a discrete areal-speed/area-time
condition at a Planck scale can yield characteristic radii proportional
to a reduced Compton length, with Newton's constant canceling when
expressed in Planck units. This is a nontrivial indication that ``a
universal area scale'' can be operationally meaningful at low energies
in \(D=4\).

\texttt{Derivation\ D1.5\ (Inverse-square\ circular\ orbit\ +\ Planck\ areal\ speed\ \textbackslash{}(\textbackslash{}Rightarrow\textbackslash{})\ Compton\ radius).}
For a circular orbit under an inverse-square central force
\(F(r)=K/r^2\) (with coupling \(K>0\)), the centripetal balance is
\(m v^2/r = K/r^2\). The areal speed is \(\dot A = \tfrac12 r v\), so
\(v = 2\dot A/r\). Substituting into the force balance gives

\[
m\left(\frac{2\dot A}{r}\right)^2=\frac{K}{r}
\quad\Longrightarrow\quad
r=\frac{4m\,\dot A^2}{K}.
\]

For Newtonian gravity between a source mass \(M\) and test mass \(m\),
\(K=GMm\), hence

\[
r=\frac{4\dot A^2}{GM},
\]

independent of the test mass \(m\). If one now imposes
\(\dot A = k\,\dot A_P\), where Rivero's Planck areal speed is
\(\dot A_P = cL_P\) {[}RiveroAreal{]}, then using \(L_P^2 = G\hbar/c^3\)
yields

\[
r
=\frac{4k^2(cL_P)^2}{GM}
=\frac{4k^2(G\hbar/c)}{GM}
=4k^2\,\frac{\hbar}{cM}.
\]

Thus \(r\) becomes a multiple of the reduced Compton length
\(L_M=\hbar/(cM)\), with Newton's constant canceled out. In particular,
\(k=\tfrac12\) gives \(r=L_M\). This is the ``Planck area per Planck
time \(\Rightarrow\) Compton scale'' cancellation highlighted in
{[}RiveroAreal{]} and summarized in {[}RiveroSimple{]}.

\subsection{6. Interface with the Main
Paper}\label{interface-with-the-main-paper}

The main manuscript argues that: 1. classical dynamics are recovered
from quantum composition by stationary-phase concentration, and 2.
refinement across scales forces RG-style consistency conditions when
naive limits diverge.

This draft adds a complementary ingredient: the kernel side is most
naturally formulated in half-density language, and stationary phase
produces the bi-half-density prefactor directly. A universal convention
for turning those half-densities into scalar amplitudes then requires a
\(\text{length}^{d/2}\) scale; in \(d=4\) this is an area scale.

\subsection{7. Open Problems (Needed for a Real
Paper)}\label{open-problems-needed-for-a-real-paper}

\begin{enumerate}
\def\labelenumi{\arabic{enumi}.}
\tightlist
\item
  Make the half-density normalization argument precise for a concrete
  groupoid or kernel model (tangent-groupoid or short-time propagator
  model).
\item
  Show how the area scale enters stationary-phase prefactors and how
  this interacts with RG scaling.
\item
  General-dimension analysis: clarify what replaces ``area'' in odd
  dimensions and whether a universal normalization is still defensible.
\item
  Identify minimal hypotheses under which ``need of half-density scale
  \(\Rightarrow\) Planck area'' is more than dimensional bookkeeping.
\end{enumerate}

\end{document}
