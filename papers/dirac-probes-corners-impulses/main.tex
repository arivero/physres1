% Options for packages loaded elsewhere
\PassOptionsToPackage{unicode}{hyperref}
\PassOptionsToPackage{hyphens}{url}
%
\documentclass[
]{article}
\usepackage{amsmath,amssymb}
\usepackage{lmodern}
\usepackage{iftex}
\ifPDFTeX
  \usepackage[T1]{fontenc}
  \usepackage[utf8]{inputenc}
  \usepackage{textcomp} % provide euro and other symbols
\else % if luatex or xetex
  \usepackage{unicode-math}
  \defaultfontfeatures{Scale=MatchLowercase}
  \defaultfontfeatures[\rmfamily]{Ligatures=TeX,Scale=1}
\fi
% Use upquote if available, for straight quotes in verbatim environments
\IfFileExists{upquote.sty}{\usepackage{upquote}}{}
\IfFileExists{microtype.sty}{% use microtype if available
  \usepackage[]{microtype}
  \UseMicrotypeSet[protrusion]{basicmath} % disable protrusion for tt fonts
}{}
\makeatletter
\@ifundefined{KOMAClassName}{% if non-KOMA class
  \IfFileExists{parskip.sty}{%
    \usepackage{parskip}
  }{% else
    \setlength{\parindent}{0pt}
    \setlength{\parskip}{6pt plus 2pt minus 1pt}}
}{% if KOMA class
  \KOMAoptions{parskip=half}}
\makeatother
\usepackage{xcolor}
\IfFileExists{xurl.sty}{\usepackage{xurl}}{} % add URL line breaks if available
\IfFileExists{bookmark.sty}{\usepackage{bookmark}}{\usepackage{hyperref}}
\hypersetup{
  pdftitle={Dirac-Supported Probes, Corners, and Impulses: A Variational Note},
  pdfauthor={Alejandro Rivero},
  hidelinks,
  pdfcreator={LaTeX via pandoc}}
\urlstyle{same} % disable monospaced font for URLs
\usepackage{longtable,booktabs,array}
\usepackage{calc} % for calculating minipage widths
% Correct order of tables after \paragraph or \subparagraph
\usepackage{etoolbox}
\makeatletter
\patchcmd\longtable{\par}{\if@noskipsec\mbox{}\fi\par}{}{}
\makeatother
% Allow footnotes in longtable head/foot
\IfFileExists{footnotehyper.sty}{\usepackage{footnotehyper}}{\usepackage{footnote}}
\makesavenoteenv{longtable}
\setlength{\emergencystretch}{3em} % prevent overfull lines
\providecommand{\tightlist}{%
  \setlength{\itemsep}{0pt}\setlength{\parskip}{0pt}}
\setcounter{secnumdepth}{-\maxdimen} % remove section numbering
\ifLuaTeX
  \usepackage{selnolig}  % disable illegal ligatures
\fi

\title{Dirac-Supported Probes, Corners, and Impulses: A Variational
Note}
\author{Alejandro Rivero}
\date{2026}

\begin{document}
\maketitle
\begin{abstract}
Variational principles routinely invoke ``point-like probes'' of
extrema, yet the precise hypotheses under which such probes are safe are
often left implicit. This note collects the functional-analytic
conditions that make mollifier-based localization of the Euler--Lagrange
equation rigorous, states them as an explicit theorem, and works through
a complete model --- the free particle with a single delta-kick --- to
illustrate corners, impulse jumps, and the role of distributional
forcing. A clean separation is maintained between \emph{Dirac-supported
variations} (always safe under stated regularity) and \emph{delta
potentials} (which in dimension \(d\ge 2\) require renormalization and
are a distinct mathematical object).
\end{abstract}

This note is a companion to the cornerstone manuscript. It expands the
content of Section 5 there into a self-contained treatment with sharper
hypotheses and a worked model.

\hypertarget{motivation}{%
\section{1. Motivation}\label{motivation}}

The cornerstone manuscript (Section 5) introduces weak stationarity,
mollifier probing, and corner/impulse conditions as Propositions
P3.1--P3.4. Those statements are sufficient for the structural chain
developed there, but they compress the hypotheses and omit worked
computations. This satellite note serves three purposes:

\begin{enumerate}
\def\labelenumi{\arabic{enumi}.}
\tightlist
\item
  State the mollifier localization result as a formal theorem with
  explicit, numbered hypotheses (Section 2).
\item
  Work through a complete model --- the delta-kick free particle ---
  showing trajectory, momentum jump, and action evaluation in full
  detail (Section 4).
\item
  Separate two superficially similar but logically distinct uses of the
  Dirac delta in variational mechanics (Section 5).
\end{enumerate}

\hypertarget{mollifier-localization-theorem}{%
\section{2. Mollifier Localization
Theorem}\label{mollifier-localization-theorem}}

We work on a time interval \([t_i,t_f]\) with Lagrangian
\(\mathcal{L}(q,\dot{q},t)\) and candidate trajectory
\(q:[t_i,t_f]\to\mathbb{R}^d\).

\texttt{Theorem\ 2.1\ (Mollifier\ localization\ of\ the\ Euler–Lagrange\ equation).}
Assume:

(H1) \(q\in C^1([t_i,t_f];\mathbb{R}^d)\) and \(\mathcal{L}\) is \(C^2\)
in \((q,\dot{q})\) and \(C^0\) in \(t\).

(H2) The first variation satisfies \(\delta S[q;\eta]=0\) for every
\(\eta\in C_c^\infty((t_i,t_f);\mathbb{R}^d)\).

(H3) The Euler--Lagrange expression \[
F[q](t):=\frac{\partial\mathcal{L}}{\partial q}(q,\dot{q},t)
-\frac{d}{dt}\frac{\partial\mathcal{L}}{\partial\dot{q}}(q,\dot{q},t)
\] is continuous at a point \(t_0\in(t_i,t_f)\).

Then \(F[q](t_0)=0\).

\texttt{Proof.} Fix a nonnegative mollifier
\(\rho\in C_c^\infty(\mathbb{R})\) with \(\int\rho=1\) and set
\(\rho_\varepsilon(s)=\varepsilon^{-1}\rho(s/\varepsilon)\). For any
unit vector \(u\in\mathbb{R}^d\), the test variation
\(\eta_\varepsilon(t)=\rho_\varepsilon(t-t_0)\,u\) is in \(C_c^\infty\)
for \(\varepsilon\) small enough. By (H2): \[
0=\delta S[q;\eta_\varepsilon]=\int_{t_i}^{t_f}F[q](t)\cdot\rho_\varepsilon(t-t_0)\,u\,dt
=u\cdot\int_{t_i}^{t_f}\rho_\varepsilon(t-t_0)\,F[q](t)\,dt.
\] By (H3) the convolution converges to \(F[q](t_0)\) as
\(\varepsilon\to0^+\). Since \(u\) is arbitrary, \(F[q](t_0)=0\).
\(\square\)

\texttt{Remark\ 2.2\ (Role\ of\ each\ hypothesis).} (H1) ensures
\(F[q]\) is locally integrable so the distributional pairing makes
sense. (H2) is the global stationarity input. (H3) is the local
regularity gate: without it, mollifier limits may fail to converge or
may converge to an averaged value rather than a pointwise one. If
\(F[q]\) is continuous on all of \((t_i,t_f)\), iteration of Theorem 2.1
recovers the classical Euler--Lagrange equation everywhere.

\hypertarget{corners-and-impulses-formal-statements}{%
\section{3. Corners and Impulses: Formal
Statements}\label{corners-and-impulses-formal-statements}}

When hypothesis (H3) fails --- because \(\dot{q}\) or external forcing
is discontinuous --- two distinct situations arise.

\hypertarget{corners-unforced-velocity-jump}{%
\subsection{3.1 Corners (unforced velocity
jump)}\label{corners-unforced-velocity-jump}}

\texttt{Theorem\ 3.1\ (Corner\ condition\ /\ Weierstrass–Erdmann).}
Assume \(q\) is piecewise \(C^2\) with a single velocity discontinuity
at \(t_0\), satisfying the unforced Euler--Lagrange equation on
\((t_i,t_0)\) and \((t_0,t_f)\) separately. Then the canonical momentum
is continuous at \(t_0\): \[
\left[\frac{\partial\mathcal{L}}{\partial\dot{q}}\right]_{t_0^-}^{t_0^+}=0.
\]

\texttt{Proof.} Integrate the Euler--Lagrange equation over
\([t_0-\varepsilon,t_0+\varepsilon]\). The integral of
\(\partial_q\mathcal{L}\) vanishes as \(\varepsilon\to0\) by
boundedness; the derivative term yields the momentum jump. \(\square\)

\hypertarget{impulses-delta-forcing}{%
\subsection{3.2 Impulses (delta forcing)}\label{impulses-delta-forcing}}

\texttt{Theorem\ 3.2\ (Impulse\ jump\ condition).} Consider the forced
distributional equation \[
\frac{d}{dt}\frac{\partial\mathcal{L}}{\partial\dot{q}}
-\frac{\partial\mathcal{L}}{\partial q}
=J\,\delta(t-t_0),
\quad J\in\mathbb{R}^d.
\] If \(\partial_{\dot{q}}\mathcal{L}\) has one-sided limits at \(t_0\),
then \[
\frac{\partial\mathcal{L}}{\partial\dot{q}}(t_0^+)
-\frac{\partial\mathcal{L}}{\partial\dot{q}}(t_0^-)
=J.
\]

\texttt{Proof.} Same integration argument: the delta integrates to
\(J\), the smooth remainder vanishes. \(\square\)

The distinction is structural: corners arise from variational boundary
conditions (matching at a junction), while impulses arise from external
forcing (a source term in the equation of motion).

\hypertarget{worked-model-free-particle-with-a-single-delta-kick}{%
\section{4. Worked Model: Free Particle with a Single
Delta-Kick}\label{worked-model-free-particle-with-a-single-delta-kick}}

We give a complete computation that illustrates both Theorem 3.2 and the
evaluation of action on a kinked trajectory.

\hypertarget{setup}{%
\subsection{4.1 Setup}\label{setup}}

Consider a particle of mass \(m\) in one dimension with Lagrangian
\(\mathcal{L}=\frac{m}{2}\dot{q}^2\) and an external impulsive force
\(J\,\delta(t-t_0)\) applied at time \(t_0\in(0,T)\). The equation of
motion is \[
m\ddot{q}=J\,\delta(t-t_0).
\]

\hypertarget{solution}{%
\subsection{4.2 Solution}\label{solution}}

The trajectory is piecewise linear: \[
q(t)=\begin{cases}
q_i+v_-\,t & 0\le t<t_0,\\
q_i+v_-\,t_0+v_+\,(t-t_0) & t_0\le t\le T,
\end{cases}
\] with the velocity jump \(v_+-v_-=J/m\) from Theorem 3.2.

Boundary conditions \(q(0)=q_i\), \(q(T)=q_f\) fix the velocities.
Writing \(\Delta v=J/m\): \[
v_-=\frac{q_f-q_i-\Delta v\,(T-t_0)}{T},
\qquad
v_+=v_-+\Delta v.
\]

\hypertarget{action-evaluation}{%
\subsection{4.3 Action evaluation}\label{action-evaluation}}

The action splits across the kink: \[
S=\frac{m}{2}\bigl(v_-^2\,t_0+v_+^2\,(T-t_0)\bigr).
\] In the unforced limit (\(J=0\), so \(\Delta v=0\)): \[
S_0=\frac{m}{2}\frac{(q_f-q_i)^2}{T},
\] the standard free-particle result. The impulse adds a
positive-definite kinetic energy cost: \[
S-S_0=\frac{m}{2}\frac{t_0(T-t_0)}{T}\,(\Delta v)^2>0\quad(J\neq0).
\] This confirms that the delta-kick raises the action above the free
minimum --- the impulsive trajectory is not an extremum of the unforced
problem.

\hypertarget{angular-momentum-preservation-under-central-impulses}{%
\subsection{4.4 Angular momentum preservation under central
impulses}\label{angular-momentum-preservation-under-central-impulses}}

For a central force in the plane, the impulse is radial:
\(J=J_r\,\hat{r}\). Since angular momentum depends only on the
transverse velocity component, \[
L=m\,r\,\dot{\theta},
\] a purely radial impulse leaves \(\dot{\theta}\) (and hence \(L\))
unchanged across the kick, recovering the equal-area property of
Newton's polygon at the distributional level.

\hypertarget{from-n-impulses-to-the-time-sliced-path-integral}{%
\subsection{4.5 From N impulses to the time-sliced path
integral}\label{from-n-impulses-to-the-time-sliced-path-integral}}

The single-impulse model extends naturally to a sequence of \(N\)
impulses. This extension bridges the distributional mechanics of
Sections 3--4 to the path-integral composition framework of the
cornerstone manuscript (Section 6 there).

Partition \([0,T]\) into \(N+1\) equal intervals of length
\(\Delta t=T/(N+1)\), with junction times \(t_k=k\,\Delta t\) for
\(k=1,\ldots,N\). Fix the endpoints \(q_0=q_i\), \(q_{N+1}=q_f\), and
let \(q_1,\ldots,q_N\) be free intermediate positions. On each segment
the particle is free, so the trajectory is piecewise linear with
velocities \[
v_k=\frac{q_{k+1}-q_k}{\Delta t},\qquad k=0,\ldots,N.
\] The discrete action is \[
S_N[\{q_k\}]=\sum_{k=0}^{N}\frac{m}{2}\frac{(q_{k+1}-q_k)^2}{\Delta t}.
\] At each junction \(t_k\), the velocity jumps from \(v_{k-1}\) to
\(v_k\). By Theorem 3.2, each jump requires an impulse
\(J_k=m(v_k-v_{k-1})\). The \emph{classical} stationary condition
\(\partial S_N/\partial q_k=0\) imposes \(v_k=v_{k-1}\) for all \(k\)
--- that is, Theorem 3.1's corner condition (momentum continuity) at
every junction --- and the path collapses to a single straight line.

In the quantum theory, one instead sums over all intermediate
configurations with amplitude weights: \[
K(q_f,q_i;T)=\lim_{N\to\infty}\left(\frac{m}{2\pi i\hbar\,\Delta t}\right)^{(N+1)/2}\int\prod_{k=1}^{N}dq_k\;
\exp\!\left(\frac{i}{\hbar}\,S_N[\{q_k\}]\right).
\] There are \(N+1\) segments and \(N\) intermediate integrations; each
segment contributes one factor of \(\sqrt{m/(2\pi i\hbar\,\Delta t)}\),
giving the exponent \((N+1)/2\). This is precisely the half-density
normalization required for the composition law to hold at each
intermediate integration --- a point treated systematically in the
cornerstone's half-density framework. The distributional
impulse-matching of Theorem 3.2 thus connects, through this
\(N\to\infty\) limit, to the composition postulate for transition
amplitudes.

\hypertarget{safe-vs-unsafe-uses-of-the-dirac-delta-in-variational-mechanics}{%
\section{5. Safe vs Unsafe Uses of the Dirac Delta in Variational
Mechanics}\label{safe-vs-unsafe-uses-of-the-dirac-delta-in-variational-mechanics}}

The preceding sections involve two related but \emph{logically distinct}
mathematical objects. Conflating them is a common source of error.

\hypertarget{dirac-supported-variations-safe-under-regularity}{%
\subsection{5.1 Dirac-supported variations (safe under
regularity)}\label{dirac-supported-variations-safe-under-regularity}}

Using mollifier sequences \(\rho_\varepsilon\to\delta\) as \emph{test
functions} against a continuous integrand is always safe --- it is
standard distribution theory. This is Theorem 2.1. No renormalization or
regularization ambiguity arises; the \(\varepsilon\to0\) limit is unique
and controlled by continuity.

\hypertarget{delta-potentials-require-renormalization}{%
\subsection{5.2 Delta potentials (require
renormalization)}\label{delta-potentials-require-renormalization}}

A point interaction \(V(q)=g\,\delta(q)\) in the Hamiltonian is a
different object. In dimensions \(d\ge2\), the naive coupling constant
\(g\) requires renormalization (the resolvent acquires a logarithmic or
power-law divergence depending on \(d\)). In \(d=1\) the delta potential
is well-defined without renormalization, but this is an accident of low
dimension, not a general principle. The companion note on delta objects
treats the half-density kernel structure of point interactions in
detail.

\hypertarget{summary-table}{%
\subsection{5.3 Summary table}\label{summary-table}}

\begin{longtable}[]{@{}lll@{}}
\toprule
Object & Math status & Renormalization? \\
\midrule
\endhead
Mollifier probe of \(F[q]\) (Thm 2.1) & Rigorous & No \\
Corner/impulse matching (Thms 3.1--3.2) & Rigorous & No \\
\(\delta\) potential, \(d=1\) & Well-defined & No \\
\(\delta\) potential, \(d\ge 2\) & Requires care & Yes \\
Products \(\delta(t)^2\) & Undefined & Always \\
\bottomrule
\end{longtable}

\hypertarget{outlook}{%
\section{6. Outlook}\label{outlook}}

\begin{enumerate}
\def\labelenumi{\arabic{enumi}.}
\tightlist
\item
  The stochastic-forcing interpretation of Section 4.5's \(N\)-impulse
  model --- random impulses with prescribed statistics --- remains open
  as a bridge to stochastic mechanics.
\item
  Treat the piecewise-smooth trajectory as a weak solution and examine
  whether the Hamilton--Jacobi equation acquires viscosity-solution
  structure at the kink.
\item
  Connect the corner-condition analysis to broken geodesics in
  Riemannian geometry (Synge's world function approach).
\end{enumerate}

\hypertarget{references}{%
\section{References}\label{references}}

\begin{enumerate}
\def\labelenumi{\arabic{enumi}.}
\tightlist
\item
  {[}Gelfand1963{]} I. M. Gelfand and S. V. Fomin, \emph{Calculus of
  Variations}, Prentice-Hall, 1963. (Reprinted by Dover, 2000.)
\item
  {[}Giaquinta1996{]} M. Giaquinta and S. Hildebrandt, \emph{Calculus of
  Variations I: The Lagrangian Formalism}, Springer, 1996.
\item
  {[}AlbeverioGesztesyHoeghKrohnHolden2005{]} S. Albeverio, F. Gesztesy,
  R. Høegh-Krohn, and H. Holden, \emph{Solvable Models in Quantum
  Mechanics}, 2nd ed., AMS Chelsea Publishing, 2005.
\item
  {[}FeynmanHibbs1965{]} R. P. Feynman and A. R. Hibbs, \emph{Quantum
  Mechanics and Path Integrals}, McGraw-Hill, 1965. (Path integral as
  refinement limit of time-sliced amplitudes.)
\end{enumerate}

\end{document}
