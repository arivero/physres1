% Options for packages loaded elsewhere
\PassOptionsToPackage{unicode}{hyperref}
\PassOptionsToPackage{hyphens}{url}
\documentclass[
]{article}
\usepackage{xcolor}
\usepackage{amsmath,amssymb}
\setcounter{secnumdepth}{-\maxdimen} % remove section numbering
\usepackage{iftex}
\ifPDFTeX
  \usepackage[T1]{fontenc}
  \usepackage[utf8]{inputenc}
  \usepackage{textcomp} % provide euro and other symbols
\else % if luatex or xetex
  \usepackage{unicode-math} % this also loads fontspec
  \defaultfontfeatures{Scale=MatchLowercase}
  \defaultfontfeatures[\rmfamily]{Ligatures=TeX,Scale=1}
\fi
\usepackage{lmodern}
\ifPDFTeX\else
  % xetex/luatex font selection
\fi
% Use upquote if available, for straight quotes in verbatim environments
\IfFileExists{upquote.sty}{\usepackage{upquote}}{}
\IfFileExists{microtype.sty}{% use microtype if available
  \usepackage[]{microtype}
  \UseMicrotypeSet[protrusion]{basicmath} % disable protrusion for tt fonts
}{}
\makeatletter
\@ifundefined{KOMAClassName}{% if non-KOMA class
  \IfFileExists{parskip.sty}{%
    \usepackage{parskip}
  }{% else
    \setlength{\parindent}{0pt}
    \setlength{\parskip}{6pt plus 2pt minus 1pt}}
}{% if KOMA class
  \KOMAoptions{parskip=half}}
\makeatother
\setlength{\emergencystretch}{3em} % prevent overfull lines
\providecommand{\tightlist}{%
  \setlength{\itemsep}{0pt}\setlength{\parskip}{0pt}}
\usepackage{bookmark}
\IfFileExists{xurl.sty}{\usepackage{xurl}}{} % add URL line breaks if available
\urlstyle{same}
\hypersetup{
  pdftitle={``Uncuttable'' as Controlled Refinement},
  hidelinks,
  pdfcreator={LaTeX via pandoc}}

\title{``Uncuttable'' as Controlled Refinement}
\author{}
\date{}

\begin{document}
\maketitle
\begin{abstract}
This note fixes a project-internal meaning of ``uncuttable'' aligned
with the refinement-compatibility thesis of \texttt{paper/main.md}. Here
\textbf{uncuttable} does not mean ``indivisible.'' It means: the
quantity of interest is not determined by any \emph{finite} dissection
alone; it is a \textbf{limit object} whose definition requires a
refinement rule and a comparison structure across refinements.

The point is structural and mathematical: once a theory is built from
composable local pieces, the continuum theory is the stable target of a
refinement limit, and extra control data may be required for that limit
to exist or be unique.
\end{abstract}

\section{1. Definition}\label{definition}

Call a quantity \(Q\) \textbf{uncuttable} (in this note's sense) if: 1.
there exists a family of finite approximants \(Q_N\) produced by a
finite dissection/refinement scheme of depth \(N\), but 2. the value of
interest is not any finite \(Q_N\); it is a controlled limit
\(Q=\lim_{N\to\infty}Q_N\), and 3. specifying the \emph{rule of
refinement} and the \emph{comparison across refinements} is part of the
definition of \(Q\).

This is the ordinary situation in analysis: finite partitions
approximate, but the object is defined by a limiting procedure together
with hypotheses that ensure convergence/uniqueness.

\section{2. Toy model: an integral is already a refinement
limit}\label{toy-model-an-integral-is-already-a-refinement-limit}

Let \(f:[a,b]\to\mathbb R\). A prototypical refinement family is a
partition \(P_N=\{a=t_0<\cdots<t_N=b\}\) with mesh
\(\|P_N\|:=\max_k(t_{k+1}-t_k)\to 0\). Define the Riemann-sum
approximants \[
Q_N:=\sum_{k=0}^{N-1} f(\xi_k)\,(t_{k+1}-t_k),
\qquad \xi_k\in[t_k,t_{k+1}].
\] In good cases, \(Q_N\to \int_a^b f(t)\,dt\) as \(\|P_N\|\to 0\), and
the limit is independent of the tags \(\xi_k\). But this is not a
tautology: the limit can fail to exist or can depend on the refinement
rule unless hypotheses are stated.

In the present program, this is the basic moral: finite cuts
approximate, but the value is defined by \textbf{controlled refinement}.

\section{3. Dynamics: action, stationarity, and the need for control
data}\label{dynamics-action-stationarity-and-the-need-for-control-data}

The cornerstone manuscript uses the same template in mechanics. Given a
partition of time, the discrete action \[
S_N[q]=\sum_k \mathcal L\!\left(q_k,\frac{q_{k+1}-q_k}{\Delta t_k},t_k\right)\Delta t_k
\] is a finite refinement approximant. The continuum action
\(S[q]=\int \mathcal L\,dt\) is a refinement limit.

Two ``uncuttable'' features appear immediately when one pushes beyond
smooth classical paths: 1. \textbf{Singular probes and corners:}
stationarity must be interpreted in weak/distributional form;
point-supported variations require mollification. 2.
\textbf{Non-uniqueness of refinement schemes:} different discretization
conventions (even if classically equivalent) can produce distinct
refined objects unless an equivalence or control map is specified.

These are exactly the obstructions enumerated in \texttt{paper/main.md}
(\texttt{Heuristic\ H0.2}): the point is not indivisible atoms, but
limit control.

\section{4. Outlook: refinement compatibility as ``the extra
structure''}\label{outlook-refinement-compatibility-as-the-extra-structure}

In this repo, the ``extra structure'' used to control refinement limits
is made explicit: - half-densities make kernel composition
coordinate-free without hidden measure choices, - control maps \(\tau\)
encode how parameters must flow under refinement to maintain stability,
- renormalization is the compatibility rule when naive refinement limits
diverge.

This note is therefore a small conceptual bridge: it isolates an early,
analysis-level instance of the same meta-problem that reappears in
quantization and in QFT.

\end{document}
