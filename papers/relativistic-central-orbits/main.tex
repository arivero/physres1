% Options for packages loaded elsewhere
\PassOptionsToPackage{unicode}{hyperref}
\PassOptionsToPackage{hyphens}{url}
%
\documentclass[
]{article}
\usepackage{amsmath,amssymb}
\usepackage{lmodern}
\usepackage{iftex}
\ifPDFTeX
  \usepackage[T1]{fontenc}
  \usepackage[utf8]{inputenc}
  \usepackage{textcomp} % provide euro and other symbols
\else % if luatex or xetex
  \usepackage{unicode-math}
  \defaultfontfeatures{Scale=MatchLowercase}
  \defaultfontfeatures[\rmfamily]{Ligatures=TeX,Scale=1}
\fi
% Use upquote if available, for straight quotes in verbatim environments
\IfFileExists{upquote.sty}{\usepackage{upquote}}{}
\IfFileExists{microtype.sty}{% use microtype if available
  \usepackage[]{microtype}
  \UseMicrotypeSet[protrusion]{basicmath} % disable protrusion for tt fonts
}{}
\makeatletter
\@ifundefined{KOMAClassName}{% if non-KOMA class
  \IfFileExists{parskip.sty}{%
    \usepackage{parskip}
  }{% else
    \setlength{\parindent}{0pt}
    \setlength{\parskip}{6pt plus 2pt minus 1pt}}
}{% if KOMA class
  \KOMAoptions{parskip=half}}
\makeatother
\usepackage{xcolor}
\IfFileExists{xurl.sty}{\usepackage{xurl}}{} % add URL line breaks if available
\IfFileExists{bookmark.sty}{\usepackage{bookmark}}{\usepackage{hyperref}}
\hypersetup{
  pdftitle={Relativistic Central Orbits as Refinement-Witnesses},
  pdfauthor={Alejandro Rivero},
  hidelinks,
  pdfcreator={LaTeX via pandoc}}
\urlstyle{same} % disable monospaced font for URLs
\setlength{\emergencystretch}{3em} % prevent overfull lines
\providecommand{\tightlist}{%
  \setlength{\itemsep}{0pt}\setlength{\parskip}{0pt}}
\setcounter{secnumdepth}{-\maxdimen} % remove section numbering
\ifLuaTeX
  \usepackage{selnolig}  % disable illegal ligatures
\fi

\title{Relativistic Central Orbits as Refinement-Witnesses}
\author{Alejandro Rivero}
\date{2026}

\begin{document}
\maketitle
\begin{abstract}
Central-force motion is a clean domain where ``refinement'' arguments
can be made explicit: Newton's polygonal limit gives exact finite-step
invariants, while relativistic kinematics introduces new admissibility
constraints even before general relativity is invoked. This dependent
note records one such constraint in special relativity for
inverse-square forces: circular orbits obey \(v=K/L\) and therefore
require \(L>K/c\). The goal is not to replace standard treatments, but
to keep a minimal derivation-first record of what changes when the same
geometric refinement story is pushed into relativistic regimes.
\end{abstract}

\hypertarget{purpose-and-relation-to-the-cornerstone-paper}{%
\section{1. Purpose and Relation to the Cornerstone
Paper}\label{purpose-and-relation-to-the-cornerstone-paper}}

The cornerstone manuscript uses central-force refinement as a structural
bridge (equal areas / angular momentum preservation \(\leftrightarrow\)
action additivity \(\leftrightarrow\) composition).

This note is outside the scope of the cornerstone paper but examines how
relativistic kinematics modifies the simplest central-force
circular-orbit conditions. The inverse-square case is singled out by an
exact simplification already at the SR level.

\hypertarget{sr-circular-motion-under-a-power-law-force}{%
\section{2. SR Circular Motion Under a Power-Law
Force}\label{sr-circular-motion-under-a-power-law-force}}

Assume a particle of rest mass \(m\) executes uniform circular motion of
radius \(r\) and speed \(v\). The Lorentz factor is \[
\gamma=\frac{1}{\sqrt{1-v^2/c^2}}.
\] For uniform circular motion the acceleration is perpendicular to the
velocity, so \[
F=\frac{dp}{dt}=\gamma m a_\perp=\gamma m \frac{v^2}{r}.
\]

Assume an attractive central force magnitude \[
F(r)=\frac{K}{r^q},\qquad K>0.
\] Then the circular-orbit condition is \[
\frac{K}{r^q}=\gamma m \frac{v^2}{r}
\quad\Longleftrightarrow\quad
K=\gamma m v^2 r^{q-1}.
\]

The angular momentum magnitude is \[
L=r p=\gamma m r v.
\] Eliminating \(v\) gives the SR circular-orbit condition \[
L^2 = K\,\gamma\,m\, r^{3-q}.
\]

This reduces to the Newtonian relation when \(\gamma\to 1\)
(i.e.~\(c\to\infty\)).

\hypertarget{inverse-square-q2-special-case-vkl-and-the-bound-lkc}{%
\section{\texorpdfstring{3. Inverse-Square (\(q=2\)) Special Case:
\(v=K/L\) and the Bound
\(L>K/c\)}{3. Inverse-Square (q=2) Special Case: v=K/L and the Bound L\textgreater K/c}}\label{inverse-square-q2-special-case-vkl-and-the-bound-lkc}}

For \(q=2\), \[
K=\gamma m v^2 r,\qquad L=\gamma m r v.
\] Dividing yields the exact identity \[
\boxed{\,v=\frac{K}{L}\,}.
\] Therefore a relativistic circular orbit requires \(v<c\), hence the
angular-momentum bound \[
\boxed{\,L>\frac{K}{c}\,}.
\]

The radius follows from \(r=L^2/(K\gamma m)\) with
\(\gamma=(1-v^2/c^2)^{-1/2}\) and \(v=K/L\): \[
r=\frac{L^2}{Km}\sqrt{1-\frac{K^2}{c^2L^2}}
\xrightarrow[c\to\infty]{}\frac{L^2}{Km}.
\]

This bound and the circular-solution branch appear in standard
treatments of the relativistic Kepler problem (see
e.g.~{[}BoscagginDambrosioFeltrin2020RelKepler{]} for a
dynamical-systems/variational analysis of the same SR equation).

\hypertarget{gr-baseline-schwarzschild-geodesics-effective-potential-photon-sphere}{%
\section{4. GR Baseline: Schwarzschild Geodesics, Effective Potential,
Photon
Sphere}\label{gr-baseline-schwarzschild-geodesics-effective-potential-photon-sphere}}

This section records the standard Schwarzschild baseline in a form
parallel to the SR ``effective 1D radial motion'' viewpoint.

Conventions: set \(G=c=1\). Then the Schwarzschild metric is \[
ds^2 = -\left(1-\frac{2M}{r}\right)dt^2 + \left(1-\frac{2M}{r}\right)^{-1}dr^2 + r^2(d\theta^2+\sin^2\theta\,d\phi^2),
\] with horizon at \(r=2M\). Restrict to the equatorial plane
\(\theta=\pi/2\).

Using an affine parameter \(\lambda\), define \(\epsilon=1\) (timelike,
\(\lambda=\tau\)) and \(\epsilon=0\) (null). Energy and angular momentum
from the Killing fields \(\partial_t,\partial_\phi\) are \[
E=\left(1-\frac{2M}{r}\right)\dot t,\qquad L=r^2\dot\phi,
\] so \(\dot t=E/(1-2M/r)\), \(\dot\phi=L/r^2\).

The normalization condition \(g_{\mu\nu}\dot x^\mu\dot x^\nu=-\epsilon\)
yields the radial equation \[
\dot r^2 = E^2 - \left(1-\frac{2M}{r}\right)\left(\epsilon+\frac{L^2}{r^2}\right),
\] i.e.~\(\dot r^2 + V_{\mathrm{eff}}(r)=E^2\) with \[
V_{\mathrm{eff}}(r)=\left(1-\frac{2M}{r}\right)\left(\epsilon+\frac{L^2}{r^2}\right).
\]

\hypertarget{null-circular-orbit-photon-sphere}{%
\subsection{4.1 Null circular orbit (photon
sphere)}\label{null-circular-orbit-photon-sphere}}

For null geodesics \(\epsilon=0\), \[
V_{\mathrm{eff}}(r)=\left(1-\frac{2M}{r}\right)\frac{L^2}{r^2}.
\] Circular null orbits satisfy \(\dot r=0\) and
\(dV_{\mathrm{eff}}/dr=0\), which gives the photon-sphere radius \[
\boxed{\,r_{\mathrm{ph}}=3M\,}.
\] The impact parameter \(b=L/E\) obeys \(b^2=r^2/(1-2M/r)\), so at
\(r=3M\) one has \(b=3\sqrt3\,M\).

Baseline anchor for these standard results: {[}Carroll1997LectureGR{]}.

\hypertarget{restoring-units}{%
\subsection{4.2 Restoring units}\label{restoring-units}}

Replace \(M\) by \(GM/c^2\): \(r_{\mathrm{ph}}=3GM/c^2\),
\(r_h=2GM/c^2\), and \(b=3\sqrt3\,GM/c^2\).

\hypertarget{timelike-circular-orbits-and-isco}{%
\subsection{4.3 Timelike circular orbits and
ISCO}\label{timelike-circular-orbits-and-isco}}

For timelike geodesics \(\epsilon=1\), \[
V_{\mathrm{eff}}(r)=\left(1-\frac{2M}{r}\right)\left(1+\frac{L^2}{r^2}\right).
\] Circular orbits satisfy \(\dot r=0\) and \(dV_{\mathrm{eff}}/dr=0\).
Writing \[
V_{\mathrm{eff}}(r)=1-\frac{2M}{r}+\frac{L^2}{r^2}-\frac{2ML^2}{r^3},
\] one finds \[
\frac{dV_{\mathrm{eff}}}{dr}=\frac{2M}{r^2}-\frac{2L^2}{r^3}+\frac{6ML^2}{r^4}.
\] Thus the circular branch obeys \[
Mr^2=L^2(r-3M)
\quad\Rightarrow\quad
\boxed{\,L^2(r)=\frac{Mr^2}{r-3M}\,},\qquad r>3M,
\] and using \(E^2=V_{\mathrm{eff}}(r)\) on the circular orbit gives \[
\boxed{\,E^2(r)=\frac{(r-2M)^2}{r(r-3M)}\,}.
\]

Stability requires \(V_{\mathrm{eff}}''(r)>0\) at the circular orbit.
Differentiating once more, \[
V_{\mathrm{eff}}''(r)=-\frac{4M}{r^3}+\frac{6L^2}{r^4}-\frac{24ML^2}{r^5},
\] and substituting the circular-orbit value \(L^2=Mr^2/(r-3M)\) yields
\[
V_{\mathrm{eff}}''(r)=\frac{2M(r-6M)}{r^3(r-3M)}.
\] So circular timelike orbits are stable iff \(r>6M\), and the
innermost stable circular orbit is \[
\boxed{\,r_{\mathrm{ISCO}}=6M\,}.
\] At the ISCO, \[
L=2\sqrt3\,M,\qquad E=\frac{2\sqrt2}{3}.
\]

\hypertarget{sr-stability-of-circular-orbits-small-radial-perturbations}{%
\section{5. SR Stability of Circular Orbits (Small Radial
Perturbations)}\label{sr-stability-of-circular-orbits-small-radial-perturbations}}

This section stays within mechanical SR (a point particle in an
external, time-independent central scalar potential \(U(r)\)). It is
used as a kinematic witness: some restrictions already appear before GR
or field-theoretic interactions enter.

Fix the (conserved) angular momentum magnitude \(L\). For purely
tangential motion (\(p_r=0\)), define the fixed-\(L\) energy function \[
W_L(r)=\sqrt{m^2c^4+\frac{L^2c^2}{r^2}}+U(r).
\] Assume the attractive power-law force \(F(r)=K/r^q\) so that
\(U'(r)=K/r^q\).

\hypertarget{circularity-and-the-effective-potential-viewpoint}{%
\subsection{5.1 Circularity and the effective-potential
viewpoint}\label{circularity-and-the-effective-potential-viewpoint}}

Circular orbits satisfy \(W_L'(r_0)=0\), i.e. \[
\frac{K}{r_0^q}=\frac{L^2c^2}{r_0^3\sqrt{m^2c^4+L^2c^2/r_0^2}}
\quad\Longleftrightarrow\quad
L^2=K\,\gamma\,m\,r_0^{3-q},
\] since \(\sqrt{m^2c^4+L^2c^2/r_0^2}=\gamma mc^2\). This recovers the
SR circular-orbit condition used earlier.

\hypertarget{stability-inequality}{%
\subsection{5.2 Stability inequality}\label{stability-inequality}}

At fixed \(L\), stability under small radial perturbations is the
local-minimum condition \[
W_L''(r_0)>0.
\] Writing the Lorentz factor at the orbit as \[
\gamma^2 = 1+\frac{L^2}{m^2c^2r_0^2},
\] one finds the compact expression \[
\boxed{\;
W_L''(r_0)=\frac{L^2}{m r_0^4\,\gamma^3}\Bigl(1+(2-q)\gamma^2\Bigr),
\;}
\] hence the stability criterion \[
\boxed{\;1+(2-q)\gamma^2>0.\;}
\]

In the Newtonian limit \(\gamma\to 1\) this reduces to the standard
condition \(q<3\). For \(2<q<3\) SR adds a speed bound: \[
\gamma^2<\frac{1}{q-2}
\quad\Longleftrightarrow\quad
\frac{v^2}{c^2}<3-q.
\] In particular, inverse-square forces (\(q=2\)) are stable for all
\(\gamma\) in this model, while \(q\ge 3\) yields no stable circular
orbits (beyond the Newtonian marginal case at \(q=3\)).

\hypertarget{outlook}{%
\section{6. Outlook}\label{outlook}}

Two natural extensions are: 1. replace the ``external potential''
modeling assumption by an explicitly field-mediated interaction model,
and compare which orbit admissibility/stability bounds survive that
change; 2. connect the SR/GR orbit constraints more explicitly to the
refinement-compatibility spine (what is preserved under refinement, and
what new kinematic admissibility conditions appear when the refinement
rules are Lorentz/GR-consistent).

\hypertarget{references}{%
\section{References}\label{references}}

\begin{enumerate}
\def\labelenumi{\arabic{enumi}.}
\tightlist
\item
  {[}BoscagginDambrosioFeltrin2020RelKepler{]} Alberto Boscaggin, Walter
  Dambrosio, and Guglielmo Feltrin, ``Periodic solutions to a perturbed
  relativistic Kepler problem,'' arXiv:\texttt{2003.03110} (v1, 6 Mar
  2020). (Contains the unperturbed SR relativistic Kepler equation and
  discusses circular solutions/constraints.)
\item
  {[}Carroll1997LectureGR{]} Sean M. Carroll, ``Lecture Notes on General
  Relativity,'' arXiv:\texttt{gr-qc/9712019} (v1, 3 Dec 1997). (Includes
  black holes/geodesic applications used as baseline GR anchors.)
\end{enumerate}

\end{document}
